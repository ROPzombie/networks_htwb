%%%%%%%%%%%%%%%%%%%%%%%VICARIOUS%%%%%%%%%%%%%%%%%%%%%%%%%%%%%%%%%%%%%%%
% Copyright ME, FUCK YOU			      		      %
% Template for presentation in Latex`s Beamer Class		      %
% Using the default Berlin theme, can be replaced by other themes     %
% logo in the upper right can be replaced by new .png, gif, eps etc   %
% 								      %
%%%%%%%%%%%%%%%%%%%%%%%%%%%%%%%%%%%%%%%%%%%%%%%%%%%%%%%%%%%%%%%%%%%%%%%
\documentclass[xcolor=dvipsnames]{beamer}
\usetheme{Berlin}
\usecolortheme[named=LimeGreen]{structure}
\usepackage{beamerthemesplit} % kam neu dazu
\usepackage[ngerman]	{babel}			
\usepackage{t1enc}						
\usepackage[utf8]{inputenc}			
\usepackage{amsmath}
\usepackage{graphicx}
\graphicspath{{pictures/}}
\usepackage{amssymb}
\usepackage{amsfonts}
\usepackage{caption}
\usepackage{multimedia}
\usepackage{tikz}
\usepackage{listings}
\usepackage{acronym}
\usepackage{uhrzeit}

\usepackage{lmodern}
\usepackage{multicol}


\definecolor{pblue}{rgb}{0.13,0.13,1}
\definecolor{pgreen}{rgb}{0,0.5,0}
\definecolor{pred}{rgb}{0.9,0,0}
\definecolor{pgrey}{rgb}{0.46,0.45,0.48}

\lstset{
    escapeinside={(*}{*)}
}

\lstdefinestyle{Java}{
  showspaces=false,
  showtabs=false,
  tabsize=2,
  breaklines=true,
  showstringspaces=false,
  breakatwhitespace=true,
  commentstyle=\color{pgreen},
  keywordstyle=\color{pblue},
  stringstyle=\color{pred},
  basicstyle=\footnotesize\ttfamily,
  numbers=left,
  numberstyle=\tiny\color{gray}\ttfamily,
  numbersep=7pt,
  %moredelim=[il][\textcolor{pgrey}]{$$},
  moredelim=[is][\textcolor{pgrey}]{\%\%}{\%\%},
  captionpos=b
}

\lstdefinestyle{basic}{  
  basicstyle=\footnotesize\ttfamily,
  breaklines=true
  numbers=left,
  numberstyle=\tiny\color{gray}\ttfamily,
  numbersep=7pt,
  backgroundcolor=\color{white},
  showspaces=false,
  showstringspaces=false,
  showtabs=false,
  frame=single,
  rulecolor=\color{black},
  captionpos=b,
  keywordstyle=\color{blue}\bf,
  commentstyle=\color{gray},
  stringstyle=\color{green},
  keywordstyle={[2]\color{red}\bf},
}


\lstdefinelanguage{custom}
{
morekeywords={public, void},
sensitive=false,
morecomment=[l]{//},
morecomment=[s]{/*}{*/},
morestring=[b]",
}


\lstdefinestyle{BashInputStyle}{
  language=bash,
  showstringspaces=false,
  basicstyle=\small\sffamily,
  numbers=left,
  numberstyle=\tiny,
  numbersep=5pt,
  frame=trlb,
  columns=fullflexible,
  backgroundcolor=\color{gray!20},
  linewidth=0.9\linewidth,
  xleftmargin=0.1\linewidth
}

%Logo in the upper right just change if you know what you are doing^^
\addtobeamertemplate{frametitle}{}{%
\begin{tikzpicture}[remember picture,overlay]
\node[anchor=north east,yshift=2pt] at (current page.north east) {\includegraphics[height=1.8cm]{htw}};
\end{tikzpicture}}

\begin{document}
\bibliographystyle{alpha}
\title{Netzwerke -- Seminaristische Übung WS17/18}
\subtitle{Application Layer\\
		\href{mailto:Benjamin.Troester@HTW-Berlin.de}{Benjamin.Troester@HTW-Berlin.de}\\
		PGP: ADE1 3997 3D5D B25D 3F8F 0A51 A03A 3A24 978D D673 }
\author{Benjamin Tröster}

\date{\today}

\begin{frame}
\titlepage
\end{frame}

\section*{Road-Map}
\begin{frame}
\frametitle{Road-Map}
\begin{multicols}{2}
  \tableofcontents
\end{multicols}
\end{frame} 

\section*{Stuff}
\begin{frame}{Nerd-Wochenmarkt}
Empfehlung der Woche:
\begin{itemize}
	\item Chaos Communication Congress -- 34c3
	\begin{itemize}
		\item NIC: \url{https://media.ccc.de/v/34c3-9159-demystifying_network_cards}
		\item Hacker Jeopardy: \url{https://media.ccc.de/v/34c3-9007-hacker_jeopardy}   
	\end{itemize}
	\item Media CCC
	\begin{itemize}
		\item \url{https://media.ccc.de/c/34c3}
	\end{itemize}
\end{itemize}
\end{frame}

\section{Orga}
\begin{frame}
	\begin{itemize}
		\item Das Semester ist (fast) vorbei!
		\item D.h. das Testat steht an\dots
		\begin{itemize}
			\item 1. Gruppe -- 19.01.2018, \vonbis{15}{45}{19}{00}
			\item 2. Gruppe -- 15.01.2018, \vonbis{8}{00}{12}{00}
			\item Gruppe zu maximal vier Studierenden
			\item Seien Sie bitte pünktlich!
		\end{itemize}
		\item Zur Klausurvorbereitung werden diese Woche Übungsblätter Online gestellt (Mail an mich, wenn es Freitag nach 21 Uhr noch nicht online ist!)
		\item Rechnen Sie ausreichend Zeit für die Vorbereitung auf Klausuren etc. ein!
		\item Für die Übungsblätter -- \textasciitilde 1-2 Stunden (bei gutem Vorwissen), ohne 3-4 Stunden
		\item Bedarf an weiteren Aufgaben?
	\end{itemize}
\end{frame}

\section{Retrospektive}
\begin{frame}{Retrospektive}
\begin{itemize}
	\item Vorlesung
	\begin{itemize}
		\item Wo stehen Sie in den Vorlesung?
		\item Fragen?
	\end{itemize}
	\item Übungsblatt 4 \& 5 -- Routing
	\begin{itemize}
		\item Stand der Gruppen
		\item Fragen?
	\end{itemize}
\end{itemize}
\end{frame}

\section{Application Layer}
\subsection{SSH}
\begin{frame}
\centering
\includegraphics[scale=0.5]{ssh}
\end{frame}

\begin{frame}{SSH}
\begin{itemize}
	\item SSH -- Secure Shell
	\item Sammlung von Programmen/Diensten \& Protokolle zur sichere Netzwerkkommunikation
	\item Sicherung der Kommunikation durch:
	\begin{itemize}
		\item Kryptografie
	\end{itemize}
	\item Aufgaben:
	\begin{itemize}
		\item Verschlüsselung der Daten
		\item Integrität von Daten 
		\item Authentizität des Absenders
		\item Autorisierung -- nur Befugte könne die Daten einsehen
	\end{itemize}
\end{itemize}
\end{frame}

\begin{frame}
\begin{figure}
\center
\includegraphics[scale=0.25]{ssh2}
\end{figure}
\end{frame}

\subsection{Crypto}
\begin{frame}{}
\begin{itemize}
	\item Arten von Chiffren:
	\begin{itemize}
		\item Symmetrische Chiffren
		\begin{itemize}
			\item AES, Towfisch, 3DES, RC2, RC4, RC5, RC6, One-Time-Pad, Serpent, \dots
			\item Unterscheidung in Stromchiffre und Blockchiffre
			\item Verschiedene Verfahren haben unterschiedliche Modi -- CBC, EBC etc.
		\end{itemize}
		\item Asymmetrische Chiffren
			\item RSA, Merkle-Hellman, Diffie-Hellman, Elgamal, \dots
			\item Generierung eines Schlüsselpaars -- private \& public
			\item Funktionsweise aufgrund von mathematisch schwer lösbaren Problemen
			\item Faktorisierungsproblem, diskretes Wurzelziehen ($e$-te Wurzel $\mod N$), diskreter Logarithmus, \dots
	\end{itemize}
\end{itemize}
\end{frame}

\begin{frame}
	\begin{figure}
	\center
	\includegraphics[scale=0.2]{symmetric_cryptography}
	\end{figure}
\end{frame}


\begin{frame}
	\begin{figure}
	\center
	\includegraphics[scale=0.2]{private_keygeneration}
	\end{figure}
\end{frame}
\begin{frame}
	\begin{figure}
	\center
	\includegraphics[scale=0.2]{public_key_cryptography}
	\end{figure}
\end{frame}

\begin{frame}
	\begin{figure}
	\center
	\includegraphics[scale=0.2]{digital_signature}
	\end{figure}
\end{frame}


\end{document}