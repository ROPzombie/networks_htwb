%%%%%%%%%%%%%%%%%%%%%%%VICARIOUS%%%%%%%%%%%%%%%%%%%%%%%%%%%%%%%%%%%%%%%
% 																	%
% Template for presentation in Latex`s Beamer Class					%
% Using the default Berlin theme, can be replaced by other themes		%
% logo in the upper right can be replaced by new .png, gif, eps etc	%
% 																	%
%%%%%%%%%%%%%%%%%%%%%%%%%%%%%%%%%%%%%%%%%%%%%%%%%%%%%%%%%%%%%%%%%%%%%%%
\documentclass[xcolor=dvipsnames, aspectratio=169]{beamer}
\usetheme{Berlin}
\usecolortheme[named=LimeGreen]{structure}
\usepackage{beamerthemesplit} % kam neu dazu
\usepackage[ngerman]	{babel}			
\usepackage{t1enc}						
\usepackage[utf8]{inputenc}			
\usepackage{amsmath}
\usepackage{graphicx}
\graphicspath{{pictures/}}
\usepackage{amssymb}
\usepackage{amsfonts}
\usepackage{caption}
\usepackage{multimedia}
\usepackage{tikz}
\usepackage{listings}
\usepackage{acronym}
\usepackage{subfig}

\usepackage{lmodern}
\usepackage{multicol}

\definecolor{pblue}{rgb}{0.13,0.13,1}
\definecolor{pgreen}{rgb}{0,0.5,0}
\definecolor{pred}{rgb}{0.9,0,0}
\definecolor{pgrey}{rgb}{0.46,0.45,0.48}

\lstset{
    escapeinside={(*}{*)}
}

\lstdefinestyle{Java}{
  showspaces=false,
  showtabs=false,
  tabsize=2,
  breaklines=true,
  showstringspaces=false,
  breakatwhitespace=true,
  commentstyle=\color{pgreen},
  keywordstyle=\color{pblue},
  stringstyle=\color{pred},
  basicstyle=\footnotesize\ttfamily,
  numbers=left,
  numberstyle=\tiny\color{gray}\ttfamily,
  numbersep=7pt,
  %moredelim=[il][\textcolor{pgrey}]{$$},
  moredelim=[is][\textcolor{pgrey}]{\%\%}{\%\%},
  captionpos=b
}

\lstdefinestyle{basic}{  
  basicstyle=\footnotesize\ttfamily,
  breaklines=true
  numbers=left,
  numberstyle=\tiny\color{gray}\ttfamily,
  numbersep=7pt,
  backgroundcolor=\color{white},
  showspaces=false,
  showstringspaces=false,
  showtabs=false,
  frame=single,
  rulecolor=\color{black},
  captionpos=b,
  keywordstyle=\color{blue}\bf,
  commentstyle=\color{gray},
  stringstyle=\color{green},
  keywordstyle={[2]\color{red}\bf},
}


\lstdefinelanguage{custom}
{
morekeywords={public, void},
sensitive=false,
morecomment=[l]{//},
morecomment=[s]{/*}{*/},
morestring=[b]",
}


\lstdefinestyle{BashInputStyle}{
  language=bash,
  showstringspaces=false,
  basicstyle=\small\sffamily,
  numbers=left,
  numberstyle=\tiny,
  numbersep=5pt,
  frame=trlb,
  columns=fullflexible,
  backgroundcolor=\color{gray!20},
  linewidth=0.9\linewidth,
  xleftmargin=0.1\linewidth
}

%Logo in the upper right just change if you know what you are doing^^
\addtobeamertemplate{frametitle}{}{%
\begin{tikzpicture}[remember picture,overlay]
\node[anchor=north east,yshift=2pt] at (current page.north east) {\includegraphics[height=1.8cm]{htw}};
\end{tikzpicture}}

\begin{document}
\bibliographystyle{alpha}
\title{Netzwerke -- Übung WiSe 2018/19}
\subtitle{Einfach Netzwerke \& OSI Layer 2\\
		\href{mailto:Benjamin.Troester@HTW-Berlin.de}{Benjamin.Troester@HTW-Berlin.de}\\
		PGP: ADE1 3997 3D5D B25D 3F8F 0A51 A03A 3A24 978D D673 }
\author{Benjamin Tröster}

\date{}

\begin{frame}
\titlepage

\end{frame}

\section*{Road-Map}
\begin{frame}
\frametitle{Road-Map}
\begin{multicols}{2}
  \tableofcontents
\end{multicols}
\end{frame}

\section{Retrospektive}
\begin{frame}{Retrospektive}
\begin{itemize}
	\item Vorlesung
	\begin{itemize}
		\item Retrospektive der Vorlesung -- was haben Sie behandelt?
		\item Fragen?
	\end{itemize}
	\item Übungsblatt
	\begin{itemize}
		\item Stand des letzten Übungsblatts
		\item Fragen?
	\end{itemize}
	\item Präsentation des Theorieteils
\end{itemize}
\end{frame}


\subsection{Präsentation}
\begin{frame}
\frametitle{1.) Topologien}
	\begin{itemize}
		\item Erläutern Sie was eine Netzwerktopologie ist.
		\item Diskutieren Sie anhand einiger Beispiele verschiedene Topologien.
		\item Diskutieren Sie geeignete Netzwerktopologie für Ihr geplantes Netzwerk und Veranschaulichen Sie deren Umsetzung mit geeigneten Symbolen. (Nur die Topologie!)
	\end{itemize}
\end{frame}

\begin{frame}
\frametitle{2.) IP}
	\begin{itemize}
		\item Erläutern Sie was eine IP-Adresse ist und nennen Sie wesentliche Unterschiede zwischen IPv4 und IPv6.
		\item Nennen Sie die drei privaten IPv4-Adressbereiche!
		\item Erklären Sie was eine Subnetzmaske ist und wofür diese gebraucht wird. 
		\item Diskutieren Sie ein theoretische Umsetzung Ihres Netzwerkes aufgrund von IPv4 mit geeigneten IP-Adressen und Subnetzmasken. (Aufbauend auf Ihre Topologie)
	\end{itemize}
\end{frame}

\begin{frame}
\frametitle{3.) Daemons \& Tools I}
	\begin{itemize}
		\item Erläutern Sie was ein Daemon ist und nennen Sie einige wichtige Dienste.
		\item Erklären Sie die Syntax von Systemd (im speziellen \emph{systemctl}) beispielhaft. D.h. wie werden Dienste gestartet, gestoppt etc.
		\item Diskutieren Sie die Unterschiede der Werkzeugsammlungen \emph{iproute2} und der BSD \emph{net-tools}.
	\end{itemize}
\end{frame}

\begin{frame}
\frametitle{4.) Tools II}
	\begin{itemize}
		\item Erläutern Sie die Syntax der Tools \emph{ip addr} und \emph{ifconfig} zur Vergabe von IP-Adressen.
		\item Erklären Sie was \emph{ping} ist und wozu dies verwendet wird.
		\item Erläutern Sie die Syntax des Tools \emph{ping} beispielhaft.
	\end{itemize}
\end{frame}

\end{document}