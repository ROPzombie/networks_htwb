%%%%%%%%%%%%%%%%%%%%%%%VICARIOUS%%%%%%%%%%%%%%%%%%%%%%%%%%%%%%%%%%%%%%%
%											%
% Template for presentation in Latex`s Beamer Class					%
% Using the default Berlin theme, can be replaced by other themes		%
% logo in the upper right can be replaced by new .png, gif, eps etc	%
% 																		%
%%%%%%%%%%%%%%%%%%%%%%%%%%%%%%%%%%%%%%%%%%%%%%%%%%%%%%%%%%%%%%%%%%%%%%%
\documentclass[xcolor=dvipsnames,aspectratio=169]{beamer}
\usetheme{Berlin}
\usecolortheme[named=LimeGreen]{structure}
\usepackage{beamerthemesplit} % kam neu dazu
\usepackage[ngerman]	{babel}			
\usepackage{t1enc}						
\usepackage[utf8]{inputenc}			
\usepackage{amsmath}
\usepackage{graphicx}
\graphicspath{{pictures/}}
\usepackage{amssymb}
\usepackage{amsfonts}
\usepackage{caption}
\usepackage{multimedia}
\usepackage{tikz}
\usepackage{listings}
\usepackage{acronym}
\usepackage{uhrzeit}
\usepackage{lmodern}
\usepackage{multicol}


\definecolor{pblue}{rgb}{0.13,0.13,1}
\definecolor{pgreen}{rgb}{0,0.5,0}
\definecolor{pred}{rgb}{0.9,0,0}
\definecolor{pgrey}{rgb}{0.46,0.45,0.48}

\newcounter{countitems}
\newcounter{nextitemizecount}
\newcommand{\setupcountitems}{%
  \stepcounter{nextitemizecount}%
  \setcounter{countitems}{0}%
  \preto\item{\stepcounter{countitems}}%
}
\makeatletter
\newcommand{\computecountitems}{%
  \edef\@currentlabel{\number\c@countitems}%
  \label{countitems@\number\numexpr\value{nextitemizecount}-1\relax}%
}
\newcommand{\nextitemizecount}{%
  \getrefnumber{countitems@\number\c@nextitemizecount}%
}
\newcommand{\previtemizecount}{%
  \getrefnumber{countitems@\number\numexpr\value{nextitemizecount}-1\relax}%
}

\makeatother    
\newenvironment{AutoMultiColItemize}{%
\ifnumcomp{\nextitemizecount}{>}{3}{\begin{multicols}{2}}{}%
\setupcountitems\begin{itemize}}%
{\end{itemize}%
\unskip\computecountitems\ifnumcomp{\previtemizecount}{>}{3}{\end{multicols}}{}}

\lstset{
    escapeinside={(*}{*)}
}

\lstdefinestyle{Java}{
  showspaces=false,
  showtabs=false,
  tabsize=2,
  breaklines=true,
  showstringspaces=false,
  breakatwhitespace=true,
  commentstyle=\color{pgreen},
  keywordstyle=\color{pblue},
  stringstyle=\color{pred},
  basicstyle=\footnotesize\ttfamily,
  numbers=left,
  numberstyle=\tiny\color{gray}\ttfamily,
  numbersep=7pt,
  %moredelim=[il][\textcolor{pgrey}]{$$},
  moredelim=[is][\textcolor{pgrey}]{\%\%}{\%\%},
  captionpos=b
}

\lstdefinestyle{basic}{  
  basicstyle=\footnotesize\ttfamily,
  breaklines=true
  numbers=left,
  numberstyle=\tiny\color{gray}\ttfamily,
  numbersep=7pt,
  backgroundcolor=\color{white},
  showspaces=false,
  showstringspaces=false,
  showtabs=false,
  frame=single,
  rulecolor=\color{black},
  captionpos=b,
  keywordstyle=\color{blue}\bf,
  commentstyle=\color{gray},
  stringstyle=\color{green},
  keywordstyle={[2]\color{red}\bf},
}


\lstdefinelanguage{custom}
{
morekeywords={public, void},
sensitive=false,
morecomment=[l]{//},
morecomment=[s]{/*}{*/},
morestring=[b]",
}


\lstdefinestyle{BashInputStyle}{
  language=bash,
  showstringspaces=false,
  basicstyle=\small\sffamily,
  numbers=left,
  numberstyle=\tiny,
  numbersep=5pt,
  frame=trlb,
  columns=fullflexible,
  backgroundcolor=\color{gray!20},
  linewidth=0.9\linewidth,
  xleftmargin=0.1\linewidth
}

%Logo in the upper right just change if you know what you are doing^^
\addtobeamertemplate{frametitle}{}{%
\begin{tikzpicture}[remember picture,overlay]
\node[anchor=north east,yshift=2pt] at (current page.north east) {\includegraphics[height=1.8cm]{htw}};
\end{tikzpicture}}

\begin{document}
\bibliographystyle{alpha}
\title{Netzwerke -- Übung SoSe 2019}
\subtitle{Organisatorisches\\

\href{mailto:Benjamin.Troester@HTW-Berlin.de}{Benjamin.Troester@HTW-Berlin.de}\\
		PGP: ADE1 3997 3D5D B25D 3F8F 0A51 A03A 3A24 978D D673 }

\author{Benjamin Tröster}

\date{}

\begin{frame}
\titlepage
\end{frame}

\section*{Road-Map}
\begin{frame}
\frametitle{Road-Map}
\begin{multicols}{2}
  \tableofcontents
\end{multicols}
\end{frame}

\section*{About Me}
\begin{frame}
\frametitle{About Me}
\begin{itemize}
	\item Informatik@FU-Berlin
	\item Forschungsgebiet:
	\begin{itemize}
		\item Mitglied @AG Security Engineering
		\item IT-Security
		\begin{itemize}
			\item CtF (Capture the Flag -- Hacking Contests)
			\item Vulnerability-Exploitation (PoC)
			\item Reverse-Engineering
		\end{itemize}
		\item Betriebssysteme
		\begin{itemize}
			\item BSD, Linux, Solaris, $\mu$-Kernel OS
			\item HPC-Computing \& Scheduling
		\end{itemize}
		\item Computernetzwerke \& Netzwerktechnik
	\end{itemize}
\end{itemize} 
\end{frame}

\section{Disclamer}
\subsection{Hinweis}
\begin{frame}
	\frametitle{Disclaimer I}
	\begin{itemize}
		\item Das Studium der (Angewandten) Informatik ist anspruchsvoll...
		\item Das Gebiet Netzwerke ist äußert umfangreich $\rightarrow$ Arbeitsaufwand
		\item Um nicht den Anschluss zu verlieren:
		\begin{itemize}
			\item Sollten Sie sich auf Vorlesung \& Übung vorbereiten
			\item Literatur/Links etc. lesen bzw. recherchieren
			\item Sich Notizen anfertigen -- VL, Übung, Hausaufgaben, vorbereitend Fragen formulieren
			\item Sich Unklarheiten notieren
			\item \textbf{Fragen}, wenn Sie etwas nicht verstehen 
	\end{itemize}		
		\item Bearbeiten Sie die Übungsblätter -- Hausaufgaben, Praxis, klausurvorbereitend
		\item Sum­ma sum­ma­rum: Beschäftigen Sie sich ausreichend mit den Inhalten!
	\end{itemize}
\end{frame}
\subsection{Nutzungsrechte \& Datenschutz}
\begin{frame}
	\frametitle{Disclaimer II}
	\begin{itemize}
		\item Folien, Arbeits-/Übungsblätter bitte nicht via Dropbox, Share-Hoster, Facebook etc. teilen
		\item Nutzungsrechte nur für dieses Semester innerhalb der Veranstaltung Netzwerke
		\item Keine Photo-, Video-/Audiomitschnitte in der Übung
		\item Fragen per Mail nur via HTW-Mail-Account
	\end{itemize}
\end{frame}

\subsection{Labor \& Geräte}
\begin{frame}
	\frametitle{Disclamer III}
	\begin{itemize}
		\item Halten Sie bitte Ordnung! D.h.:
		\begin{itemize}
			\item Räumen Sie \textbf{alle} verwendeten Geräte wieder weg!
			\item Seien Sie sozial, falls Kommilitonen dies vergessen.
			\item Achten Sie darauf die Raspberry Pis vorsichtig zu behandeln $\rightarrow$ Leihgabe
			\item Falls Geräte defekt sind: bitte melden!
		\end{itemize}
		\item Versuchen Sie nicht erst $X$ Minuten später zu erscheinen! 
		\item Sie arbeiten in Gruppen, was es nicht einfacher macht
	\end{itemize}
\end{frame}

\section{Organisatorisches}
\subsection{Übungsablauf}
\begin{frame}
	\frametitle{Übungsablauf I}
	\begin{itemize}
		\item Labor WH C 625
		\item Übungsgruppe \& -zeit $\Rightarrow$ LSF
		\begin{itemize}
			\item regulär Mittwochs: 
			\item Zug 2: gerade/ungerade Kalenderwoche \vonbis{8}{00}{11}{15}.
			\item Zug 1, gerade/ungerade Kalenderwoche \vonbis{12}{15}{15}{30}
		\end{itemize}
		\item Übungsblätter, Folien, Literatur \& Links etc. $\Rightarrow$ \url{moodle.htw-berlin.de}
		\item Zweiwöchentlicher Tonus 
		\begin{itemize}
			\item \textasciitilde 6 Übungsblätter -- bestehend aus Theorie- \& Laborteil
			\item Jedes Übung in sich abgeschlossen
			\item \textbf{Aber:} Übungen bauen aufeinander auf
		\end{itemize}
	\end{itemize}
\end{frame}

\begin{frame}
	\frametitle{Übungsablauf II}
	\vspace{-0.8cm}
	\begin{itemize}
		\item Woche ohne Laborübung -- Hausaufgaben
		\begin{itemize}
			\item Recherche, Literatur, Hintergrundwissen -- Grundlagenwissen aufbauen
			\item Planungsphase -- Erste Verbindung von Theorie in die Praxis
			\begin{itemize}
				\item Lesen Sie nach den Hausaufgaben die kommenden Laboraufgaben!
			\end{itemize}
			\item Zeit Wissen zu verarbeiten \& Fragen vorzubereiten
			\item Sinn:
			\begin{itemize}
				\item Laborzeit effizienter nutzen
				\item Erlernen von Arbeitsstrategien
				\item Vorbereitung
			\end{itemize}
		\end{itemize}
		\item Praktische Laborübung
		\begin{itemize}
			\item Diskussion der theoretischen Ausarbeitungen im Plenum, als auch in Ihrer Gruppe bzw. mit Ihren Kommilitonen
			\item Umsetzung der Planung
			\item Dokumentation der Umsetzung
		\end{itemize}
	\end{itemize}
\end{frame}

\begin{frame}
	\frametitle{Übungsablauf III}
	\begin{itemize}
		\item Beginn der Übung:
		\begin{itemize}
			\item Retrospektive -- Fragen zur Vorlesung bzw. der letzten Übung
			\item Fragen zu den aktuellen Haus- bzw. Laboraufgaben 
			\item Präsentation \& Diskussion einiger Hausaufgaben (s. \nameref{klausurzulassung})
		\end{itemize}
		\item Kurze Diskussion der Lösungsansätze (Hausaufgaben) der Gruppe
		\item Umsetzung des theoretischen Teils in die Praxis
		\item Ende der Übung: Bearbeitungsstand der aktuelle Übung
		\item Preview auf nächste Übung \& Vorstellung des neuen Übungsblatts  
	\end{itemize}
\end{frame}

\subsection{ECTS}
\begin{frame}
	\frametitle{Arbeitsaufwand -- ECTS}
	\begin{itemize}
		\item Modul Netzwerke: 5 ECTS (European Credit Transfer System) -- manchmal auch LP (Leistungspunkte)
			\begin{itemize}
				\item 1 ECTS $\widehat{=}$ 30h
				\item Workload Netzwerke: 150h/Semester oder 37,5h/Monat oder $\sim$ 9,375h/Woche
				\item 2 SWS \footnote{Semesterwochenstunden} VL + 2 SWS Übung $\rightarrow$ 4 SWS oder 3h
				\item D.h. restlichen 6h/Woche Vorbereitung \& Nachbereitung, Hausaufgaben etc.
			\end{itemize}
	\end{itemize}
\end{frame}
\subsection{Klausur \& Klausurzulassung} \label{klausurzulassung}
\begin{frame}
	\frametitle{Klausurzulassung}
	\begin{itemize}
		\item Für die Klausurzulassung muss jede(r) Studierende zwei Protokolle abgegeben haben:
		\begin{itemize}
			\item Obligatorisch: in Gruppen von maximal vier Studierenden Aufgabenblatt drei -- Routing
			\item Fakulativ: eine Teilaufgabe aus den Blättern vier bis sechs
			\item \textbf{Autor ist wer aktiv an der Übung und an der Nachbesprechung teilgenommen hat!}
			\item Ein Beispielprotokoll wird zum Aufgabenblatt zwei als Hilfestellung bereitgestellt
			\item Abgabe der Protokolle $\rightarrow$ Moodle!
			\item Die Protokolle gehen zu 10\% in die Endnote ein.
		\end{itemize}
		\item \textbf{Wiederholer} müssen erneut eine Klausurzulassung erwerben, alte Klausurzulassungen gelten nicht mehr!
		\begin{itemize}
			\item Sie müssen sich via LSF oder Handzettel selbst um eine Kurszulassung kümmern!
		\end{itemize}
	\end{itemize}
\end{frame}

\section{Motivation}
\subsection{Studien- \& Prüfungsordnung AMB30/2012}
\begin{frame}
	\frametitle{Motivation}
	Studien- \& Prüfungsordnung AMB30/2012 S. 375\\
	B12 Netzwerke
	\begin{itemize}
		\item Die Studierenden erwerben Kenntnisse wichtiger Netzwerkprotokolle und -dienste
		\item Sie erwerben Fertigkeiten im Aufbau von Rechnernetzen
		\item Sie erwerben Grundlagen sicherer Netzwerkkommunikation
		\item Sie können einfache Netzwerke realisieren
	\end{itemize}
\end{frame}
\begin{frame}
	\frametitle{Motivation}
	\vspace{-0.7cm}
	Etwas genauer:\\
	Studienordnung des Bachelorstudiengangs Angewandte Informatik -- B12 Netzwerke:
	\begin{itemize}
		\item Die Studierenden erwerben Kenntnisse wichtiger Netzwerkprotokolle und -dienste
		\item Sie erwerben Fertigkeiten im Aufbau von Rechnernetzen
		\item Sie erwerben Grundlagen sicherer Netzwerkkommunikation
		\item Sie können einfache Netzwerke realisieren
		\item Grundlagen der Systemverwaltung
		\item OSI-Referenzmodell
		\item Netzwerkprotokolle TCP, UDP, IP, ...
		\item Routing
		\item Name Service
		\item HTTP
	\end{itemize}
\end{frame}

\subsection{Netzwerktechnik in anderen Modulen}
\begin{frame}
\frametitle{Motivation II}
	\begin{itemize}
		\item Netzwerke $\rightarrow$ Fundament der digitalen Infrastruktur
		\item Grundlagen der Veranstaltungen Netzwerke sind Grundlagen anderer Kurse
		 \begin{AutoMultiColItemize}
 			\item[1.] Netzwerke
 			\item[2.] Betriebssysteme -- Client-Server-Programmierung, Sockets, Grid-Computing...
 			\item[3.] Programmieren 3 \& Datenbanken -- Sockets, JDBC...
 			\item[4.] Webentwicklung \& Verteilte Systeme -- MVC-Pattern, Messaging...
 			\item[5.] KBE, Spez. Anwd. \& Projektstudium -- JSF, JPA, CDI...
 			\item[6.] Möglicherweise Abschlussarbeit
  \end{AutoMultiColItemize}
	\end{itemize}
\end{frame}

\end{document}