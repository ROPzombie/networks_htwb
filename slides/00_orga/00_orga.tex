\documentclass[xcolor=dvipsnames,aspectratio=169]{beamer}

\usepackage{fontspec}
\defaultfontfeatures{Ligatures=TeX}
%\setsansfont{Liberation Sans}
\usepackage{polyglossia}
\setdefaultlanguage{german}

\usetheme{Berlin}
\usecolortheme[named=LimeGreen]{structure}
\usepackage{beamerthemesplit} % kam neu dazu
	
\usepackage{amsmath}
\usepackage{graphicx}
\graphicspath{{pictures/}}
\usepackage{amssymb}
\usepackage{amsfonts}
\usepackage{caption}
\usepackage{multimedia}
\usepackage{tikz}
\usepackage{listings}
\usepackage{acronym}
\usepackage{uhrzeit}
\usepackage{lmodern}
\usepackage{multicol}


\definecolor{pblue}{rgb}{0.13,0.13,1}
\definecolor{pgreen}{rgb}{0,0.5,0}
\definecolor{pred}{rgb}{0.9,0,0}
\definecolor{pgrey}{rgb}{0.46,0.45,0.48}

\newcounter{countitems}
\newcounter{nextitemizecount}
\newcommand{\setupcountitems}{%
  \stepcounter{nextitemizecount}%
  \setcounter{countitems}{0}%
  \preto\item{\stepcounter{countitems}}%
}
\makeatletter
\newcommand{\computecountitems}{%
  \edef\@currentlabel{\number\c@countitems}%
  \label{countitems@\number\numexpr\value{nextitemizecount}-1\relax}%
}
\newcommand{\nextitemizecount}{%
  \getrefnumber{countitems@\number\c@nextitemizecount}%
}
\newcommand{\previtemizecount}{%
  \getrefnumber{countitems@\number\numexpr\value{nextitemizecount}-1\relax}%
}

\makeatother    
\newenvironment{AutoMultiColItemize}{%
\ifnumcomp{\nextitemizecount}{>}{3}{\begin{multicols}{2}}{}%
\setupcountitems\begin{itemize}}%
{\end{itemize}%
\unskip\computecountitems\ifnumcomp{\previtemizecount}{>}{3}{\end{multicols}}{}}

\lstset{
    escapeinside={(*}{*)}
}

\lstdefinestyle{Java}{
  showspaces=false,
  showtabs=false,
  tabsize=2,
  breaklines=true,
  showstringspaces=false,
  breakatwhitespace=true,
  commentstyle=\color{pgreen},
  keywordstyle=\color{pblue},
  stringstyle=\color{pred},
  basicstyle=\footnotesize\ttfamily,
  numbers=left,
  numberstyle=\tiny\color{gray}\ttfamily,
  numbersep=7pt,
  %moredelim=[il][\textcolor{pgrey}]{$$},
  moredelim=[is][\textcolor{pgrey}]{\%\%}{\%\%},
  captionpos=b
}

\lstdefinestyle{basic}{  
  basicstyle=\footnotesize\ttfamily,
  breaklines=true
  numbers=left,
  numberstyle=\tiny\color{gray}\ttfamily,
  numbersep=7pt,
  backgroundcolor=\color{white},
  showspaces=false,
  showstringspaces=false,
  showtabs=false,
  frame=single,
  rulecolor=\color{black},
  captionpos=b,
  keywordstyle=\color{blue}\bf,
  commentstyle=\color{gray},
  stringstyle=\color{green},
  keywordstyle={[2]\color{red}\bf},
}


\lstdefinelanguage{custom}
{
morekeywords={public, void},
sensitive=false,
morecomment=[l]{//},
morecomment=[s]{/*}{*/},
morestring=[b]",
}


\lstdefinestyle{BashInputStyle}{
  language=bash,
  showstringspaces=false,
  basicstyle=\small\sffamily,
  numbers=left,
  numberstyle=\tiny,
  numbersep=5pt,
  frame=trlb,
  columns=fullflexible,
  backgroundcolor=\color{gray!20},
  linewidth=0.9\linewidth,
  xleftmargin=0.1\linewidth
}

%Logo in the upper right just change if you know what you are doing^^
\addtobeamertemplate{frametitle}{}{%
\begin{tikzpicture}[remember picture,overlay]
\node[anchor=north east,yshift=2pt] at (current page.north east) {\includegraphics[height=1.8cm]{htw}};
\end{tikzpicture}}

\begin{document}
\bibliographystyle{alpha}
\title{Netzwerke -- Übung Wintersemester 2020/21}
\subtitle{Organisatorisches\\

\href{mailto:Benjamin.Troester@HTW-Berlin.de}{Benjamin.Troester@HTW-Berlin.de}\\
		PGP: ADE1 3997 3D5D B25D 3F8F 0A51 A03A 3A24 978D D673 }

\author{Benjamin Tröster}

\date{}

\begin{frame}
\titlepage
\end{frame}

\section*{Road-Map}
\begin{frame}
\frametitle{Road-Map}
\begin{multicols}{2}
  \tableofcontents
\end{multicols}
\end{frame}

\section{Disclamer}
\begin{frame}
	\frametitle{Disclamer I}
	\begin{itemize}
		\item Objektive Kritik ist immer willkommen!
		\item Wenn sie Fehler oder Anmerkungen haben, tragen sie diese ruhig vor.
		\item Sie wollen ehrliches Feedback genauso wie ich
		\item Wenn Sie Anregungen, Verbesserungsvorschläge haben, nehme ich diese ebenfalls gerne an.
	\end{itemize}
\end{frame}

\section{Organisatorisches}
\subsection{Grundsätzliches}
\begin{frame}
	\frametitle{Grundsätzliches}
	\begin{itemize}
		\item Übungen: ab dem 02.11.2020 via Distanzlehre
		\item Vorlesung: ab dem 03/04.11.2020 via Distanzlehre
		\item Meeting/Übung via Moodle: Telco-Plugin \emph{BigBlueButton (BBB)}
		\begin{itemize}
			\item Falls BBB nicht skaliert -- Zoom
		\end{itemize}
		\item Übungsgruppe \& -zeit $\Rightarrow$ LSF
		\item Übungsblätter, Folien, Literatur \& Links etc. $\Rightarrow$ \url{moodle.htw-berlin.de}
		\item Zweiwöchentlicher Tonus 
		\begin{itemize}
			\item \textasciitilde 5-6 Übungsblätter -- bestehend aus Theorie- \& Laborteil
			\item Jede Übung in sich abgeschlossen
			\item \textbf{Aber:} Übungen bauen aufeinander auf
		\end{itemize}
	\end{itemize}
\end{frame}

\begin{frame}
	\frametitle{Klausur/Mündliche Prüfung}
	\begin{itemize}
		\item How to Pass:
		\begin{itemize}
			\item Dieses Semester: keine Klausurzulassung
			\item Keine Anwesenheitspflicht
			\item Ende des Semesters: mündliche Prüfung (50\% Übung, 50\% Vorlesung)
			\item Mündliche Prüfung: sehr praxisnah, klare Verbindung von VL und Ü
			\begin{itemize}
				\item Dauer ca. 30 Minuten, in Gruppen von 4-5 Studierenden
			\end{itemize}
			\item Das letzte Semester zeigt, dass sich die Anwesenheit in der Übung lohnt!
		\end{itemize}
	\end{itemize}
\end{frame}

\begin{frame}
	\frametitle{Übungsablauf}
	\begin{itemize}
		\item Beginn der Übung Nachbesprechung:
		\begin{itemize}
			\item Vorlesung
			\item Letzte Übung
		\end{itemize}
		\item Besprechung der Hausaufgaben
		\begin{itemize}
			\item Allgemeine Frage im Plenum für alle
			\item Fragen zum Laborübungsblatt
		\end{itemize}
		\item Umsetzung: Laboraufgaben
		\begin{itemize}
			\item Aufteilung in Kleingruppen via Breakout-Rooms
		\end{itemize}
	\end{itemize}
\end{frame}

\subsection{ECTS}
\begin{frame}
	\frametitle{Arbeitsaufwand -- ECTS}
	\begin{itemize}
		\item Modul Netzwerke: 5 ECTS (European Credit Transfer System) -- manchmal auch LP/CP (Leistungspunkte)
			\begin{itemize}
				\item 1 ECTS $\widehat{=}$ 30h
				\item Workload Netzwerke: 150h/Semester oder 37,5h/Monat oder $\sim$ 9,375h/Woche
				\item 2 SWS \footnote{Semesterwochenstunden} VL + 2 SWS Übung $\rightarrow$ 4 SWS oder 3h
				\item D.h. restlichen \textbf{6h/Woche}
				\begin{itemize}
					\item Vorbereitung \& Nachbereitung der Vorlesung
					\item Hausaufgaben, Recherche...
				\end{itemize}
			\end{itemize}
			\item Total Workload pro Semester: 30 CP $\widehat{=}$ 900h $\rightarrow$  $\sim$56.25/Woche
	\end{itemize}
\end{frame}

\section{Motivation}
\subsection{Studien- \& Prüfungsordnung AMB30/2012}
\begin{frame}
	\frametitle{Was soll bei diesem Kurs raus springen?}
	Studien- \& Prüfungsordnung AMB30/2012 S. 375\\
	B12 Netzwerke
	\begin{itemize}
		\item Die Studierenden erwerben Kenntnisse wichtiger Netzwerkprotokolle und -dienste
		\item Sie erwerben Fertigkeiten im Aufbau von Rechnernetzen
		\item Sie erwerben Grundlagen sicherer Netzwerkkommunikation
		\item Sie können einfache Netzwerke realisieren
	\end{itemize}
\end{frame}

\begin{frame}
	\frametitle{Was soll bei diesem Kurs raus springen?}
	\vspace{-0.7cm}
	Etwas genauer:\\
	Studienordnung des Bachelorstudiengangs Angewandte Informatik -- B12 Netzwerke:
	\begin{itemize}
		\item Die Studierenden erwerben Kenntnisse wichtiger Netzwerkprotokolle und -dienste
		\item Sie erwerben Fertigkeiten im Aufbau von Rechnernetzen
		\item Sie erwerben Grundlagen sicherer Netzwerkkommunikation
		\item Sie können einfache Netzwerke realisieren
		\item Grundlagen der Systemverwaltung
		\item OSI-Referenzmodell
		\item Netzwerkprotokolle TCP, UDP, IP, ...
		\item Routing
		\item Name Service
		\item HTTP
	\end{itemize}
\end{frame}

\subsection{Netzwerktechnik in anderen Modulen}
\begin{frame}
\frametitle{Was soll bei diesem Kurs raus springen?}
	\begin{itemize}
		\item Netzwerke $\rightarrow$ Fundament der digitalen Infrastruktur
		\item Grundlagen der Veranstaltungen Netzwerke sind Grundlagen anderer Kurse
		 \begin{AutoMultiColItemize}
 			\item[1.] Netzwerke
 			\item[2.] Betriebssysteme -- Client-Server-Programmierung, Sockets, Grid-Computing...
 			\item[3.] Programmieren 3 \& Datenbanken -- Sockets, JDBC...
 			\item[4.] Webentwicklung \& Verteilte Systeme -- MVC-Pattern, Messaging...
 			\item[5.] KBE, Spez. Anwd. \& Projektstudium -- JSF, JPA, CDI...
 			\item[6.] Möglicherweise Abschlussarbeit
  		\end{AutoMultiColItemize}
	\end{itemize}
\end{frame}

\end{document}