%%%%%%%%%%%%%%%%%%%%%%%VICARIOUS%%%%%%%%%%%%%%%%%%%%%%%%%%%%%%%%%%%%%%%
% 																	%
% Template for presentation in Latex`s Beamer Class					%
% Using the default Berlin theme, can be replaced by other themes		%
% logo in the upper right can be replaced by new .png, gif, eps etc	%
% 																	%
%%%%%%%%%%%%%%%%%%%%%%%%%%%%%%%%%%%%%%%%%%%%%%%%%%%%%%%%%%%%%%%%%%%%%%%
\documentclass[xcolor=dvipsnames, aspectratio=169]{beamer}
\usetheme{Berlin}
\usecolortheme[named=LimeGreen]{structure}
\usepackage{beamerthemesplit} % kam neu dazu
\usepackage[ngerman]	{babel}			
\usepackage{t1enc}						
\usepackage[utf8]{inputenc}			
\usepackage{amsmath}
\usepackage{graphicx}
\graphicspath{{pictures/}}
\usepackage{amssymb}
\usepackage{amsfonts}
\usepackage{caption}
\usepackage{multimedia}
\usepackage{tikz}
\usepackage{listings}
\usepackage{acronym}
\usepackage{subfig}

\usepackage{lmodern}
\usepackage{multicol}

\definecolor{pblue}{rgb}{0.13,0.13,1}
\definecolor{pgreen}{rgb}{0,0.5,0}
\definecolor{pred}{rgb}{0.9,0,0}
\definecolor{pgrey}{rgb}{0.46,0.45,0.48}

\lstset{
    escapeinside={(*}{*)}
}

\lstdefinestyle{Java}{
  showspaces=false,
  showtabs=false,
  tabsize=2,
  breaklines=true,
  showstringspaces=false,
  breakatwhitespace=true,
  commentstyle=\color{pgreen},
  keywordstyle=\color{pblue},
  stringstyle=\color{pred},
  basicstyle=\footnotesize\ttfamily,
  numbers=left,
  numberstyle=\tiny\color{gray}\ttfamily,
  numbersep=7pt,
  %moredelim=[il][\textcolor{pgrey}]{$$},
  moredelim=[is][\textcolor{pgrey}]{\%\%}{\%\%},
  captionpos=b
}

\lstdefinestyle{basic}{  
  basicstyle=\footnotesize\ttfamily,
  breaklines=true
  numbers=left,
  numberstyle=\tiny\color{gray}\ttfamily,
  numbersep=7pt,
  backgroundcolor=\color{white},
  showspaces=false,
  showstringspaces=false,
  showtabs=false,
  frame=single,
  rulecolor=\color{black},
  captionpos=b,
  keywordstyle=\color{blue}\bf,
  commentstyle=\color{gray},
  stringstyle=\color{green},
  keywordstyle={[2]\color{red}\bf},
}


\lstdefinelanguage{custom}
{
morekeywords={public, void},
sensitive=false,
morecomment=[l]{//},
morecomment=[s]{/*}{*/},
morestring=[b]",
}


\lstdefinestyle{BashInputStyle}{
  language=bash,
  showstringspaces=false,
  basicstyle=\small\sffamily,
  numbers=left,
  numberstyle=\tiny,
  numbersep=5pt,
  frame=trlb,
  columns=fullflexible,
  backgroundcolor=\color{gray!20},
  linewidth=0.9\linewidth,
  xleftmargin=0.1\linewidth
}

%Logo in the upper right just change if you know what you are doing^^
\addtobeamertemplate{frametitle}{}{%
\begin{tikzpicture}[remember picture,overlay]
\node[anchor=north east,yshift=2pt] at (current page.north east) {\includegraphics[height=1.8cm]{htw}};
\end{tikzpicture}}

\begin{document}
\bibliographystyle{alpha}
\title{Netzwerke -- Übung SoSe 2019}
\subtitle{Application Layer\\
		\href{mailto:Benjamin.Troester@HTW-Berlin.de}{Benjamin.Troester@HTW-Berlin.de}\\
		PGP: ADE1 3997 3D5D B25D 3F8F 0A51 A03A 3A24 978D D673 }
\author{Benjamin Tröster}

\date{}

\begin{frame}
\titlepage

\end{frame}

\section*{Road-Map}
\begin{frame}
\frametitle{Road-Map}
\begin{multicols}{2}
  \tableofcontents
\end{multicols}
\end{frame}

\section{Aktueller Stand}
\begin{frame}
	\frametitle{Aktueller Stand}
	\begin{itemize}
		\item Infrastruktur steht! Sie haben verschied komplexe Netzwerke aufgebaut.
		\item Einen Uplink ins Internet eingerichtet (inklusive NAT).
		\item DNS konfiguriert
		\item Erste Protokolle analysiert -- via Wireshark
		\item ARP, HTTP und die darunter liegenden Protokolle bereits kennengelernt.
	\end{itemize}
\end{frame}

\section{Traceroute}
\begin{frame}
	\frametitle{Traceroute \& Paris-Traceroute}
	\begin{itemize}
		\item Ermittelt via ICMP-Echo-Requests den Weg eines Paketes von der Quelle zum Ziel
		\item Nutzt hierfür das TTL-Feld des IP-Headers
		\item Somit kann der Weg eines Paketes mitverfolgt werden
		\item Problem: Traceroute kann keine Routen finden, wenn Router Load-Balancing anwenden
		\item Paris-Traceroute kann mit den durch das Load-Balancing entstehenden Anomalien umgehen    
	\end{itemize}
\end{frame}

\section{DNS}
\begin{frame}
	\frametitle{DNS}
	\begin{itemize}
		\item Domain Name System: Mapping von Domain Name auf IP-Adressen -- besserer mnemonisch Effekt
		\item DNS nutzt UDP für den Transport, Standardport: 53
		\item DNS bietet für verschiedene Dienste unterschiedliche Record-Types
		\begin{multicols}{2}
		\begin{itemize}
			\item A-Record: IPv4 
			\item AAAA-Record: IPv6
			\item CNAME: Verweis auf anderen Name
			\item MX-Record: Name für Mailserver
			\item PTR-Ressource-Record: Reverse Lookup
			\item TXT Record: Zuweisung Name auf beliebigen Text
		\end{itemize}
		\end{multicols}
		\item Abfrage von Records via \emph{dig}, \emph{nsloopup}, \emph{host}
		\item Zusätzlich: \emph{whois}
	\end{itemize}
\end{frame}

\section{HTTP(S) \& HTML}
\begin{frame}
	\frametitle{URL \& URI}
	\begin{itemize}
		\item URI -- Uniform Resource Identifier, Identificator: dient dem Auffinden von Ressourcen
		\item URL -- Uniform Resource Locator, zu Identifizierung von Ressourcen
		\item Legt Zugriffsmethode und Ort fest
		\item URL ist eine spezielle Ausformung der URI
	\end{itemize}
\end{frame}

\begin{frame}
	\frametitle{HTTP(S) \& HTML}
	\begin{itemize}
		\item Zustandsloses Protokoll des Application Layers
		\item HTTP nutzt zumeist verbindungsorientiertes Transportprotokoll: \emph{TCP}, \emph{MPTCP}, \emph{Quic} etc
		\item Standardport: 80, Verschlüsselt: 443
	\end{itemize}
\end{frame}

\subsection{HTTP Methoden}
\begin{frame}
	\frametitle{HTTP Methoden}
	\begin{itemize}
		\item HTTP besitzt Methoden:
		\begin{itemize}
			\item GET: Anfordern von Ressourcen
			\item POST: Sendet Daten an Ressource
			\item OPTIONS: Informationen über Kommunikationsoptionen (Beschränkungen, Proxies, etc.)
			\item HEAD: spezielles GET, fordert nur den Header an
			\item PUT: Modifikation bestehender Daten auf dem Server
			\item DELETE: löschen von Daten auf dem Server durch URL
			\item TRACE: Requests von Clients verfolgen
			\item CONNECT: reserviere Verbindung -- für Tunneling
		\end{itemize} 
	\end{itemize}
\end{frame}

\subsection{Einschub: SSL/TLS}
\begin{frame}
	\frametitle{Einschub: SSL/TLS mit openSSL oder libreSSL}
	\begin{itemize}
		\item Secure Sockets Layer (SSL) \& Transport Layer Security (TLS)
		\item SSL/TLS ermöglicht eine gesicherte Kommunikation durch Kryptographie
		\item Schützt die Protokolle des Transport Layers, d.h. TCP, UDP, ... 
		\item Im wesentliche drei Teile:
		\begin{itemize}
			\item Zertifikate: sorgen für Sicherstellung der PKI, Authentizität, Integrität
			\item PKI nutzt asymmetrische Chiffren für Schlüsselaustausch
			\item Eigentlicher Datenverkehr ist symmetrisch verschlüsselt, mit dem zuvor ausgetauschten Schlüssel
		\end{itemize}
	\end{itemize}
\end{frame}

\section{POP3 \& IMAP}
\begin{frame}
	\frametitle{POP3 }
	\begin{itemize}
		\item Post Office Protocol (POP) Übertragungsprotokoll für Clients
		\item Dient dem \glqq Abholen\grqq\ von E-Mails
		\item Rein textbasiertes Protokoll $\rightarrow$ \emph{ASCII}
		\item Beschränkte Funktionalität:
		\begin{itemize}
			\item Auflisten
			\item Abholen
			\item Löschen
		\end{itemize}
	\end{itemize}
\end{frame}

\begin{frame}
	\frametitle{IMAP}
	\begin{itemize}
		\item Internet Message Access Protocol (IMAP)
		\item Im wesentlichen wie POP3
		\item Jedoch mit mehr Features
		\item Mehrere Clients können sich mit Server Connecten
		\item Ordnerstruktur, Dateien bleibt erhalten $\rightarrow$ Dateisystem auf dem Mailserver
		\item Einheitlicher Zugriff
	\end{itemize}
\end{frame}

\begin{frame}
	\frametitle{SMTP}
	\begin{itemize}
		\item Simple Mail Transfer Protocol (SMTP)
		\item Für den Versand und Weiterleitung von Mails
		\item Wesentliche Komponenten:
		\begin{itemize}
			\item Mail User Agent (MUA): Client
			\item Mail Submission Agent (MSA): Server
			\item MSA sendet via Mail Transfer Agent Mails weiter
		\end{itemize}
	\end{itemize}
	
\end{frame}
\end{document}