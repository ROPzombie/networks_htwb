%%%%%%%%%%%%%%%%%%%%%%%VICARIOUS%%%%%%%%%%%%%%%%%%%%%%%%%%%%%%%%%%%%%%%
% 											%
% Template for presentation in Latex`s Beamer Class					%
% Using the default Berlin theme, can be replaced by other themes		%
% logo in the upper right can be replaced by new .png, gif, eps etc	%
% 																		%
%%%%%%%%%%%%%%%%%%%%%%%%%%%%%%%%%%%%%%%%%%%%%%%%%%%%%%%%%%%%%%%%%%%%%%%
\documentclass[xcolor=dvipsnames,aspectratio=169]{beamer}
\usetheme{Berlin}
\usecolortheme[named=LimeGreen]{structure}
\usepackage{beamerthemesplit} % kam neu dazu
\usepackage[ngerman]	{babel}			
\usepackage{t1enc}						
\usepackage[utf8]{inputenc}			
\usepackage{amsmath}
\usepackage{graphicx}
\graphicspath{{pictures/}}
\usepackage{amssymb}
\usepackage{amsfonts}
\usepackage{caption}
\usepackage{multimedia}
\usepackage{tikz}
\usepackage{listings}
\usepackage{acronym}
\usepackage{lmodern}
\usepackage{multicol}
\usepackage{menukeys}


\definecolor{pblue}{rgb}{0.13,0.13,1}
\definecolor{pgreen}{rgb}{0,0.5,0}
\definecolor{pred}{rgb}{0.9,0,0}
\definecolor{pgrey}{rgb}{0.46,0.45,0.48}

\lstset{
    escapeinside={(*}{*)}
}

\lstdefinestyle{Java}{
  showspaces=false,
  showtabs=false,
  tabsize=2,
  breaklines=true,
  showstringspaces=false,
  breakatwhitespace=true,
  commentstyle=\color{pgreen},
  keywordstyle=\color{pblue},
  stringstyle=\color{pred},
  basicstyle=\footnotesize\ttfamily,
  numbers=left,
  numberstyle=\tiny\color{gray}\ttfamily,
  numbersep=7pt,
  %moredelim=[il][\textcolor{pgrey}]{$$},
  moredelim=[is][\textcolor{pgrey}]{\%\%}{\%\%},
  captionpos=b
}

\lstdefinestyle{basic}{  
  basicstyle=\footnotesize\ttfamily,
  breaklines=true
  numbers=left,
  numberstyle=\tiny\color{gray}\ttfamily,
  numbersep=7pt,
  backgroundcolor=\color{white},
  showspaces=false,
  showstringspaces=false,
  showtabs=false,
  frame=single,
  rulecolor=\color{black},
  captionpos=b,
  keywordstyle=\color{blue}\bf,
  commentstyle=\color{gray},
  stringstyle=\color{green},
  keywordstyle={[2]\color{red}\bf},
}


\lstdefinelanguage{custom}
{
morekeywords={public, void},
sensitive=false,
morecomment=[l]{//},
morecomment=[s]{/*}{*/},
morestring=[b]",
}


\lstdefinestyle{BashInputStyle}{
  language=bash,
  showstringspaces=false,
  basicstyle=\small\sffamily,
  numbers=left,
  numberstyle=\tiny,
  numbersep=5pt,
  frame=trlb,
  columns=fullflexible,
  backgroundcolor=\color{gray!20},
  linewidth=0.9\linewidth,
  xleftmargin=0.1\linewidth
}

\lstdefinestyle{Bash}{
  language=bash,
  showstringspaces=false,
  basicstyle=\small\sffamily,
  numbers=left,
  numberstyle=\tiny,
  numbersep=5pt,
  frame=trlb,
  columns=fullflexible,
  backgroundcolor=\color{gray!20},
  linewidth=0.9\linewidth,
  %xleftmargin=0.5\linewidth
}

%Logo in the upper right just change if you know what you are doing^^
\addtobeamertemplate{frametitle}{}{%
\begin{tikzpicture}[remember picture,overlay]
\node[anchor=north east,yshift=2pt] at (current page.north east) {\includegraphics[height=1.8cm]{htw}};
\end{tikzpicture}}

\begin{document}
\bibliographystyle{alpha}
\title{Netzwerke -- Übung SoSe 2019}
\subtitle{Grundlagen *nix \& Shell\\
\href{mailto:Benjamin.Troester@HTW-Berlin.de}{Benjamin.Troester@HTW-Berlin.de}\\
		PGP: ADE1 3997 3D5D B25D 3F8F 0A51 A03A 3A24 978D D673 }

\author{Benjamin Tröster}

\date{}

\begin{frame}
\titlepage
\end{frame}

\section*{Road-Map}
\begin{frame}
\frametitle{Road-Map}
\begin{multicols}{2}
  \tableofcontents
\end{multicols}
\end{frame}

\section{Laborhardware}
\subsection{Laborrechner \& Infrastruktur}
\begin{frame}
	\frametitle{Laborrechner \& Infrastruktur}
\begin{itemize}
	\item WH C 625
	\item 21 Dell Optiplex
	\begin{itemize}
		\item Intel Core i5-7500 -- 4 Cores/ 4 Threads, 3.40 - 3.80 GHz
		\item 16 GB RAM -- 2 $\times$ 8 GB @2400MHz DDR4 Memory
		\item 256 GB SATA-SSD
		\item Ubuntu 18.04 / Windows 10
	\end{itemize}
	\item GBit-\ac{lan}
	\item GBit-\ac{wan} ins \ac{dfn} -- IPv4 Only
\end{itemize}
\end{frame}

\subsection{Raspberry Pi}
\begin{frame}
	\frametitle{Raspberry Pi}
\begin{itemize}
	\item Raspberry Pi 3 Model B+ -- \ac{soc} \url{https://www.raspberrypi.org/}
	\item Architektur: ARM Cortex (64-Bit) Broadcom BCM2837
	\item Quad Core: 4 $\times$ ARM Cortex-A53 @ 1.2GHz
	\item 1GB LPDDR2 (1,2 GMHz)
	\item 10/100/1000 Ethernet, 2.4GHz 802.11n wireless
	\item 4 USB2 ports
	\item Raspbian 9 Stretch -- Debian Fork \\ \url{https://raspbian.org/}
\end{itemize}
\end{frame}

\section{Unixoide Betriebssysteme}
\subsection{Historisches zu Unix}

\begin{frame}
	\frametitle{Historisches zu Unix}
 \begin{tabular}{cc}
 \hspace*{-1.3cm} 
 \parbox{0.65\linewidth}{
\begin{itemize}
	\item Eigentlich von \ac{unics} -- Anspielung auf Multics \footnote{\url{https://de.wikipedia.org/wiki/Multics}}
	\item 1969 entwickelt in den Bell Laboratories
	\item Bekannte Vertreter:
	\begin{itemize}
	\item \ac{bsd}, SunOS/ Solaris, Minix, OpenBSD, IllumOS, FreeBSD
	\item \url{https://de.wikipedia.org/wiki/Berkeley_Software_Distribution}
	\end{itemize}
\end{itemize} } 
& \begin{tabular}{l}
 \begin{tabular}{c}
 \hspace*{-1.1cm}
           \includegraphics[scale=0.65]{Ken_n_dennis}
           \end{tabular}
\end{tabular}\\
\end{tabular}
\end{frame}

\begin{frame}
\begin{figure}
 \vspace*{-.25cm} 
\includegraphics[scale=0.22]{unix_hist}
\end{figure}
\end{frame}

\subsection{Linux}

\begin{frame}
	\frametitle{Linux}
 \begin{tabular}{lc}
\hspace{-1.3cm}
 \parbox{0.65\linewidth}{
 \vspace{-1cm}
\begin{itemize}
	\item 1991 im Usenet \footnote{\url{https://de.wikipedia.org/wiki/Usenet}} veröffentlicht von Linus Torvalds
	\item Linux im wesentlichen Kernel (Betriebssystemkern) + GNU-Tools (Compiler, Debugger etc.)
	\item Distributionen nutzen (angepassten) Linux Kernel + (eigene) Standardsoftware -- wie Paketmanager etc.
	\item Bekannte Linux Distributionen:
	\begin{itemize}
	\item Slackware, Red Hat, Debian, Gentoo, Arch Linux
	\item \url{https://de.wikipedia.org/wiki/Linux}
	\item \url{https://www.kernel.org/}
	\end{itemize}
\end{itemize} } 
& \begin{tabular}{c}
 \hspace{-0.5cm}
           \includegraphics[scale=0.215]{linus}
           \end{tabular}
\end{tabular}
\end{frame}

\subsection{Aufgaben des OS}
\begin{frame}
	\frametitle{Hauptaufgaben des Betriebssystems}
	\begin{itemize}
		\item Bereitstellen einer virtuellen Maschine \url{https://de.wikipedia.org/wiki/Virtuelle_Maschine}
		\begin{itemize}
			\item als Abstraktion des Rechnersystems
		\end{itemize}
		\item Verwaltung und Operationen auf den Ressourcen
		\begin{itemize}
			\item physische Ressourcen
			\item logische Ressourcen
		\end{itemize}
		\item Adaption der Rechnerstruktur für Nutzeranforderungen
		\item Legt die Grundlage für geregelten, nebenläufigen Ablauf der Aktivitäten
		\item Verwaltung der Daten \& Ressourcen
		\item Unterstützung bei Fehlern \& Ausfällen ...
	\end{itemize}
\end{frame}

\subsection{Architektur (monolithischer Kernel)}
\begin{frame}
	\frametitle{Aufbau eines Betriebssystems: Ringmodell}
	\vspace{-.7cm}
 \begin{tabular}{cc} 
 \hspace*{-1.3cm} 
 \parbox{0.65\linewidth}{
\begin{itemize}
	\item Hardware
	\begin{itemize}
		\item CPU, RAM, Mainboard ...
	\end{itemize}
	\item Kernel -- Betriebssystemkern
	\begin{itemize}
		\item Gerätetreiber, Dateisystem, Prozessteuerung, Systemaufrufe ...
	\end{itemize}
	\item Shell -- Schnittstelle zwischen Nutzer \& Diensten des Betriebssystems
	\begin{itemize}
		\item \ac{cli} oder \ac{gui}
		\item Interpretiert \& bearbeitet Eingaben des Nutzers
	\end{itemize}
	\item Anwendungsprogramme
	\begin{itemize}
		\item Standardsoftware \& 3rd-Party-Software
	\end{itemize}

\end{itemize} } 
& \begin{tabular}{l}
 \begin{tabular}{c}
 \vspace{-.05cm}
 %\hspace*{-.5cm}
           \includegraphics[scale=0.5]{unix_arch}
           \end{tabular}
\end{tabular}\\
\end{tabular}
\end{frame}


\subsection{Aufbau Filesystem}
\begin{frame}
	\frametitle{Dateisystem}
	\begin{itemize}
		\item Dateisystem ist die Abstraktion der eigentlichen physischen Ressource (HDDs, SSDs,...)
		\item Dateien sind logische Ressourcen $\rightarrow$ Kollektion von logischen Dateneinheiten -- Records
		\item Dateisysteme richten sich (wie Betriebssystem) nach den Anforderungen
		\item Beispiele:
		\begin{itemize}
			\item UFS -- Unix File System
			\item FAT -- File Allocation Table
			\item NTFS -- New Technology File System
			\item ZFS -- Zettabyte File System
			\item CEPH
		\end{itemize}
	\end{itemize}
\end{frame}

\begin{frame}
\begin{figure}
  \vspace*{-0.225cm}
\includegraphics[scale=0.4]{filesystem}
\end{figure}
\end{frame}

\begin{frame}
In Linux/ Unix ist grundsätzlich alles eine Datei!\\
\textbf{Baumstruktur -- statt separate Massenspeicher}
Exemplarisch:
\begin{itemize}
	\item / -- Wurzelverzeichnis
	\item /bin -- wichtigste Programme in Binäreformat
	\item /boot -- Boot-Loader
	\item /etc -- System Konfiguration

	\item /usr -- Konfiguration, Shared-Software
	\begin{itemize}
		\item /usr/local -- lokale Software
		\item /usr/share -- statische Daten
		\item /usr/bin -- User-Land-Software/ Commands
		\item /usr/include -- Standard-Bibliotheken für C/C++
		\item /usr/lib -- Bibliotheken für Programmiersprachen
		\item /usr/sbin -- andere Binaries
	\end{itemize}

\end{itemize}
\end{frame}
\begin{frame}
\begin{itemize}
\item /var -- Variable Daten
	\begin{itemize}
		\item /var/log -- zentrale Log-File-Stelle
	\end{itemize}
	\item /sbin -- System Binaries
	\item /tmp -- Temporäre Dateien
	\item /dev -- Geräte
	\item /home -- User-Bereich
	\item /lib -- Bibliotheken \& Kernel-Module
	\item /opt -- zusätzliche Software -- Add Ons
	\item /root -- Verzeichnis vom Root-User
\end{itemize}
Ausführlicher: \url{https://en.wikipedia.org/wiki/Unix_filesystem\#Conventional_directory_layout}
\end{frame}

\subsection{User, Gruppen}
\begin{frame}
Linux/ Unix sind Mehrbenutzersysteme, d.h. mehrere Nutzer können simultan auf einem Rechner arbeiten
\begin{itemize}
	\item Zuordnung der Nutzer zu User \& Group
	\item Regelt Zugangskontrolle im System auf
	\begin{itemize}
	\item  Dateien, Ordner \& Peripheriegeräte
	\end{itemize}
	\item Unterschiedliche Nutzer/ Gruppen $\rightarrow$ unterschiedliche Rechte
	\item Im Labor:
	\begin{itemize}
		\item Benutzername: Matrikelnummer
		\item Gruppen: student, domain, users,...
	\end{itemize}
	\item Raspberry Pi:
	\begin{itemize}
		\item Benutzername: student
		\item Gruppen: student, users, wireshark,...
	\end{itemize}
\end{itemize}
\end{frame}

\subsection{Nutzerrechte}
\begin{frame}
Um die Zugriffsrechte der jeweiligen Nutzer zu regeln bietet Linux/ Unix ein Berechtigungsmodell\\
Abbildung der Nutzer, Gruppen auf Zugriffsmöglichkeiten der Dateien
\begin{itemize}
	\item Grundsätzlich in drei Kategorien:
	\begin{itemize}
		\item Owner -- regelt Berechtigung des Eigentümers
		\item Group -- regelt Berechtigung der Gruppe
		\item Other (world) -- regelt Berechtigung aller anderen Nutzer 
	\end{itemize}
	\item Unix Zugriffsmodi:
	\begin{itemize}
		\item read (r) -- Lesezugriff
		\item write (w) -- Schreibzugriff
		\item execute (x) -- Ausführzugriff
	\end{itemize}
\end{itemize}
\end{frame}

\begin{frame}
Zahlensystem:
\begin{itemize}
	\item Dezimalsystem -- Basis 10
	\begin{itemize}
		\item Werte 0 - 9 
	\end{itemize}
	\item Dualsystem/ Binärsystem -- Basis 2
	\begin{itemize}
		\item Werte 0 oder 1
		\item Bit-Darstellung in der Informatik/ Rechnertechnik
	\end{itemize}
	\item Oktalsystem -- Basis 8
	\begin{itemize}
		\item Werte 0 - 7
		\item Für Darstellung der Zahlen 0 - 7 $\rightarrow$ 3 Bit notwendig, $2^3 = 8$
	\end{itemize}
\end{itemize}
\end{frame}


\begin{frame}
Darstellung im System via Oktalzahlen:
\begin{itemize}
	\item Zuordnung der Berechtigung r,w,x bestimmten Werten
	\begin{itemize}
		\item Lesen (r) $\rightarrow$ $4_8$ oder $100_2$
		\item Schreiben (w) $\rightarrow$ $2_8$ oder $010_2$
		\item Ausführen (x) $\rightarrow$ $1_8$ oder $001_2$
		\item None $\rightarrow$ $0_8$ oder $000_2$
	\end{itemize}
	\item Zusammensetzen der Oktalwerte ergibt Zugriffsrechte:
	\begin{itemize}
		\item Lesen, schreiben und ausführen $\rightarrow$ $7_8$ oder $111_2$
		\item Lesen und Schreiben $\rightarrow$ $6_8$ oder $110_2$
		\item Lesen und Ausführen $\rightarrow$ $5_8$ oder $101_2$
		\item ...
	\end{itemize}
\end{itemize}
\end{frame}

\begin{frame}
Zusammensetzung der Berechtigung
\begin{itemize}
	\item 3er-Oktett gibt Zugriffmodalitäten an
	\begin{enumerate}
		\item User r,w,x -- erstes Oktett
		\item group r,w,x -- zweites Oktett
		\item other r,w,x -- drittes Oktett
	\end{enumerate}
\end{itemize}
\begin{figure}
	\includegraphics[scale=0.5]{rights}
\end{figure}
\end{frame}

\section{Shell}
\subsection{Einführung}
\begin{frame}
	\frametitle{Shell}
	\begin{itemize}
		\item Textbasierte Schnittstelle
		\item Schnittstelle zwischen OS-Kernel \& OS-Util. \& User
		\item Shell ist ein Kommando-Interpreter $\rightarrow$ führt Schrittweise Befehle aus
		\begin{itemize}
			\item Kommandos können auch Binärdatei aufgerufen werden
			\item Kommandos können direkt aufgerufen werden
			\item Aufruf von Systemcalls möglich $\rightarrow$ Administration des Systems
		\end{itemize}
	\end{itemize}
\end{frame}

\subsection{Unix-Philosophie}
\begin{frame}
	\frametitle{Unix-Philosophie}
	Nach Douglas McIlroy:
	\begin{itemize}
		\item Schreibe Computerprogramme so, dass sie nur eine Aufgabe erledigen und diese gut machen.
		\item Schreibe Programme so, dass sie zusammenarbeiten.
		\item Schreibe Programme so, dass sie Textströme verarbeiten, denn das ist eine universelle Schnittstelle.
	\end{itemize}
	Bottom-Line: Einfach Programme die zusammenarbeiten können, sodass komplexere Probleme lösbar sind
\end{frame}

\begin{frame}
	\frametitle{Shells}
	\begin{itemize}
		\item Ur-Shell: Thompson Shell -- OSH
		\item SH -- Bourne Shell 
		\item BASH -- Bourne Again Shell
		\item CSH -- C Shell
		\item TCSH -- TENEX C Shell
		\item KSH -- Korn Shell
		\item ZSH -- Zhong Shao Shell
		\item ...
	\end{itemize}
\end{frame}

\section{Systemcalls}
\begin{frame}
	\frametitle{Systemcalls \& Daemons -- !Short Version}
	\begin{itemize}
		\item Systemcalls -- Methode für Anwendungsprogramme, um Funktionalitäten des BS nutzen zu können
		\item Systemcalls vollführen Wechsel von Anwenderebene auf BS-Kern 
		\item Übergabe der Kontrolle von Anwender an das Betriebssystem
		\begin{itemize}
			\item Bsp.: anlegen von Dateien auf SSD, Verbindung des Browsers zu einer Webseite ...
		\end{itemize}
		\item Daemons -- Hintergrunddienste
		\item Stellen Dienste des BS bereit, auf die Programme zugreifen können
		\begin{itemize}
			\item Bsp: Netzwerkdienste, Sockets ...
		\end{itemize} 
	\end{itemize}
\end{frame}

\section{Shell 101}
\begin{frame}
	\frametitle{Shell 101}
	\begin{itemize}
		\item Auf den Laborrechner: \keys{\winmenu} (Windowstaste) und dann einfach Terminal eingeben
		\item Alternativ: \keys{\ctrl + \Alt + t} (Str + Alt + t)
		\item Abbrechen eines Kommandos: \keys{\ctrl + c} (Str + c)
		\item Beenden/Schließen des Terminals: \keys{\ctrl + d} (Str + d) oder einfach \emph{exit} eintippen
	\end{itemize}
\end{frame}

\begin{frame}
	\frametitle{Shell Input/Output}
	\begin{itemize}
		\item Kommandozeile hat drei Standardkanäle:
		\begin{enumerate}
			\item Standardinput (stdin) -- Eingabe von Daten
			\item Standardoutput (stdout) -- Ausgabe von Daten
			\item Standarderror (stderr) -- Ausgabekanal im Fall von Fehlern
		\end{enumerate}
		\item Ausgabe von Tools zumeist auf stdout
		\item Ein- \& Ausgabe kann jedoch auch umgelenkt werden
		\item Ausgabe kann somit in Datei geschrieben bzw. aus Datei gelesen werden
		\item Verbinden von Kommandos durch \emph{Piping}
		\begin{itemize}
			\item Ausgabe eines Programms wird Eingabe des anderen Programms
		\end{itemize}
	\end{itemize}
\end{frame}

\begin{frame}
	\frametitle{Input/Output Redirection}
\begin{itemize}
	\item Umlenken der Ausgabe in eine Datei: \keys{>}
		\begin{itemize}
			\item Lenkt Ausgabe in eine Datei
			\item Datei wird dabei vollständig neu geschrieben
			\item Alter Inhalt geht verloren
		\end{itemize}
	\item Anfügen einer Datei: \keys{>>}
	\begin{itemize}
		\item Hängt Ausgabe an das Ende der Datei 
	\end{itemize}
	\item Umlenken der Eingabe aus einer Datei: \keys{<}
		\begin{itemize}
			\item Programm erhält zeilenweise Eingabe aus der Ressource
		\end{itemize}
\end{itemize}
\end{frame}

\begin{frame}
	\frametitle{Piping}
\begin{itemize}
	\item Verbinden mehrerer Kommandos durch eine \emph{Pipe}
	\item Pipe: Datenstruktur -- Folgt dem First In - First Out Prinzip
	\item Wirkt wie ein Puffer, eingehende Daten werde gepuffert und bei Bedarf wieder ausgegeben
	\item Piping ermöglicht es Programme zu verbinden
	\item Ausgabestrom eines Programm wird Eingabestrom des nächsten Programms
	\item \textbf{Vorteil:} Einfache Programme können zu mächtigeren Programmen zusammengesetzt werden
	\item Folgt der Unix-Philosophie
\end{itemize}
\end{frame}

\begin{frame}[fragile]
	\frametitle{Beispiele}
	\begin{lstlisting}[style=Bash, language=Bash]
date > foo.txt
cat id_rsa.pub >> authorized_keys
head < index.html
sort studi_list_1.csv studi_list_2.csv | uniq -u > drop_out_list.txt
	\end{lstlisting}
\end{frame}

\begin{acronym}
\acro{bsd}[BSD]{Berkeley Software Distribution}
\acro{cli}[CLI]{Command Line Interface}
\acro{dfn}[DFN]{Deutsches Forschungsnetz}
\acro{gui}[GUI]{Graphical User Interface}
\acro{gw}[GW]{Gateway}
\acroplural{gws}[GW`s]{Gateways}
\acro{lan}[LAN]{Local Area Network}
\acro{moco}[MOCO]{Mobile Computing}
\acro{soc}[SoC]{System on a chip}
\acro{wan}[WAN]{Wide Are Network}
\acro{unics}[UNICS]{Uniplexed Information and Computing Service}
\end{acronym}

\end{document}