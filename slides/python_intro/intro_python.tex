\documentclass[xcolor=dvipsnames,aspectratio=169]{beamer}

\usepackage{fontspec}
\defaultfontfeatures{Ligatures=TeX}
%\setsansfont{Liberation Sans}
\usepackage{polyglossia}
\setdefaultlanguage{german}

\usetheme{Berlin}
\usecolortheme[named=LimeGreen]{structure}
\usepackage{beamerthemesplit} % kam neu dazu
			
\usepackage{amsmath}
\usepackage{graphicx}
\graphicspath{{pictures/}}
\usepackage{amssymb}
\usepackage{amsfonts}
\usepackage{caption}
\usepackage{multimedia}
\usepackage{tikz}
\usepackage{listings}
\usepackage{acronym}
\usepackage{lmodern}
\usepackage{multicol}
\usepackage{menukeys}

\definecolor{pblue}{rgb}{0.13,0.13,1}
\definecolor{pgreen}{rgb}{0,0.5,0}
\definecolor{pred}{rgb}{0.9,0,0}
\definecolor{pgrey}{rgb}{0.46,0.45,0.48}

\lstset{
    escapeinside={(*}{*)}
}

\lstdefinestyle{Java}{
  showspaces=false,
  showtabs=false,
  tabsize=2,
  breaklines=true,
  showstringspaces=false,
  breakatwhitespace=true,
  commentstyle=\color{pgreen},
  keywordstyle=\color{pblue},
  stringstyle=\color{pred},
  basicstyle=\footnotesize\ttfamily,
  numbers=left,
  numberstyle=\tiny\color{gray}\ttfamily,
  numbersep=7pt,
  %moredelim=[il][\textcolor{pgrey}]{$$},
  moredelim=[is][\textcolor{pgrey}]{\%\%}{\%\%},
  captionpos=b
}

\lstdefinestyle{basic}{  
  basicstyle=\footnotesize\ttfamily,
  breaklines=true
  numbers=left,
  numberstyle=\tiny\color{gray}\ttfamily,
  numbersep=7pt,
  backgroundcolor=\color{white},
  showspaces=false,
  showstringspaces=false,
  showtabs=false,
  frame=single,
  rulecolor=\color{black},
  captionpos=b,
  keywordstyle=\color{blue}\bf,
  commentstyle=\color{gray},
  stringstyle=\color{green},
  keywordstyle={[2]\color{red}\bf},
}


\lstdefinelanguage{custom}
{
morekeywords={public, void},
sensitive=false,
morecomment=[l]{//},
morecomment=[s]{/*}{*/},
morestring=[b]",
}


\lstdefinestyle{BashInputStyle}{
  language=bash,
  showstringspaces=false,
  basicstyle=\small\sffamily,
  numbers=left,
  numberstyle=\tiny,
  numbersep=5pt,
  frame=trlb,
  columns=fullflexible,
  backgroundcolor=\color{green!20},
  linewidth=0.9\linewidth,
  xleftmargin=0.1\linewidth
}

\lstdefinestyle{Bash}{
basicstyle=\tiny\ttfamily,
  language=bash,
  showstringspaces=false,
  basicstyle=\tiny\sffamily,
  numbers=left,
  numberstyle=\tiny,
  numbersep=5pt,
  frame=trlb,
  columns=fullflexible,
  backgroundcolor=\color{green!20},
  linewidth=0.9\linewidth,
  %xleftmargin=0.5\linewidth
}

\lstloadlanguages{Python}
\lstset{language=Python,
basicstyle=\small\ttfamily,
	tabsize=4,
	frame=single,
	basicstyle=\tiny,
	upquote=true,
	numbersep=5pt,
	numbers=left,
	%commentstyle=\color{green} % Kommentare blau drucken
	stringstyle=\ttfamily, % Strings im Code Schreibmaschinenähnlich - setzt sich etwas vom Code ab
	showstringspaces=false,
	columns=fullflexible,
  	backgroundcolor=\color{gray!10},
  	%linewidth=1.0\linewidth,
  	breaklines=true
}

%Logo in the upper right just change if you know what you are doing^^
\addtobeamertemplate{frametitle}{}{%
\begin{tikzpicture}[remember picture,overlay]
\node[anchor=north east,yshift=2pt] at (current page.north east) {\includegraphics[height=1.8cm]{htw}};
\end{tikzpicture}}

\begin{document}
\bibliographystyle{alpha}
\title{Netzwerke -- Übung SoSe 2020}
\subtitle{Python 101\\
\href{mailto:Benjamin.Troester@HTW-Berlin.de}{Benjamin.Troester@HTW-Berlin.de}\\
		PGP: ADE1 3997 3D5D B25D 3F8F 0A51 A03A 3A24 978D D673 }

\author{Benjamin Tröster}

\date{}

\begin{frame}
\titlepage
\end{frame}

\begin{frame}{Python}
	\begin{itemize}
		\item Nicht die Schlange, sondern großartige britische Comedy
	\end{itemize}
\end{frame}

\section*{Road-Map}
\begin{frame}
\frametitle{Road-Map}
\begin{multicols}{2}
  \tableofcontents
\end{multicols}
\end{frame}


\section{Einführung}
\begin{frame}{Was ist Python}
Python: Dynamische Programmiersprache, die unterschiedliche Programmierparadigma unterstützt:
\begin{itemize}
	\item Prozedural -- C, Fortran
	\item Objektorientiert -- Smalltalk, Java
	\item Funktionla -- LISP, Haskell
\end{itemize}
Also ist Python eine Multi-Paradigma Sprache, wie C++ oder Rust.
Standard: Python Byte-Code wird im Python-Interpreter ausgeführt $\rightarrow$ 
Plattformunabhängiger Code
\end{frame}

\begin{frame}{Warum Python?}
Extrem vielfältige Programmiersprache
\begin{itemize}
	\item Website development, data analysis, server maintenance, numerical analysis, ...
	\item Syntax is clear, easy to read and learn (almost pseudo code)
	\item Common language
	\item Intuitive object oriented programming
	\item Full modularity, hierarchical packages
	\item Comprehensive standard library for many tasks
	\item Big community
	\item Simply extendable via C/C++, wrapping of C/C++ libraries
	\item \textbf{Focus: Programming speed}
\end{itemize}
\end{frame}

\begin{frame}{Python Zen}
\begin{itemize}
	\item 20 software principles that influence the design of Python:
	\begin{enumerate}
		\item Beautiful is better than ugly.
		\item Explicit is better than implicit.
		\item Simple is better than complex.
		\item Complex is better than complicated.
		\item Flat is better than nested.
		\item Sparse is better than dense.
		\item Readability counts.
		\item Special cases aren’t special enough to break the rules.
		\item Although practicality beats purity.
		\item Errors should never pass silently.
		\item Unless explicitly silenced.
		\item ...
	\end{enumerate}
\end{itemize}
\end{frame}

\section{Hello, World!}
\begin{frame}[fragile]{Hello, World!}
\begin{lstlisting}[language=Python]
#!/usr/bin/env python3
#This is a commentary
print("Hello world!")
\end{lstlisting}
\begin{lstlisting}[language=Bash ,backgroundcolor=\color{green!20}]]
$ python3 hello_world.py
Hello world!
$
\end{lstlisting}
\begin{lstlisting}[language=Bash ,backgroundcolor=\color{green!20}]]
$ chmod u+x hello_world.py
$ ./hello_world.py
Hello world!
$
\end{lstlisting}
\end{frame}

\begin{frame}[fragile]{Hello, World!}
\begin{lstlisting}[language=Python]
#!/usr/bin/env python3
name = input("What’s your name?")
print("Hello", name)
\end{lstlisting}
\begin{lstlisting}[language=Bash ,backgroundcolor=\color{green!20}]]
$./hello_user.py
What’s your name? Dieter
Hello Dieter
$
\end{lstlisting}
\end{frame}

\section{Typing}
\begin{frame}{Strong and Dynamic Typing}
\textbf{Strong Typing:}
\begin{itemize}
	\item Object is of exactly one type! A string is always a string, an integer always an integer
	\item Counterexamples: PHP, JavaScript, C: char can be interpreted as short, void * can be everything
\end{itemize}
\textbf{Dynamic Typing:}
	\begin{itemize}
		\item No variable declaration
		\item Variable names can be assigned to different data types in the course of a program
		\item An object’s attributes are checked only at run time 
		\item Duck typing (an object is defined by its methods and attributes)
		\begin{quote}
		When I see a bird that walks like a duck and swims like a duck and quacks like a duck, I call that bird a duck.
		\end{quote}
	\end{itemize}
\end{frame}

\begin{frame}[fragile]{Example: Strong and Dynamic Typing}
\begin{lstlisting}[language=Python]
#!/usr/bin/env python3
number = 3
print(number, type(number))
print(number + 42)
number = "3"
print(number ,type(number))
print(number + 42)
\end{lstlisting}

\begin{lstlisting}[language=Bash ,backgroundcolor=\color{green!20}]]
3 <class ’ int ’>
45
3 <class ’ str ’>
Traceback(most recent call last ):
File "types.py" ,line 7, in <module>
print(number + 42)
TypeError: can only concatenate str (not "int") to str
\end{lstlisting}
\end{frame}

\begin{frame}[fragile]{REPL}
The interpreter can be started in interactive mode:
\begin{lstlisting}[language=Bash ,backgroundcolor=\color{green!20}]]
$ python3
Python 3.7.2 ( default , May 23 2019 , 03:15:18)
[GCC 10.1.0] on freebsd
Type "help" , "copyright" , "credits" or "license" for
more information .
>>> print (" hello world")
hello world
>>> a = 3 + 4
>>> print (a)
7
>>> 3 + 4
7
>>>
\end{lstlisting}
\end{frame}

\section{Python2 vs. Python3}
\begin{frame}
\begin{table}
\centering
\begin{tabular}{|l|c|c|} \hline
& \textbf{Python2} & \textbf{Python3} \\ \hline
shebang & \texttt{\#!/usr/bin/python} & \texttt{\#!/usr/bin/python3} \\ \hline
IDLE & idle & idle3 \\ \hline
print cmd (syntax) & print & print() \\ \hline
input cmd (syntax) & raw\_input() & input() \\ \hline
unicode & u"..." & all strings \\ \hline
integer type & int/long & int (infinite) \\ \hline
\end{tabular}
\end{table}
\end{frame}

\section{Numerical Data Types}
\begin{frame}[fragile]{Numerical Data Types}
\begin{itemize}
	\item int: integer numbers (infinite)
	\item float: corresponds to double in C
	\item complex: complex numbers (j is the imaginary unit)
\end{itemize}

\begin{lstlisting}[language=Python]
a = 1
c = 1.0
c = 1 e0
d = 1 + 0j
\end{lstlisting}
\end{frame}

\begin{frame}[fragile]{Operators on Numbers}
\begin{itemize}
	\item Basic arithmetics: + , - , * , /\\
	hint: Python $2 \Rightarrow 1/2 = 0$\\
	Python $3 \Rightarrow 1/2 = 0.5$
	\item Div and modulo operator: // , \% , divmod(x, y)
	\item Absolute value: abs(x)
	\item Rounding: round(x)
	\item Conversion: int(x) , float(x) , complex(re [, im=0])
	\item Conjugate of a complex number: x.conjugate()
	\item Power: x ** y , pow(x, y)
	\item Result of a composition of different data types is of the "bigger" data type.
\end{itemize}
\end{frame}



\begin{frame}[fragile]
 \begin{tabular}{cc}
 \hspace*{-1.3cm} 
 \parbox{0.65\linewidth}{
\begin{itemize}
	\item Operations:
	\begin{itemize}
		\item AND: x \& y
		\item OR: x | y
		\item exclusive OR (XOR): x \textasciitilde~ y
		\item invert: \textasciitilde  x
		\item shift right n bits: x >> n
		\item shift left n bits: x << n
	\end{itemize}
Use bin(x) to get binary representation string of x .
\end{itemize} } 
& \begin{tabular}{l}
 \begin{tabular}{c}
 \hspace*{-1.1cm}
           \begin{lstlisting}[language=Python]
>>> print ( bin (6) , bin (3))
0 b110 0 b11
>>> 6 & 3
2
>>> 6 | 3
7
>>> 6 ^ 3
5
>>> ~0
-1
>>> 1 << 3
8
>>> pow (2 ,3)
8
>>> 9 >> 1
4
>>> print ( bin (
\end{lstlisting}
           \end{tabular}
\end{tabular}\\
\end{tabular}

\section{Strings}
\begin{frame}{Strings}
Data type: str
\begin{itemize}
	\item s = 'spam' , s = "spam"
	\item Multiline strings: s = """spam"""
	\item No interpretation of escape sequences: $s = r"sp nam"$
	\item Generate strings from other data types: str(1.0)
\end{itemize}

\end{frame}
\end{frame}





\end{document}