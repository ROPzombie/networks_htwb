%%%%%%%%%%%%%%%%%%%%%%%VICARIOUS%%%%%%%%%%%%%%%%%%%%%%%%%%%%%%%%%%%%%%%
% 																	%
% Template for presentation in Latex`s Beamer Class					%
% Using the default Berlin theme, can be replaced by other themes		%
% logo in the upper right can be replaced by new .png, gif, eps etc	%
% 																	%
%%%%%%%%%%%%%%%%%%%%%%%%%%%%%%%%%%%%%%%%%%%%%%%%%%%%%%%%%%%%%%%%%%%%%%%
\documentclass[xcolor=dvipsnames, aspectratio=169]{beamer}
\usetheme{Berlin}
\usecolortheme[named=LimeGreen]{structure}
\usepackage{beamerthemesplit} % kam neu dazu
\usepackage[ngerman]	{babel}			
\usepackage{t1enc}						
\usepackage[utf8]{inputenc}			
\usepackage{amsmath}
\usepackage{graphicx}
\graphicspath{{pictures/}}
\usepackage{amssymb}
\usepackage{amsfonts}
\usepackage{caption}
\usepackage{multimedia}
\usepackage{tikz}
\usepackage{listings}
\usepackage{acronym}
\usepackage{subfig}

\usepackage{lmodern}
\usepackage{multicol}

\definecolor{pblue}{rgb}{0.13,0.13,1}
\definecolor{pgreen}{rgb}{0,0.5,0}
\definecolor{pred}{rgb}{0.9,0,0}
\definecolor{pgrey}{rgb}{0.46,0.45,0.48}

\lstset{
    escapeinside={(*}{*)}
}


\lstdefinestyle{basic}{  
  basicstyle=\footnotesize\ttfamily,
  breaklines=true
  numbers=left,
  numberstyle=\tiny\color{gray}\ttfamily,
  numbersep=7pt,
  backgroundcolor=\color{white},
  showspaces=false,
  showstringspaces=false,
  showtabs=false,
  frame=single,
  rulecolor=\color{black},
  captionpos=b,
  keywordstyle=\color{blue}\bf,
  commentstyle=\color{gray},
  stringstyle=\color{green},
  keywordstyle={[2]\color{red}\bf},
}


\lstdefinelanguage{custom}
{
morekeywords={public, void},
sensitive=false,
morecomment=[l]{//},
morecomment=[s]{/*}{*/},
morestring=[b]",
}


\lstdefinestyle{BashInputStyle}{
  language=bash,
  showstringspaces=false,
  basicstyle=\small\sffamily,
  numbers=left,
  numberstyle=\tiny,
  numbersep=5pt,
  frame=trlb,
  columns=fullflexible,
  backgroundcolor=\color{gray!20},
  linewidth=0.9\linewidth,
  xleftmargin=0.1\linewidth
}

%Logo in the upper right just change if you know what you are doing^^
\addtobeamertemplate{frametitle}{}{%
\begin{tikzpicture}[remember picture,overlay]
\node[anchor=north east,yshift=2pt] at (current page.north east) {\includegraphics[height=1.8cm]{htw}};
\end{tikzpicture}}

\begin{document}
\bibliographystyle{alpha}
\title{Netzwerke -- Übung WiSe2018/19}
\subtitle{Routing\\
		\href{mailto:Benjamin.Troester@HTW-Berlin.de}{Benjamin.Troester@HTW-Berlin.de}\\
		PGP: ADE1 3997 3D5D B25D 3F8F 0A51 A03A 3A24 978D D673 }
\author{Benjamin Tröster}

\date{}

\begin{frame}
\titlepage

\end{frame}

\section*{Road-Map}
\begin{frame}
\frametitle{Road-Map}
\begin{multicols}{2}
  \tableofcontents
\end{multicols}
\end{frame}

\section{Retrospektive}
\begin{frame}{Retrospektive}
\begin{itemize}
	\item Vorlesung
	\begin{itemize}
		\item Retrospektive der Vorlesung -- was haben Sie behandelt?
		\item Fragen?
	\end{itemize}
	\item Übungsblatt
	\begin{itemize}
		\item Stand des letzten Übungsblatts
		\item Fragen?
	\end{itemize}
	\item Präsentation des Theorieteils
\end{itemize}
\end{frame}


\section{Präsentation}
\subsection{Routing I}
\begin{frame}
\frametitle{1.) Routing}
	\begin{itemize}
		\item Erläutern Sie die Aufgaben eines Routers! Auf welcher Ebene des OSI-Modells arbeitet der Router, auf welcher der Switch?
		\item Wie erfolgt im Groben die Umsetzung des Routings? D.h. was macht der Router?
		\item Nennen Sie einige Routing-Protokolle.
		\item Ist \emph{IPv4}/\emph{IPv6} ein Routing-Protokoll?
	\end{itemize}
\end{frame}

\subsection{Routing II}
\begin{frame}
	\frametitle{2.) Routing II}
	\begin{itemize}
		\item Erläutern Sie, wie Router und IP-Protokoll zusammenhängen.
		\item Erläutern Sie, wenn möglich anhand eines Beispiels, wie sich bis jetzt (Stand letztes Übungsblatt) Ihre Raspberry Pis gefunden haben.
		\item Erklären Sie wie der Router die Zuordnung der Geräte vornimmt. (D.h. woher weiß ein Router, wann er ein Paket weiterschicken soll und wann nicht.)
	\end{itemize}
\end{frame}

\subsection{CIDR \& LAN-Umsetzung}
\begin{frame}
	\frametitle{3.) CIDR \& Routing -- theoretische Umsetzung}
	\begin{itemize}
		\item Erläutern Sie die wesentlichen unterscheide zwischen CIDR und klassenbasierten Adressen? (Verdeutlichen Sie dies anhand von Beispielen)
		\item Diskutieren Sie die theoretische Umsetzung Ihres Netzwerkes! (Inkl. Skizze, IP-Adressen, Subnetzmasken etc.)
	\end{itemize}
\end{frame}

\subsection{Tools}
\begin{frame}
	\frametitle{4.) Tools}
	\begin{itemize}
		\item Erläutern Sie die Funktion und Syntax des \emph{iproute2}-Tools \emph{ip route}.
		\item Erläutern Sie analog dazu das Tool \emph{route} aus dem \emph{net-tools}-Werkzeugkasten.
		\item Diskutieren Sie, wie eine persistente Lösung des Routers aussehen könnte.
	\end{itemize}
\end{frame}

\begin{frame}
\frametitle{5.) Gateways \& Router}
	\begin{itemize}
		\item Erklären Sie die den Unterschied zwischen Router und Gateway. Nennen und erläutern Sie unterschiedliche Gateway-Arten. 
		\item Erläutern Sie den Unterschied zwischen Forwarding und Routing.
		\item Nennen Sie die für das Routing im Linux-Kernel genutzte Datenstruktur.
		\item Erläutern Sie im wesentlichen welche zwei Möglichkeiten für das Einschalten des Routings im Linux-OS vorhanden sind. Zeigen Sie die Umsetzung beispielhaft.
	\end{itemize}
\end{frame}

\subsection{ICMP}
\begin{frame}
\frametitle{6.) ICMP}
	\begin{itemize}
		\item Erklären Sie kurz was \emph{ICMP} ist. 
		\item Erläutern Sie die Fehlermeldungen:
		\begin{itemize}
			\item connect: network is unreachable
			\item Destination Host Unreachable
			\item Destination Network Unreachable
			\item keine Antwort auf ein Ping
		\end{itemize}
		Was will Ihnen der Rechner hiermit sagen?
	\end{itemize}
\end{frame}

\end{document}
