%%%%%%%%%%%%%%%%%%%%%%%VICARIOUS%%%%%%%%%%%%%%%%%%%%%%%%%%%%%%%%%%%%%%%
% 																	%
% Template for presentation in Latex`s Beamer Class					%
% Using the default Berlin theme, can be replaced by other themes		%
% logo in the upper right can be replaced by new .png, gif, eps etc	%
% 																	%
%%%%%%%%%%%%%%%%%%%%%%%%%%%%%%%%%%%%%%%%%%%%%%%%%%%%%%%%%%%%%%%%%%%%%%%
\documentclass[xcolor=dvipsnames, aspectratio=169]{beamer}
\usetheme{Berlin}
\usecolortheme[named=LimeGreen]{structure}
\usepackage{beamerthemesplit} % kam neu dazu
\usepackage[ngerman]	{babel}			
\usepackage{t1enc}						
\usepackage[utf8]{inputenc}			
\usepackage{amsmath}
\usepackage{graphicx}
\graphicspath{{pictures/}}
\usepackage{amssymb}
\usepackage{amsfonts}
\usepackage{caption}
\usepackage{multimedia}
\usepackage{tikz}
\usepackage{listings}
\usepackage{acronym}
\usepackage{subfig}

\usepackage{lmodern}
\usepackage{multicol}

\definecolor{pblue}{rgb}{0.13,0.13,1}
\definecolor{pgreen}{rgb}{0,0.5,0}
\definecolor{pred}{rgb}{0.9,0,0}
\definecolor{pgrey}{rgb}{0.46,0.45,0.48}

\lstset{
    escapeinside={(*}{*)}
}

\lstdefinestyle{Java}{
  showspaces=false,
  showtabs=false,
  tabsize=2,
  breaklines=true,
  showstringspaces=false,
  breakatwhitespace=true,
  commentstyle=\color{pgreen},
  keywordstyle=\color{pblue},
  stringstyle=\color{pred},
  basicstyle=\footnotesize\ttfamily,
  numbers=left,
  numberstyle=\tiny\color{gray}\ttfamily,
  numbersep=7pt,
  %moredelim=[il][\textcolor{pgrey}]{$$},
  moredelim=[is][\textcolor{pgrey}]{\%\%}{\%\%},
  captionpos=b
}

\lstdefinestyle{basic}{  
  basicstyle=\footnotesize\ttfamily,
  breaklines=true
  numbers=left,
  numberstyle=\tiny\color{gray}\ttfamily,
  numbersep=7pt,
  backgroundcolor=\color{white},
  showspaces=false,
  showstringspaces=false,
  showtabs=false,
  frame=single,
  rulecolor=\color{black},
  captionpos=b,
  keywordstyle=\color{blue}\bf,
  commentstyle=\color{gray},
  stringstyle=\color{green},
  keywordstyle={[2]\color{red}\bf},
}


\lstdefinelanguage{custom}
{
morekeywords={public, void},
sensitive=false,
morecomment=[l]{//},
morecomment=[s]{/*}{*/},
morestring=[b]",
}


\lstdefinestyle{BashInputStyle}{
  language=bash,
  showstringspaces=false,
  basicstyle=\small\sffamily,
  numbers=left,
  numberstyle=\tiny,
  numbersep=5pt,
  frame=trlb,
  columns=fullflexible,
  backgroundcolor=\color{gray!20},
  linewidth=0.9\linewidth,
  xleftmargin=0.1\linewidth
}

%Logo in the upper right just change if you know what you are doing^^
\addtobeamertemplate{frametitle}{}{%
\begin{tikzpicture}[remember picture,overlay]
\node[anchor=north east,yshift=2pt] at (current page.north east) {\includegraphics[height=1.8cm]{htw}};
\end{tikzpicture}}

\begin{document}
\bibliographystyle{alpha}
\title{Netzwerke -- Übung SoSe 2019}
\subtitle{Routing\\
		\href{mailto:Benjamin.Troester@HTW-Berlin.de}{Benjamin.Troester@HTW-Berlin.de}\\
		PGP: ADE1 3997 3D5D B25D 3F8F 0A51 A03A 3A24 978D D673 }
\author{Benjamin Tröster}

\date{}

\begin{frame}
\titlepage

\end{frame}

\section*{Road-Map}
\begin{frame}
\frametitle{Road-Map}
\begin{multicols}{2}
  \tableofcontents
\end{multicols}
\end{frame}

\section{Retrospektive}
\begin{frame}{Retrospektive}
\begin{itemize}
	\item Vorlesung
	\begin{itemize}
		\item Retrospektive der Vorlesung -- was haben Sie behandelt?
		\item Fragen?
	\end{itemize}
	\item Übungsblatt
	\begin{itemize}
		\item Stand des letzten Übungsblatts
		\item Fragen?
	\end{itemize}
\end{itemize}
\end{frame}


\subsection{Requirements}
\begin{frame}{1.) Requirements}
Für die Routing Übung sind wichtig:
\begin{itemize}
	\item \glqq Sattelfest\grqq\ in \emph{IPv4} \& \emph{IPv6}
	\begin{itemize}
		\item IP-Adressierung, Aufbau und Nutzung von \emph{IPv4}, \emph{IPv6}
		\item Aufbau, Funktion, Nutzung von Subnetzmasken	
	\end{itemize}
	\item Routing, Routing-Tables, Forwarding
	\item Topologien \& NW-Architektur
	\item NAT
	\item Tooling: \emph{ip addr}, \emph{ifconfig}, \emph{ip route}, \emph{route} \emph{ping}
	\item \emph{iptables}, \emph{sysctl}
\end{itemize}
\end{frame}

\begin{frame}{Architektur}
	\begin{itemize}
		\item Welche Topologie/Architektur kann für das Netzwerk samt Backbone eingesttzt werden?
		\item Welche Topoliegie haben Sie für die Umsetzung eines geswitchten LANs benutzt?
	\end{itemize}
\end{frame}

\begin{frame}{IPv4, IPv6}
	\begin{itemize}
		\item Wie sind \emph{IPv4}-, \emph{IPv6}-Adressen aufgebaut?
		\item Wie müssen die Subnetzmasken für Ihre NW gesetzt werden, was gilt es zu beachten?
		\item Wie sehen die Subnetzmasken unter \emph{IPv6} aus?
	\end{itemize}
\end{frame}

\begin{frame}{NAT}
	\begin{itemize}
		\item Was wird unter \emph{NAT} verstanden? Wo und wie wird \emph{NAT} eingesetzt?
	\end{itemize}
\end{frame}

\begin{frame}{Tooling}
	\begin{itemize}
		\item Wie kann die konfiguration der Netzwerkadapter vorgenommen werden?
		\item Mit welchen Werkzeugen könne \emph{IPv4}-, \emph{IPv6}-Adressen und Subnetzmasken gesetzt werden?
		\item Wie können Routing-Tabellen bearbeitet werden?
		\item Wie kann das \emph{NAT} umgesetzt werden?
	\end{itemize}
\end{frame}
\end{document}