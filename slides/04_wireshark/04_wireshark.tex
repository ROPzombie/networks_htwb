%%%%%%%%%%%%%%%%%%%%%%%VICARIOUS%%%%%%%%%%%%%%%%%%%%%%%%%%%%%%%%%%%%%%%
% 																	%
% Template for presentation in Latex`s Beamer Class					%
% Using the default Berlin theme, can be replaced by other themes		%
% logo in the upper right can be replaced by new .png, gif, eps etc	%
% 																	%
%%%%%%%%%%%%%%%%%%%%%%%%%%%%%%%%%%%%%%%%%%%%%%%%%%%%%%%%%%%%%%%%%%%%%%%
\documentclass[xcolor=dvipsnames, aspectratio=169]{beamer}
\usetheme{Berlin}
\usecolortheme[named=LimeGreen]{structure}
\usepackage{beamerthemesplit} % kam neu dazu
\usepackage[ngerman]	{babel}			
\usepackage{t1enc}						
\usepackage[utf8]{inputenc}			
\usepackage{amsmath}
\usepackage{graphicx}
\graphicspath{{pictures/}}
\usepackage{amssymb}
\usepackage{amsfonts}
\usepackage{caption}
\usepackage{multimedia}
\usepackage{tikz}
\usepackage{listings}
\usepackage{acronym}
\usepackage{subfig}

\usepackage{lmodern}
\usepackage{multicol}

\definecolor{pblue}{rgb}{0.13,0.13,1}
\definecolor{pgreen}{rgb}{0,0.5,0}
\definecolor{pred}{rgb}{0.9,0,0}
\definecolor{pgrey}{rgb}{0.46,0.45,0.48}

\lstset{
    escapeinside={(*}{*)}
}


\lstdefinestyle{basic}{  
  basicstyle=\footnotesize\ttfamily,
  breaklines=true
  numbers=left,
  numberstyle=\tiny\color{gray}\ttfamily,
  numbersep=7pt,
  backgroundcolor=\color{white},
  showspaces=false,
  showstringspaces=false,
  showtabs=false,
  frame=single,
  rulecolor=\color{black},
  captionpos=b,
  keywordstyle=\color{blue}\bf,
  commentstyle=\color{gray},
  stringstyle=\color{green},
  keywordstyle={[2]\color{red}\bf},
}


\lstdefinelanguage{custom}
{
morekeywords={public, void},
sensitive=false,
morecomment=[l]{//},
morecomment=[s]{/*}{*/},
morestring=[b]",
}


\lstdefinestyle{BashInputStyle}{
  language=bash,
  showstringspaces=false,
  basicstyle=\small\sffamily,
  numbers=left,
  numberstyle=\tiny,
  numbersep=5pt,
  frame=trlb,
  columns=fullflexible,
  backgroundcolor=\color{gray!20},
  linewidth=0.9\linewidth,
  xleftmargin=0.1\linewidth
}

%Logo in the upper right just change if you know what you are doing^^
\addtobeamertemplate{frametitle}{}{%
\begin{tikzpicture}[remember picture,overlay]
\node[anchor=north east,yshift=2pt] at (current page.north east) {\includegraphics[height=1.8cm]{htw}};
\end{tikzpicture}}

\begin{document}
\bibliographystyle{alpha}
\title{Netzwerke -- Übung WiSe2018/19}
\subtitle{OSI-Stack \& Wireshark\\
		\href{mailto:Benjamin.Troester@HTW-Berlin.de}{Benjamin.Troester@HTW-Berlin.de}\\
		PGP: ADE1 3997 3D5D B25D 3F8F 0A51 A03A 3A24 978D D673 }
\author{Benjamin Tröster}

\date{}

\begin{frame}
\titlepage

\end{frame}

\section*{Road-Map}
\begin{frame}
\frametitle{Road-Map}
\begin{multicols}{2}
  \tableofcontents
\end{multicols}
\end{frame}

\section*{Nachfrage}
\begin{frame}{In English?}
\centering{Soll die Veranstaltung, d.h. Hausaufgaben, Übungsblätter, Slides, Protokolle etc. in Englisch gehalten werden?}

\end{frame}

\section{Hinweis}
\begin{frame}
	\frametitle{Semesterendspurt: Klausur}
	\begin{itemize}
		\item Die Klausuren rücken näher...
		\item Jetzt ist der ideale Zeitpunkt um sich auf Klausuren vorzubereiten.
		\item Stellen Sie fragen zu den VL, Übungsblättern, etc. 
		\item Teilen Sie sich Ihre Zeit sorgfältig ein, lernen Sie konsequent -- es lohnt sich!
		\begin{itemize}
			\item Für die Freunde der Didaktik: Pomodoro-Technik
			\item \url{https://de.wikipedia.org/wiki/Pomodoro-Technik}
		\end{itemize}
	\end{itemize}
\end{frame}

\section{Retrospektive}
\begin{frame}{Retrospektive}
\begin{itemize}
	\item Vorlesung
	\begin{itemize}
		\item Retrospektive der Vorlesung -- was haben Sie behandelt?
		\item Fragen?
	\end{itemize}
\end{itemize}
\end{frame}



\section{Präsentation}
\subsection{Wireshark}
\begin{frame}
\frametitle{1.) Introdcution to Wireshark}
	\begin{itemize}
		\item Erläutern Sie was ein Network-Sniffer ist.
		\item Erklären Sie was Filter sind und wozu diese genutzt werden können.
		\item Zeigen Sie anhand von Beispielen:
		\begin{itemize}
			\item Wie stellen Sie Netzwerkinterfaces ein -- auf welchem Interface soll der Mitschnitt laufen.
			\item Wie filtern Sie nach Protokollen?
			\item Wie filtern Sie \emph{MAC}-Adressen?
			\item Wie filtern Sie \emph{IP}-Adressen?
		\end{itemize}
	\end{itemize}
\end{frame}

\subsection{ARP \& NDP}
\begin{frame}
	\frametitle{2.) ARP}
	\begin{itemize}
		\item Wie sieht das Adressschema für MAC-Adressen aus?
		\item Erläutern Sie wie die Adressauflösung von logischer Adresse (IP) zu physikalischer Adresse vonstatten geht.
		\item Erläutern Sie die Nutzung der beiden Tools \emph{arp} und \emph{ip neigh} anhand von Beispielen.
		\item Nennen und erklären Sie was unter den Begriffen ARP-Tabele und ARP-Cache zu verstehen ist.
	\end{itemize}
\end{frame}

\subsection{ARP-Spoofing}
\begin{frame}
\frametitle{3.) ARP-Spoofing}
	\begin{itemize}
		\item Erläutern Sie was sich hinter dem Akronym \emph{MITM} verbirgt.
		\item Erläutern Sie weiterhin, was APR-Spoofing und ARP-Cache-Poisoning ist.
		\item Diskutieren Sie die Vorgehensweise bei den eben genannten Angriffen. 
		\item \textbf{Anmerkung:} Möglicherweise hilft Ihnen eine Skizze...
	\end{itemize}
\end{frame}


\end{document}
