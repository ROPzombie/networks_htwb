%%%%%%%%%%%%%%%%%%%%%%%VICARIOUS%%%%%%%%%%%%%%%%%%%%%%%%%%%%%%%%%%%%%%%
% 																	%
% Template for presentation in Latex`s Beamer Class					%
% Using the default Berlin theme, can be replaced by other themes		%
% logo in the upper right can be replaced by new .png, gif, eps etc	%
% 																	%
%%%%%%%%%%%%%%%%%%%%%%%%%%%%%%%%%%%%%%%%%%%%%%%%%%%%%%%%%%%%%%%%%%%%%%%
\documentclass[xcolor=dvipsnames, aspectratio=169]{beamer}
\usetheme{Berlin}
\usecolortheme[named=LimeGreen]{structure}
\usepackage{beamerthemesplit} % kam neu dazu
\usepackage[ngerman]	{babel}			
\usepackage{t1enc}						
\usepackage[utf8]{inputenc}			
\usepackage{amsmath}
\usepackage{graphicx}
\graphicspath{{pictures/}}
\usepackage{amssymb}
\usepackage{amsfonts}
\usepackage{caption}
\usepackage{multimedia}
\usepackage{tikz}
\usepackage{listings}
\usepackage{acronym}
\usepackage{subfig}

\usepackage{lmodern}
\usepackage{multicol}

\definecolor{pblue}{rgb}{0.13,0.13,1}
\definecolor{pgreen}{rgb}{0,0.5,0}
\definecolor{pred}{rgb}{0.9,0,0}
\definecolor{pgrey}{rgb}{0.46,0.45,0.48}

\lstset{
    escapeinside={(*}{*)}
}


\lstdefinestyle{basic}{  
  basicstyle=\footnotesize\ttfamily,
  breaklines=true
  numbers=left,
  numberstyle=\tiny\color{gray}\ttfamily,
  numbersep=7pt,
  backgroundcolor=\color{white},
  showspaces=false,
  showstringspaces=false,
  showtabs=false,
  frame=single,
  rulecolor=\color{black},
  captionpos=b,
  keywordstyle=\color{blue}\bf,
  commentstyle=\color{gray},
  stringstyle=\color{green},
  keywordstyle={[2]\color{red}\bf},
}


\lstdefinelanguage{custom}
{
morekeywords={public, void},
sensitive=false,
morecomment=[l]{//},
morecomment=[s]{/*}{*/},
morestring=[b]",
}


\lstdefinestyle{BashInputStyle}{
  language=bash,
  showstringspaces=false,
  basicstyle=\small\sffamily,
  numbers=left,
  numberstyle=\tiny,
  numbersep=5pt,
  frame=trlb,
  columns=fullflexible,
  backgroundcolor=\color{gray!20},
  linewidth=0.9\linewidth,
  xleftmargin=0.1\linewidth
}

%Logo in the upper right just change if you know what you are doing^^
\addtobeamertemplate{frametitle}{}{%
\begin{tikzpicture}[remember picture,overlay]
\node[anchor=north east,yshift=2pt] at (current page.north east) {\includegraphics[height=1.8cm]{htw}};
\end{tikzpicture}}

\begin{document}
\bibliographystyle{alpha}
\title{Netzwerke -- Übung WiSe2018/19}
\subtitle{Application Layer\\
		\href{mailto:Benjamin.Troester@HTW-Berlin.de}{Benjamin.Troester@HTW-Berlin.de}\\
		PGP: ADE1 3997 3D5D B25D 3F8F 0A51 A03A 3A24 978D D673 }
\author{Benjamin Tröster}

\date{}

\begin{frame}
\titlepage

\end{frame}

\section*{Road-Map}
\begin{frame}
\frametitle{Road-Map}
\begin{multicols}{2}
  \tableofcontents
\end{multicols}
\end{frame}

\section{Retrospektive}
\begin{frame}{Retrospektive}
\begin{itemize}[<+->]
	\item Vorlesung
	\begin{itemize}
		\item Retrospektive der Vorlesung -- was haben Sie behandelt?
		\begin{itemize}[<+->]
			\item OSI Schicht 4
		\end{itemize}
		\item Fragen?
	\end{itemize}
	\item Übungsblatt
	\begin{itemize}
		\item Stand des letzten Übungsblatts.
		\begin{itemize}
			\item Fragen?
		\end{itemize}
	\end{itemize}
\end{itemize}
\end{frame}


\section{Präsentation}
\subsection{Tracerouting}
\begin{frame}
\frametitle{1.) Traceroute \& Paris-Traceroute}
	\begin{itemize}
		\item Erläutern Sie was \emph{Traceroute} ist und wie die Umsetzung des erfolgt.
		\item Nennen und erklären Sie die Limitationen von \emph{Traceroute}!
		\item Erläutern Sie die Anomalien die bei verfolgen von Routen entstehen können und warum \emph{Paris-Traceroute} diese auflösen kann.
		\item Erklären Sie anhand von Beispielen die Syntax von \emph{Traceroute} und \emph{Paris-Traceroute}.
	\end{itemize}
\end{frame}

\subsection{DNS}
\begin{frame}
	\frametitle{2.) DNS}
	\begin{itemize}
		\item Erläutern Sie kurz was DNS ist und welche Komponenten das DNS im wesentlichen benötigt um Domainnamen aufzulösen.
		\item Erläutern Sie anhand jeweils eines Beispiels, wie die Namensauflösung stattfinden kann.
		\item Erläutern Sie die Semantik und Syntax der Tool \emph{dig} und \emph{host}. 
		\item Wozu kann das Tool \emph{whois} genutzt werden?
	\end{itemize}
\end{frame}


\subsection{HTTP(S)}
\begin{frame}
	\frametitle{3.)HTTP(S)}
	\begin{itemize}
		\item Erklären Sie die Ihnen bekannten \emph{HTTP}-Methoden (möglicherweise anhand von Beispielen).
		\item Erläutern Sie anhand von Beispielen wie mit den Tools \emph{netcat}, \emph{telnet} bzw. \emph{openssl s\_client} Verbindungen zu einem Server aufgebaut werden können.
		\item Erläutern Sie kurzen was \emph{STARTTLS} bedeutet und wie dies im Groben funktioniert.
	\end{itemize}
\end{frame}

\subsection{Mail via IMAP, POP3 \& SMPT}
\begin{frame}
\frametitle{4.) Mail via IMAP, POP3 \& SMPT}
	\begin{itemize}
		\item Diskutieren Sie kurz anhand der Protokolle \emph{POP3}, \emph{IMAP} und \emph{SMTP} die Bedeutung und Aufgaben für E-Mail. 
		\item Erläutern Sie den Unterschied zwischen \emph{POP3} und \emph{IMAP}?
		\item Erläutern Sie anhand von Beispielen wie Sie nur mithilfe von Kommandozeilenwerkzeugen auf Mails zugreifen können.
	\end{itemize}
\end{frame}
\end{document}
