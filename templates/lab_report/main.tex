%Template for a pretty simple lab report @ HTW Berlin.
%This template is styled to compile with PDFLaTeX!

%start preamble
\documentclass[paper=a4,fontsize=11pt]{scrartcl}%kind of doc, font size, paper size
\usepackage[ngerman]{babel}%for special german letters etc			
\usepackage[T1]{fontenc}%same as t1enc but better						
\usepackage[utf8]{inputenc}%utf-8 encoding, other systems could use others encoding		
\usepackage{amsmath}%get math done
\usepackage{graphicx}%get pictures & graphics done
\graphicspath{{pictures/}}%folder to stash all kind of pictures etc
\usepackage[pdftex,hidelinks]{hyperref}%for links to web
\usepackage{amssymb}%symbolics for math
\usepackage{amsfonts}%extra fonts
\usepackage []{natbib}%citation style
\usepackage{caption}%captions under everything
\usepackage{listings}
\usepackage[titletoc]{appendix}
\numberwithin{equation}{section} 
\usepackage[printonlyused,withpage]{acronym}%how to handle acronyms
\usepackage{float}%for garphics and how to let them floating around in the doc
\usepackage{xcolor}%nicer colors, here used for links
\usepackage{wrapfig}%making graphics floated by text and not done by minipage

\pdfpkresolution=2400%higher resolution

%settings colors for links
\hypersetup{
    colorlinks,
    linkcolor={blue!50!black},
    citecolor={blue},
    urlcolor={blue!80!black}
}

%%here begins the actual document%%

%%starts with title page%%
\begin{document}
\bibliographystyle{alpha}

\begin{titlepage}
%deckblatt start
\thispagestyle{empty}
\begin{center}
\includegraphics[width=0.45\textwidth]{HTW_Logo_rgb}\\
\end{center}
 
 
\begin{center}
\Large{Fachbereich 4 Wirtschaftswissenschaften II}
\end{center}
\begin{verbatim}
 
 
\end{verbatim}
\begin{center}
\textbf{\LARGE{Laborprotokoll}}\\
Netzwerke Übung\\
Sommersemester 2019
\end{center}
\begin{verbatim}
 

\end{verbatim}
\begin{center}
\textbf{im Studiengang Angewandte Informatik (B.Sc.)}
\end{center}
\begin{verbatim}
 
 
\end{verbatim}
 
\begin{flushleft}
\begin{tabular}{lllll}
\textbf{Thema:} & & Hier das Thema bzw. Titel der Aufgabe \\
& & Hier noch mehr Zeug\\
& & \\
& & \\
& & \\
\textbf{eingereicht von:}
& & Name1 \href{mailto: name1@htw-berlin.de}{name1@htw-berlin.de} & matrikel1\\
& & Name2 \href{mailto: name2@htw-berlin.de}{name2@htw-berlin.de} & matrikel2\\
& & Name3 \href{mailto: name3@htw-berlin.de}{name3@htw-berlin.de} & matrikel3\\
& & Name4 \href{mailto: name4@htw-berlin.de}{name4@htw-berlin.de} & matrikel4\\
\\
\textbf{eingereicht am:} & & \today\\
& & \\
& & \\
\end{tabular}
\end{flushleft}
\end{titlepage}

%starting TOC
\newpage
\tableofcontents
\newpage

\section{Einleitung}
\subsection{Aufgabenstellung}
\subsection{Problemstellung}

\section{Hauptteil}

\section{Schluss}

\newpage
\begin{appendices}
%start of acronyms
\section{Abkürzungsverzeichnis}
\begin{acronym}[]
 \acro{SSH}{Secure Shell}
 \acro{SSH-AUTH}{Secure Shell Authentication Protocoll}
 \acro{SSH-CONN}{Secure Shell Connection Protocoll}
 \acro{SSH-ARCH}{Secure Shell Architecture Protocoll}
  \acro{SSH-TRANS}{Secure Shell Transmission Protocoll}
 \acro{NOT}{negiert das Attribute}
 \acro{MUST}{Vorgeschriebene, zwingend durchzuführen}
 \acro{REQUIRED}{benötigt}
 \acro{SHALL}{soll}
 \acro{SHOULD}{sollte}
 \acro{RECOMMENDED}{empfohlen}
 \acro{MAY}{möglich}
 \acro{OPTIONAL}{zusätzlich}
\acro{r}{remote - entfernt}
\acro{DSS}{Digital Signature Standard}
\acro{DSA}{Digital Signature Algorithm}
\acro{RSA}{Rivest Shamir Adleman}
\acro{ECDSA}{Elliptic Curve Digital Signature Algorithm}
\acro{pgp}{Pretty Good Privacy}
\acro{tcp/ip}{Transmission Control Protocol/Internet Protocol}
\acro{OSI}{Open Systems Interconnection}
\acro{RFC}{Request for Comments}
\acro{IETF}{Internet Engineering Task Force}
\acro{ISO}{International Standardization Organisation}
\acro{IANA}{Internet Assigned Numbers Authority}
\acro{DoS}{Denial-of-Service}
\acro{MAC}{Message-Authentication-Code}
\acro{SHA1}{Secure Hash Algorithm}
\acro{NIST}{National Institute of Standards and Technology}
\acro{sftp}{secure file transport protocol}
\acro{AES}{Advanced Encryption Standard}
\acro{PK}{Public Key}
\acro{SAFER}{Secure And Fast Encryption Routine}
\acro{DH}{Diffie-Hellman}
\acro{CBC}{Cypher Block Chain}
\acro{HMAC}{Keyed-Hash Message Authentication Code}
\end{acronym}

%\addcontentsline{toc}{section}{}
\listoffigures

\newpage
\section{Literatur}
\bibliography{literatur}
%\addcontentsline{toc}{}{}
\end{appendices}

\end{document}