\documentclass[paper=a4,fontsize=11pt]{scrartcl}%kind of doc, font size, paper size

\usepackage{fontspec}
\defaultfontfeatures{Ligatures=TeX}
%\setsansfont{Liberation Sans}
\usepackage{polyglossia}	
\setdefaultlanguage[spelling=new, babelshorthands=true]{german}			
\usepackage{amsmath}%get math done
\usepackage{amsthm}%get theorems and proofs done
\usepackage{graphicx}%get pictures & graphics done
\graphicspath{{pictures/}}%folder to stash all kind of pictures etc
\usepackage{amssymb}%symbolics for math
\usepackage{amsfonts}%extra fonts
\usepackage []{natbib}%citation style
\usepackage{caption}%captions under everything
\usepackage{listings}
\usepackage[titletoc]{appendix}
\numberwithin{equation}{section} 
\usepackage[printonlyused,withpage]{acronym}%how to handle acronyms
\usepackage{float}%for garphics and how to let them floating around in the doc
\usepackage{cclicenses}%license!
\usepackage{xcolor}%nicer colors, here used for links
\usepackage{wrapfig}%making graphics floated by text and not done by minipage
\usepackage{dsfont}
\usepackage{stmaryrd}
\usepackage{geometry}
\usepackage{hyperref}
\usepackage{fancyhdr}
\usepackage{csquotes}
\usepackage{tasks}

%settings colors for links
\hypersetup{
    colorlinks,
    linkcolor={blue!50!black},
    citecolor={blue},
    urlcolor={blue!80!black}
}

\pagestyle{fancy}
\lhead{Netzwerke Übung (SoSe 2020)}
\rhead{FB 4 -- Angewandte Informatik\\ HTW-Berlin}
\lfoot{Übungsblatt 05 -- Netzwerksicherheit}
\cfoot{}
\fancyfoot[R]{\thepage}
\renewcommand{\headrulewidth}{0.4pt}
\renewcommand{\footrulewidth}{0.4pt}

\lstdefinestyle{Bash}{
  language=bash,
  showstringspaces=false,
  basicstyle=\small\sffamily,
  numbers=left,
  numberstyle=\tiny,
  numbersep=5pt,
  frame=trlb,
  columns=fullflexible,
  backgroundcolor=\color{gray!20},
  linewidth=0.9\linewidth,
  %xleftmargin=0.5\linewidth
}


%%here begins the actual document%%
\newcommand{\horrule}[1]{\rule{\linewidth}{#1}} % Create horizontal rule command with 1 argument of height

\DeclareMathOperator{\id}{id}

\begin{document}
\begin{center}
\Large{\textbf{Übungsblatt 05 -- Netzwerksicherheit}}
\end{center}
\paragraph{Voraussetzungen:}
Wie in den vorigen Übungen sollte Sie in der Lage sein ein eigenes statisches Netzwerk aufzubauen. D.h. Ihr Netzwerk sollten den Anforderungen des Übungsblattes 3 genügen  -- es gibt einen Router der zwei Subnetze verbindet.

Nachdem Ihr Netzwerk aufgebaut ist:
\begin{itemize}
	\item Überprüfen Sie ob sich Ihre VMs untereinander erreichen können.
	\item Überprüfen Sie ob alle VMs den Uplink erreichen können und ob externe Adresse (d.h. IP-Adressen außerhalb des Labors) erreichbar sind.
	\item Überprüfen Sie ob die Namensauflösung funktioniert.
\end{itemize}


\begin{center}\Large{\textbf{Aufgabe A -- SSH Basics}}\end{center}\vskip0.25in
Die folgenden Aufgaben stellen \emph{openSSH} als Werkzeug für eine sichere Kommunikation zwischen Prozessen (oder Rechnern) vor.\\
Zunächst sollen Sie sich mit dem Umgang mit \emph{SSH} vertraut machen, anschließend werden theoretische Konzepte aus den Hausaufgaben auf die Umsetzung in der Praxis geprüft.
\begin{enumerate}
\item Starten Sie \emph{Wireshark}, sodass Sie den anfallenden Netzwerkverkehr analysieren können.
\begin{enumerate}
	\item Loggen Sie sich via \emph{SSH} auf dem Uranus-Server (\url{uranus-ai.f4.htw-berlin.de}) ein! Achten Sie darauf, dass zu viele Fehlversuche dazu führen, dass der Server Ihren Client blockiert. Das ist gewollt, um Bruteforce-Angriffe auf schlecht geschützte Account zu verhindern.
	\item Erläutern Sie die Bedeuten der Authentizitätsabfrage des \emph{SSH}-Servers ( \enquote{authenticity}).
	\item Welche Bedeutung hat der Fingerprint?
	\item Wie können Sie den Fingerprint prüfen? Mit welchem Programm können Sie sich diesen anzeigen lassen?\\
	Bspw.: \small{ \emph{SHA256:KsUg4lOc91/iJBYFkQhxeI/YGkcnKV2uKUXFNP1ymiw root@xen (ECDSA)}})
	\item Starten Sie in \emph{Wireshark} einen neuen Traffic-Mitschnitt auf dem Ethernet-Netzwerkinterface. Anschließend soll eine neue \emph{SSH}-Session von einem anderen Rechner gestartet werden. Analysieren Sie auszugsweise die entsprechenden Pakete! Welche Teile sind im Plaintext lesbar, ab wann greift die Verschlüsselung?
	\item Sie müssen sich bis jetzt immer via Passwort authentifizieren, d.h. Ihr Login erfolgt aufgrund eines Passworts. Ist Ihr Passwort in einem der ersten Pakete zu finden? Wenn es nicht zu finden ist, wie konnten Sie sich dennoch erfolgreich anmelden? (Welcher kryptografische Mechanismus greift hier ein...)
	\item Wenn Sie die entsprechenden \emph{Wireshark}-Mitschnitte ausgewertet haben, ist Ihnen aufgefallen, dass dort ein \enquote{Key Exchange} stattfindet. Welches kryptografische Verfahren wird dort verwendet und ist dies eine symmetrisches oder asymmetrisches Chiffrierverfahren?
\end{enumerate}
	\item Ermöglichen Sie nun den Login mittels \emph{SSH} zum Uranus-Server \textbf{ohne} das Nutzerpasswort angeben zu müssen. Dies sollten Sie auch für den Verbindungsaufbau zwischen Hosts in Ihrem Netz vornehmen (Debians ohne GUI).\\
	\textbf{Achtung:} Wenn Sie sich auf dem Uranus ohne Passwort anmelden wollen, muss eine bereits existierende \emph{SSH}-Verbindung auf dem Uranus-Server vorhanden sein, da ihr Home-Directory erst im Anschluss gemountet wird und ihr hinterlegter Public-Key ansprechbar ist.\\
	Sie können dabei wie folgt vorgehen:
\begin{enumerate}
	\item Welches Public-Key-Verfahren wollen Sie für eine passwortlose Verschlüsselung nutzen? Welche Länge soll Ihr Key haben? Überlegen Sie sich vor dem eigentlichen generieren, was sinnvoll ist.
	\item Generieren Sie sich ein \emph{SSH}-Schlüsselpaar! Nutzen Sie hierfür die recherchierten Parameter aus Ihren Notizen.
	\item Beim generieren des Schlüssels werden Sie aufgefordert eine Passphrase einzugeben. Der Private-Key ist durch eine Passphrase geschützt, sodass dieser geheime Schlüssel nur von Ihnen geöffnet werden kann.\\
	Wie ist die Passphrase zu wählen? Was gilt es zu beachten?
	\item Verbinden Sie sich von Rechner zu Rechner ohne ein Passwort zu nutzen. D.h. Sie sollten über das eigene LAN hinaus auf eine andere VM via \emph{SSH} zugreifen können. Sie könne sich hierfür ein neuen Nutzer anlegen (\emph{useradd}).
	\item Auf einer der VMs soll die Sicherheit etwas erhöht werden durch Anpassen der Settings.
	\begin{itemize}
		\item Setzen Sie die Anzahl der maximalen Login-Fehlversuche auf drei! 
		\item Erlauben Sie dem Nutzer \emph{student} nur noch das Auflisten des Home-Verzeichnis, wenn er sich via SSH verbunden hat.
		\item Erlauben Sie zugriffe nur noch aus dem LAN 172.16.X.Y.
		\item Setzen Sie als Anmeldeverfahren \emph{SSH} auf reine Public-Key-Kryptografie. Hat dies eventuell auch Nachteile?
	\end{itemize}
\end{enumerate}

\begin{center}\Large{\textbf{Aufgabe B -- SSH-Forwarding}}\end{center}\vskip0.25in
	\item Mit \emph{SSH} können Sie beliebige TCP-Verbindungen über die verschlüsselte \emph{SSH}-Verbindung \enquote{tunneln}. Somit wird es Ihnen möglich, Server zu erreichen, zu denen Sie ansonsten keinen direkten Zugriff hätten (bspw. weil sie hinter einer Firewall befinden oder der Datenverkehr anderweitig gefiltert wird -- Packet-Filtering).\\
	Konfigurieren Sie das Port-Forwarding unter SSH -- ermöglichen Sie dazu folgende Zugriffe:
	\begin{enumerate}
		\item Nehmen Sie ein lokales Port-Forwarding auf die Website der HTW Berlin vor. Hierzu soll ein \emph{SSH}-Tunnel (Source-Port 8080) aufgebaut werden, der den Verkehr auf den Standard HTTP-Port 80 führt (Destination Port 80).
		\item Ihre VM logt sich per \emph{SSH} auf einem anderen VM \emph{SSH}-Server ein und leitet den lokalen Port 2200 auf den Port 22 des dortigen Systems weiter. Danach sollten Sie sich mit \emph{SSH} über den lokalen Port mit dem \emph{SSH}-Server des fremden \emph{SSH}-Server verbinden können.
		\item Konfigurieren Sie ein Remote-Port-Forwarding -- stellen Sie dazu eine \emph{SSH}-Verbindung vom einer anderen VM zu Ihrer VM als \emph{SSH}-Server her. Leiten Sie den Port 8880 des \emph{SSH}-Server nun über Ihren Client zum Webserver der URL \url{www.htw-berlin.de} weiter. Im folgenden kann sich nun jede VM mit Ihrer VM auf Port 8880 verbinden, um die Webseite der HTW-Berlin zu besuchen.
	\end{enumerate}
\end{enumerate}

\begin{center}\Large{\textbf{Aufgabe C -- VPN via Wireguard}}\end{center}\vskip0.25in
In den Hausaufgaben haben Sie sich bereits mit dem Wireguard-Tutorial auseinandergesetzt. Nun ist es an der Zeit ein eigenes VPN aufzusetzen.
\begin{enumerate}
	\item Setzen Sie mithilfe Wireguards ein eigenes VPN um. Ihr Router sollte die Aufgabe des VPN-Servers übernehmen. Somit können sich die VMs beim VPN-Server anmelden und sind anschließend Teil dieses VPNs.
	\item 
	\begin{itemize}
		\item Als Server müssen Sie einen entsprechenden VPN-Server aufsetzten. Im wesentlichen müssen Sie dem Tutorial folgen.
		\item Nehmen Sie entsprechen die Konfiguration der Clients vor, sodass Sie sich mit dem Server verbinden können.
	\end{itemize}
	\item Finden Sie heraus, ob Sie tatsächlich Teil des VPN-Netzwerkes sind. Wie können Sie überprüfen, ob Sie tatsächlich Teilnehmer eines virtuellen Netzwerkes sind?
	\item Untersuchen Sie via \emph{Wireshark} wie der Traffic aussieht. Können Sie den kryptografischen Handshake ausfindig machen?
\end{enumerate}

\end{document}