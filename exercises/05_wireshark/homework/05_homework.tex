%start preamble
\documentclass[paper=a4,fontsize=11pt]{scrartcl}%kind of doc, font size, paper size
\usepackage[ngerman]{babel}%for special german letters etc			
%\usepackage{t1enc} obsolete, but some day we go back in time and could use this again
\usepackage[T1]{fontenc}%same as t1enc but better						
\usepackage[utf8]{inputenc}%utf-8 encoding, other systems could use others encoding
%\usepackage[latin9]{inputenc}			
\usepackage{amsmath}%get math done
\usepackage{amsthm}%get theorems and proofs done
\usepackage{graphicx}%get pictures & graphics done
\graphicspath{{pictures/}}%folder to stash all kind of pictures etc
\usepackage{amssymb}%symbolics for math
\usepackage{amsfonts}%extra fonts
\usepackage []{natbib}%citation style
\usepackage{caption}%captions under everything
\usepackage{listings}
\usepackage[titletoc]{appendix}
\numberwithin{equation}{section} 
\usepackage[printonlyused,withpage]{acronym}%how to handle acronyms
\usepackage{float}%for garphics and how to let them floating around in the doc
\usepackage{cclicenses}%license!
\usepackage{xcolor}%nicer colors, here used for links
\usepackage{wrapfig}%making graphics floated by text and not done by minipage
\usepackage{dsfont}
\usepackage{stmaryrd}
\usepackage{geometry}
\usepackage{hyperref}
\usepackage{fancyhdr}
\usepackage{menukeys}

%settings colors for links
\hypersetup{
    colorlinks,
    linkcolor={blue!50!black},
    citecolor={blue},
    urlcolor={blue!80!black}
}

\definecolor{pblue}{rgb}{0.13,0.13,1}
\definecolor{pgreen}{rgb}{0,0.5,0}
\definecolor{pred}{rgb}{0.9,0,0}
\definecolor{pgrey}{rgb}{0.46,0.45,0.48}

\pagestyle{fancy}
\lhead{Netzwerke Übung (SoSe 2019)}
\rhead{FB 4 -- Angewandte Informatik\\ HTW-Berlin}
\lfoot{Übungsblatt 05 -- Wireshark}
\cfoot{}
\fancyfoot[R]{\thepage}
\renewcommand{\headrulewidth}{0.4pt}
\renewcommand{\footrulewidth}{0.4pt}

\lstdefinestyle{Bash}{
  language=bash,
  showstringspaces=false,
  basicstyle=\small\sffamily,
  numbers=left,
  numberstyle=\tiny,
  numbersep=5pt,
  frame=trlb,
  columns=fullflexible,
  backgroundcolor=\color{gray!20},
  linewidth=0.9\linewidth,
  %xleftmargin=0.5\linewidth
}

\newlength\labelwd
\settowidth\labelwd{\bfseries viii.)}
\usepackage{tasks}
\settasks{counter-format =tsk[a].), label-format=\bfseries, label-offset=3em, label-align=right, label-width
=\labelwd, before-skip =\smallskipamount, after-item-skip=0pt}
\usepackage[inline]{enumitem}
\setlist[enumerate]{% (
labelindent = 0pt, leftmargin=*, itemsep=12pt, label={\textbf{\arabic*.)}}}

\pdfpkresolution=2400%higher resolution

%%here begins the actual document%%
\newcommand{\horrule}[1]{\rule{\linewidth}{#1}} % Create horizontal rule command with 1 argument of height

\DeclareMathOperator{\id}{id}

\begin{document}
\begin{center}
\Large{\textbf{Übungsblatt 5 -- Wireshark}}
\end{center}
\begin{center}\Large{\textbf{Aufgabe A -- Wireshark}}\end{center}\vskip0.25in
Nachdem Sie sich in den letzten Übungen vor allem mit dem \glqq handwerklichen\grqq\ Aufsetzen von Netzwerken beschäftigt haben, schauen Sie sich in der kommenden Übung genauer an, was im Netzwerkstack vor sich geht. D.h. Sie analysieren, was im Inneren eines Netzwerkes passiert, wie die Realisierung der Protokolle aussieht und ob sich die theoretischen Ideen aus der Vorlesung auch in der Praxis wiederfinden lassen.\\
Um all dies in Angriff nehmen zu können nutzen wir den Netzwerk-Sniffer \emph{Wireshark}. Wireshark ist eine Open-Source-Software, mithilfe dessen Analysen, Fehlerbehebungen, Software- und Protokollkommunikation untersucht werden können. \emph{Wireshark} ähnelt \emph{tcpdump} in gewisser Weise, jedoch bietet \emph{Wireshark} ein grafisches Frontend (GUI), sodass die Analysen visuell ansprechend dargestellt werden können.\footnote{Wireshark kann natürlich auch als \emph{CLI} genutzt werden.} \\
Hilfreiche Links:
\begin{itemize}
	\item \url{https://en.wikipedia.org/wiki/Wireshark}
	\item \url{https://www.lifewire.com/wireshark-tutorial-4143298}
	\item \url{https://www.wireshark.org/download/docs/user-guide.pdf}
	\item \url{https://wiki.wireshark.org/}
	\item \url{https://wiki.wireshark.org/CaptureFilters}
	\item \url{https://wiki.wireshark.org/DisplayFilters}
\end{itemize}
\begin{enumerate}
	\item Rekapitulieren Sie Ihr Wissen zum OSI-Modell. 
	\item Erläutern Sie was in Netzwerken unter Datenkapselung verstanden wird.	
	\item Lesen Sie folgendes Tutorial in Hinblick auf die Fragen in Aufgabe 4.): \url{https://tinyurl.com/yby2kukf}
	%\url{https://cs.gmu.edu/~astavrou/courses/ISA_674_F12/Wireshark-Tutorial.pdf} 
	, alternativ gibt es ein Wireshark 101 unter: \url{https://youtu.be/f4zqMDzXt6k} 
	\footnote{Lohnenswert ist das Wireshark 101 Buch im PDF Format (eine Suche bei Google hilft bestimmt).}
	\item Nachdem Sie die Tutorials abgearbeitet haben:
	\begin{tasks}(1)
		\task~ Was ist ein \emph{Network-Sniffer}?
		\task~ Wozu kann ein Netzwerk-Sniffer genutzt werden?
		\task~ Verschaffen Sie sich einen Überblick, sodass Sie einen Überblick haben wo was zu finden ist, bzw. wo Sie Hilfe finden können.
		\task~ Recherchieren Sie wozu Filter in \emph{Wireshark} eingesetzt werden.
		\task~ Bringen Sie in Erfahrung wie Filter genutzt werden.
		\task~ Welche beiden unterschiedlichen Mitschnitt-Modi (Caputre Modes) bietet \emph{Wireshark}? Worin unterscheiden sich diese?
	\end{tasks}
	\item Erläutern Sie anhand von Beispielen den grundlegende Umgang mit \emph{Wireshark}. \footnote{Vielleicht sind Screenshots von Wireshark sinnvoll.}
	\begin{tasks}(1)
		\task~ Wie stellen Sie Netzwerkinterfaces ein -- auf welchem Interface soll der Mitschnitt laufen.
		\task~ Wie filtern Sie nach Protokollen?
		\task~ Wie filtern Sie \emph{MAC}-Adressen?
		\task~ Wie filtern Sie \emph{IP}-Adressen?
	\end{tasks}
	\item \textbf{Fakultativ:} Wenn Sie mögen, können Sie \emph{Wireshark} auf Ihren Gerät(en) installieren oder in der Virtuellen Maschine laufen lassen und ihren Netzwerkverkehr ein wenig analysieren. 
	\begin{itemize}
		\item \url{https://www.wireshark.org/download.html}
		\item \url{https://www.wireshark.org/docs/wsug_html_chunked/ChapterBuildInstall.html}
	\end{itemize}
\end{enumerate}

\begin{center}\Large{\textbf{Aufgabe B -- Address Resolution Protocol (ARP) \& Neighbor Discovery Protocol (NDP)}}\end{center}\vskip0.25in
Es sollte Ihnen aufgefallen sein, dass in der zweiten Übung (Geswitchte Netze) Ihr Netzwerk in der Planung zwar IP-Adressen nutzt, aber kein Router Verwendung fand. Der verwendete Switch ist ein OSI-Layer 2 Gerät und kommt ohne IP-Adressen zurecht. Ihre Raspberry Pis verlangen jedoch zwingend eine IP-Adresse von Ihnen.\\
Um den Knoten ein wenig zu lösen, schauen Sie sich das \emph{Address Resolution Protocol (ARP)} an.
\begin{enumerate}
		\item Recherchieren Sie mithilfe folgenden links: \url{https://en.wikipedia.org/wiki/Address_Resolution_Protocol}, was \emph{ARP} ist und wie dies funktioniert.
		\item Wie adressiert ein Switch die Pakte zwischen den Endknoten (also den Raspberry Pis)? \textbf{Hinweis:} Wie oben bereits erwähnt geschieht dies nicht mittels IP-Adressen.
		\item Erläutern Sie das Adressschema von \emph{MAC}-Adressen. Kann dieses Adressschema auch zu Problemen führen?
		\item Da unser Uplink (Gateway des Labors) \glqq nur\grqq\ das alte \emph{IPv4} spricht ist \emph{ARP} notwendig. Unter \emph{IPv6} gibt es kein \emph{ARP}, wie wird dies dort gehandhabt?
		\item Recherchieren Sie wozu die Werkzeuge \emph{arp} und \emph{ip neigh} in unixoiden Betriebssystemen genutzt werden können.
		\item Recherchieren Sie die grundlegende Syntax und Semantik von \emph{arp} sowie \emph{ip neigh}.
\end{enumerate}

\begin{center}\Large{\textbf{Aufgabe C -- MITM \& ARP-Cache}}\end{center}\vskip0.25in
Da Sie in den Übungen bis dato hauptsächlich Infrastruktur aufgebaut haben, soll dieser Teil als Vorbereitung für einen ersten Einblick in die Netzwerksicherheit bieten. Aufgrund Ihres Wissen ahnen Sie schon, dass mit ein wenig List viele Netzwerke leicht manipulierbar sind. \footnote{Oftmals genügt es wie der böse Wolf ein wenig zu pusten (Die drei kleinen Schweinchen)...}
\begin{enumerate}
	\item Vergegenwärtigen Sie sich, was ein Cache ist und wozu dieser eingesetzt wird. Anschließend daran, recherchieren Sie was es mit dem ARP-Cache auf sich hat.
	\item Erläutern Sie wie der ARP-Cache-Mechanismus funktioniert.
	\item Recherchieren Sie was ein Man-In-The-Middle-Angriff (\emph{MITM}) ist.
	\item Da der ARP-Cache keinerlei Validierungsmöglichkeiten hat, ist ein Manipulation des ARP-Caches möglich. Überlegen Sie sich zunächst, welche Schritte hierfür notwendig wären, wenn Sie als Angreifer, den Cache eines anderen Systems verändern wollen.
	\begin{itemize}
		\item Welche Voraussetzungen müssen gegeben sein?
		\item Welche Informationen über das Angriffsziel benötigen Sie?
		\item Welche Schritte muss ein Angriff auf den ARP-Cache folgen?
	\end{itemize}
	\item Im Moodle (sowie auf den Raspberry Pis) steht ein Angrifftool für das sogenannte ARP-Spoofing bzw. ARP-Cache-Poisoning bereit. Lesen dieses Skript und versuchen Sie es weitestgehend zu verstehen. Notieren Sie sich den Ablauf! Bzw. Fragen zu den Stellen im Quellcode, die Sie nicht verstehen.
\end{enumerate}

\begin{center}\Large{\textbf{Aufgabe D -- Ethernet \& UDP|TCP/IP}}\end{center}\vskip0.25in
Da Sie in der kommenden Übung mithilfe \emph{Wiresharks} Ihren Netzwerkverkehr untersuchen sollen, müssen Sie zumindest grundlegend verstanden haben, auf welche Protokolle Sie dort stoßen werden. Natürlich können wir uns nicht alle Protokolle en dé­tail anschauen, die wichtigsten für die kommende Übung sollten Sie jedoch kennen.
\begin{enumerate}
	\item Recherchieren Sie die Funktion, sowie den Aufbau von Ethernet (\emph{IEEE 802.3} Protokollfamilie).
	\begin{tasks}(1)
		\task~ Auf welcher Ebene im OSI-Modell arbeitet \emph{Ethernet}?
		\task~ Welche Aufgabe übernimmt das oben genannte Protokoll?
		\task~ Aus welchen Segmenten besteht ein Ethernet-Frame?
		\task~ Zeigen Sie beispielhaft den Aufbau eine Ethernet-Frames.
	\end{tasks}
	\item Recherchieren Sie die Funktion, sowie den Aufbau des IP-Protokolls (\emph{IPv4}).
	\begin{tasks}(1)
		\task~ Auf welcher Ebene im OSI-Modell arbeitet IP?
		\task~ Welche Aufgabe übernimmt das oben genannte Protokoll?
		\task~ Aus welchen Segmenten besteht ein IP-Paket?
		\task~ Zeigen Sie beispielhaft den Aufbau eine IP-Pakets.
	\end{tasks}
	\item Recherchieren Sie die Funktion, sowie den Aufbau des \emph{TCP}-Protokolls.
	\begin{tasks}(1)
		\task~ Auf welcher Ebene im OSI-Modell arbeitet \emph{TCP}?
		\task~ Welche Aufgabe übernimmt das oben genannte Protokoll?
		\task~ Aus welchen Segmenten besteht ein \emph{TCP}-Datagram?
		\task~ Zeigen Sie beispielhaft den Aufbau eine \emph{TCP}-Datagram.
	\end{tasks}
	\item Recherchieren Sie die Funktion, sowie den Aufbau des \emph{UDP}-Protokolls.
	\begin{tasks}(1)
		\task~ Auf welcher Ebene im OSI-Modell arbeitet \emph{UDP}?
		\task~ Welche Aufgabe übernimmt das oben genannte Protokoll?
		\task~ Aus welchen Segmenten besteht ein \emph{UDP}-Datagram?
		\task~ Zeigen Sie beispielhaft den Aufbau eine \emph{UDP}-Datagram.
	\end{tasks}
	\item Worin unterscheiden sich \emph{TCP} und \emph{UDP} grundlegend?
\end{enumerate}

\begin{center}\Large{\textbf{Aufgabe E -- Routing \& Traceroute}}\end{center}\vskip0.25in
Nachdem Sie komplexere Netzwerke aufgesetzt haben, betrachten wir in der kommenden Übung bereits erste Anwendungen eines Netzwerkes. Viele dieser Applications nutzen Sie bereits, teilweise ohne es bewusst wahrgenommen zu haben. Da Sie als angehende Netzwerkprofis aber nicht nur daran interessiert sind Dinge zu nutzen, sondern deren Aufbau zu verstehen, sollen Sie eines der Fundamente des Internets besser verstehen -- das Routing.\\  
Hilfreiche Links:
\begin{itemize}
	\item \url{https://en.wikipedia.org/wiki/Traceroute}
	\item \url{https://paris-traceroute.net/} Achtung HTTPS ist kaputt!
\end{itemize}
\begin{enumerate}
	\item Im wesentlichen gibt es zwei fundamentale Routing-Algorithmen. Dies sind das Distanz-Vektor- und Link-State-Routing. Um den kürzesten Weg durch einen Graphen zu finden (\emph{Shortest-Path-Problem}), wird für das Distanz-Vektor-Routing gewöhnlich der Bellman-Ford-Algorithmus verwandt, das Link-State-Routing nutzt den Dijkstra-Algorithmus.
	\begin{tasks}(1)
		\task~ Recherchieren Sie wie das Link-State-Routing unter Nutzung des Dijkstra-Algorithmus funktioniert.
		\task~ Recherchieren Sie wie das Distanz-Vektor-Routing unter Nutzung des Bellman-Ford-Algorithmus funktioniert.
		\task~ Erläutern Sie die fundamentalen Unterschiede beider Lösungsansätze.
		\task~ Diskutieren Sie ob der Bellman-Ford-Algorithmus (bzw. warum nicht) für das Link-State-Routing und der Dijkstra-Algorithmus für das Distanz-Vektor-Routing genutzt werden könnte.
	\end{tasks}	
	\item Lesen Sie folgende Artikel:\\
	\url{https://en.wikipedia.org/wiki/Traceroute},\\
	\url{https://linux.die.net/man/8/traceroute}.\\
	Beantworten Sie anschließend folgende Fragen:
	\begin{tasks}(1)
		\task~ Wofür wird Traceroute genutzt?
		\task~ Wie wird Traceroute umgesetzt, d.h. wie läuft eine \glqq Routen-Verfolgung\grqq\ ab?
		\task~ Welche Limitationen ergeben sich aus der Umsetzung?
		\task~ Dokumentieren Sie die Syntax, sowie die Bedeutung von Traceroute beispielhaft.
	\end{tasks}
	\item Lesen Sie folgendes Paper zu Paris-Traceroute von der ACM International Measurement Conference 2006:\\
	\url{http://conferences.sigcomm.org/imc/2006/papers/p15-augustin.pdf}
	\begin{tasks}(1)
		\task~ Warum ist eine \glqq neue\grqq\ Traceroute-Applikation notwendig?
		\task~ Nennen Sie drei Topologie-Anomalien die durch Paris-Traceroute erkannt werden können.
	\end{tasks}
\end{enumerate}

\end{document}
