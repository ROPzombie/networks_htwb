%start preamble
\documentclass[paper=a4,fontsize=11pt]{scrartcl}%kind of doc, font size, paper size

\usepackage{fontspec}
\defaultfontfeatures{Ligatures=TeX}
%\setsansfont{Liberation Sans}
\usepackage{polyglossia}	
\setdefaultlanguage[spelling=new, babelshorthands=true]{german}

\usepackage{amsmath}%get math done
\usepackage{amsthm}%get theorems and proofs done
\usepackage{graphicx}%get pictures & graphics done
\graphicspath{{pictures/}}%folder to stash all kind of pictures etc
\usepackage{amssymb}%symbolics for math
\usepackage{amsfonts}%extra fonts
\usepackage []{natbib}%citation style
\usepackage{caption}%captions under everything
\usepackage{listings}
\usepackage[titletoc]{appendix}
\numberwithin{equation}{section} 
\usepackage[printonlyused,withpage]{acronym}%how to handle acronyms
\usepackage{float}%for garphics and how to let them floating around in the doc
\usepackage{wrapfig}%making graphics floated by text and not done by minipage
\usepackage{geometry}
\usepackage{hyperref}
\usepackage{fancyhdr}
\usepackage{menukeys}
\usepackage{xcolor}%nicer colors, here used for links
\usepackage{csquotes}
\usepackage{enumitem}

%settings colors for links
\hypersetup{
    colorlinks,
    linkcolor={blue!50!black},
    citecolor={blue},
    urlcolor={blue!80!black}
}

\definecolor{pblue}{rgb}{0.13,0.13,1}
\definecolor{pgreen}{rgb}{0,0.5,0}
\definecolor{pred}{rgb}{0.9,0,0}
\definecolor{pgrey}{rgb}{0.46,0.45,0.48}

\pagestyle{fancy}
\lhead{Netzwerke -- Übung\\Wintersemester 2020/21}
\rhead{FB 4 -- Angewandte Informatik\\ HTW-Berlin}
\lfoot{Übungsblatt 02 -- Netzwerkinfrastruktur Teil 1}
\cfoot{}
\fancyfoot[R]{\thepage}
\renewcommand{\headrulewidth}{0.4pt}
\renewcommand{\footrulewidth}{0.4pt}

\lstdefinestyle{Bash}{
  language=bash,
  showstringspaces=false,
  basicstyle=\small\sffamily,
  numbers=left,
  numberstyle=\tiny,
  numbersep=5pt,
  frame=trlb,
  columns=fullflexible,
  backgroundcolor=\color{gray!20},
  linewidth=0.9\linewidth,
  %xleftmargin=0.5\linewidth
}

%%here begins the actual document%%
\newcommand{\horrule}[1]{\rule{\linewidth}{#1}} % Create horizontal rule command with 1 argument of height

\DeclareMathOperator{\id}{id}

\begin{document}
\begin{center}
\Large{\textbf{Übungsblatt 2 -- Netzwerkinfrastruktur}}
\end{center}

\begin{center}\Large{\textbf{Aufgabe A - Wiederholung Vorlesung\\Planung des physischen Netzes}}\end{center}

Sie planen ein kleines Netzwerk, bestehend aus drei Rechnern (also drei VMs). Hierfür sollen sie zunächst die Infrastruktur planen.
\begin{enumerate}
	\item \textbf{Wiederholung:} Recherchieren sie mithilfe \cite[S. 461ff]{Kurose2012}  was eine Netzwerktopologie ist.
	\item  \textbf{Wiederholung:} Lesen Sie Kapitel 5.4 in \cite[S. 461]{Kurose2012} zum Thema Switched Local Area Networks.
	\item \textbf{Wiederholung:} Lesen Sie Kapitel 4.3 in \cite[S. 320ff]{Kurose2012} zum Thema Routing und Router.
	\item Ihr Netzwerk soll aus drei Rechnern bestehen. Diese drei Rechner sind über einen Switch verbunden. In der Virtualisierung haben sie keinen physischen Switch, wir konfigurieren lediglich die Virtualisierungsumgebung, sodass das Netzwerk, wie ein Switched Network arbeitet.
	Wählen Sie eine geeignete Netztopologie und skizzieren Sie diese mit geeigneten Symbolen.\\ 
	\textbf{Hinweis:} Unter \url{http://iacis.org/iis/2008/S2008_967.pdf} finden Sie auf S. 241 eine Möglichkeit, wie dies aussehen könnte.\\
	\item Planen Sie die Netzkonfiguration:
	\begin{enumerate}
		\item  \textbf{Wiederholung:} Rekapitulieren sie was eine IP-Adresse ist. Welche Aufgabe haben diese Adressentypen in einem Netzwerk? s. \cite[S. 331ff]{Kurose2012}
		\item Ordnen sie \emph{IP} im OSI-Modell ein!
		\item Momentan werden vor allem \emph{IPv4} und \emph{IPv6} als Netzwerkschichtprotokolle genutzt. Recherchieren sie einige wichtige Unterschiede zwischen \emph{IPv4} und \emph{IPv6}.
		\item  \textbf{Wiederholung:} Rekapitulieren sie was eine Subnetzmaske ist und wofür diese gebraucht wird.
		\item \textbf{Wiederholung:} Wie spielen IP-Adresse und Subnetzmaske zusammen?
		\item Bestimmte IP-Adressbereiche werden nicht ins Internet weitergeleitet, sie werden als private IP-Adressen bezeichnet. Diese Adressen gibt es sowohl unter \emph{IPv4} als auch unter \emph{IPv6}. Recherchieren sie, welche IP-Adressbereiche nicht ins Internet geroutet werden.
		\item \textbf{Wiederholung:} Wählen Sie beispielhaft eine Netzwerkadresse (IP-Addresse -- ip address) und Subnetzmaske (subnet mask) für einen möglichst kleinen IP-Adressbereich, die genau für ihre Anzahl Rechner ausreicht.\\
		Wie sähe die Subnetzmaske für $7, 23, 42$ oder $72$ Rechner aus? 
		\item Ich habe ein Video für sie bereitgestellt, dass zeigt, wie sie ein virtuelles Netzwerk mit virtualBox organisieren können \footnote{Mit Anmeldung in der HTW-Mediathek: \url{https://mediathek.htw-berlin.de/video/Virtualbox-Network-Preperations-amp-Cloning/276fab5dbd663d7589d12a30234da003}}. Schauen Sie sich das Video an und setzen Sie das Netzwerk um! Die Images finden Sie im Moodle oder via \url{https://cloud.htw-berlin.de/s/JNwcn7LL2qgt87n}. Ein FreeBSD können sie ebenfalls nutzen, wenn Sie möchten: \url{https://cloud.htw-berlin.de/s/yxn6ZQDczxrED4w}
		\item Die IP-Range für das Netzwerk ist $172.16.0.0/24$ und $172.16.1.0/24$. Wie viele Maschinen könnten sie untergebracht werden? 
		\item Sie sollen jedoch Ihre Netzwerke minimal planen. Welche Netzadressen und Subnetzmasken müssen Sie in Ihre Skizze eintragen?
	\end{enumerate}
\end{enumerate}

\begin{center}\Large{\textbf{Aufgabe B -- Tools}}\end{center}
Ziel der nächsten Übung ist es das Netzwerk nicht nur theoretisch, sondern auch praktisch umzusetzen. Daher sollen sie die Nutzung einiger Tools in Erfahrung bringen.\\
Mithilfe der Werkzeugsammlungen \emph{iproute2} sowie \emph{net-tools} wird dies in der Regel unter Linux und Unix-Betriebssystemen bewerkstelligt.
\begin{enumerate}
	\item Im ersten Übungsblatt haben Sie bereits das Rechtemodell kennengelernt. Verschiedene Nutzer*innen haben verschiedene Rechte. Für die Konfiguration des Systems soll im allgemeinen nur bestimmte Nutzer*innen zuständig sein. Recherchieren Sie welche Rechte der \emph{root}-User hat und was das Kommando \emph{sudo} in diesem Zusammenhang leistet.
	\item In Betriebssystemen gibt es verschiedene Hintergrunddienste (Daemons), die die Verwaltung des Systems in Teilen organisieren. Da \emph{freeBSD} das Betriebssystem unser Wahl ist, kommt \emph{System V} zum Einsatz.
	\begin{enumerate}
		\item \emph{System-V/service} verfügt über die Möglichkeit bestimmte Dienste zu starten, stopen, etc. Recherchieren sie wie der entsprechende Befehl lautet. Die Man-Pages oder der Link der Fußnote sind gute Anlaufstellen!
		\footnote{\url{https://www.freebsd.org/doc/handbook/configtuning-rcd.html}}\\
		Notieren sie sich die Syntax Wort für Wort, sowie die Bedeutung jedes Wortes (Tokens). 
		\item Wichtige Dienste für die nächste Laborübung ist der \emph{dhcp} . Notieren Sie sich:
		\begin{enumerate}
			\item Wie der Status eines Daemons/Dienstes abgefragt werden kann.
			\item Wie ein Daemon/Dienstes gestartet, gestoppt werden kann.
			\item Wie ein Daemon/Dienstes permanent eingeschaltet bzw. ausgeschaltet werden kann (d.h. auch nach einem Neustart automatisch gestartet werden kann.)
		\end{enumerate}
	\end{enumerate}
	\item Übliche Befehle zum Einrichten von Netzwerkadaptern sind \emph{ifconfig} (BSD \emph{net-tools}) oder auch \emph{ip} aus der Werkzeugsammlung \emph{iproute2}. Der Befehl \emph{ifconfig} gilt in manchen Linux-Distributionen als veraltet (In BSD, Solaris etc. ist dies nicht der Fall!). Recherchieren sie kurz, worin sich beide Tools-Sammlungen unterscheiden und notieren sie sich wesentliche Unterschiede.
 	\item Bringen sie in Erfahrung, wie sie die Konfiguration bereits existierender Netzwerkkonfigurationen mit den Tools \emph{ip} und \emph{ifconfig} in Erfahrung bringen.
	\item Recherchieren und notieren sie sich, wie mithilfe des Befehls \emph{ip addr} Netzwerkadapter(n) eine (oder mehrere) IP-Adressen und Subnetzmasken zugewiesen wird.\\
	Wie wird dies mit \emph{ifconfig} gehandhabt? \footnote{\url{http://linux-ip.net/linux-ip/linux-ip.pdf} Appendix C: S 108}\\
	(Auch hier gilt: Notieren Sie sich das Kommando sowie dessen Bedeutung Wort/Schrittweise)!
	\item Recherchieren sie, wie die IP-Konfiguration in einer Datei festlegen und speichern können, sodass diese weiterhin nach einem Neustart gültig ist.  \footnote{\url{https://www.freebsd.org/doc/de_DE.ISO8859-1/articles/linux-users/network.html}, \url{http://linux-ip.net/linux-ip/linux-ip.pdf} Kapitel 1 S 4.}
	\begin{enumerate}
		\item In welcher Datei wird unter \emph{freeBSD} die Konfiguration abgelegt?
		\item Welcher User kann auf diese Datei zugreifen?
		\item Notieren sie sich, wie eine Konfiguration beispielhaft aussieht und was die einzelnen Zeilen bedeuten!
	\end{enumerate}
	\item Recherchieren sie wie der Status eines Netzwerkadapters mit den \emph{net-tools} und \emph{iproute2} abgefragt werden kann.\\
	Welche Stati kann ein Adapter besitzen und wie kann der Status geändert werden?
	\item Recherchieren Sse beispielhaft wie eine persistente Lösung aussieht. Kommentieren sie 	Ihr Beispiel anschließend, sodass Sie wissen was die einzelnen Wörter/Token bedeuten. Dies ist im wesentlich die Lösung für die Umsetzung im Labor.
	\item \emph{ICMP} ist ein Diagnose-Protokoll, dass Sie bei der Wartung/Nutzung von Netzwerken unterstützt. Recherchieren Sie welchen Hinweis Ihnen dabei die Folgenden \emph{ICMP}-Messages geben. Wo wird jeweils der Fehler in der Konfiguration liegen?
	\begin{enumerate}
		\item Connect: network is unreachable
		\item Destination Host Unreachable
		\item Destination Network Unreachable
		\item keine Antwort auf ein Ping
	\end{enumerate}
	\item Zwei weitere bekannte Netzwerkanalyse-Tools sind \emph{netstat} (aus \emph{net-tools}) und \emph{ss} (aus der \emph{iproute2}-Werkzeugsammlung).
	\begin{enumerate}
		\item Recherchieren sie die wesentliche Funktionen von \emph{netstat}, sowie \emph{ss} \footnote{\url{http://linux-ip.net/linux-ip/linux-ip.pdf} Kapitel 4, S 151ff; für \emph{ss} \url{https://www.linux.com/topic/networking/introduction-ss-command/}}.
		\item Notieren sie sich anhand von Beispielen die Syntax der eben genannten Tools. 
	\end{enumerate}
\end{enumerate}

\begin{center}\Large{\textbf{Aufgabe C -- Ping}}\end{center}
Um festzustellen ob eine Verbindung funktionstüchtig ist, wird oftmals das Tool \emph{ping} genutzt. D.h. \emph{ping} analysiert, ob Datenpakete überhaupt und wie viele Pakete von einem Host (bspw. Ihrem Rechner) zu einem Ziel (wie etwa der Webserver der HTW-Berlin) gelangen. Falls sie ein wenig mehr zu Ping recherchieren wollen, kann ich Ihnen folgenden Artikel empfehlen: \url{https://openmaniak.com/ping.php}
\begin{enumerate}
	\item Recherchieren sie die Syntax von \emph{ping}. Ein guter Anlaufpunkt wäre die Man-Page (\emph{man ping}) oder \url{https://linux.die.net/man/8/ping}.
\end{enumerate}
\begin{center}\Large{\textbf{Aufgabe D -- Laborblatt}}\end{center}
\begin{enumerate}
	\item Lesen sie vorbereitend das Laborübungsblatt einmal komplett durch.
	\item Notieren sie sich alle aufkommenden Fragen und Unklarheiten!
	\item Stellen sie ggf. Fragen (Online oder am Anfang der Übung)! 
\end{enumerate}
\bibliographystyle{plain}
\bibliography{sources}
\end{document}


