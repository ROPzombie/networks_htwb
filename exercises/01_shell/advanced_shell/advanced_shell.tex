%%%%%%%%%%%%%%%%%%%%%%%%%%%%%%%%%%%%%%%%%%%%%%%%%%%%%%%%%%%%%%%%%%%%%%%%%%
%%LaTeX template for papers && theses									%%
%%Done by the incredible ||Z01db3rg||									%%
%%Under the do what ever you want license								%%
%%%%%%%%%%%%%%%%%%%%%%%%%%%%%%%%%%%%%%%%%%%%%%%%%%%%%%%%%%%%%%%%%%%%%%%%%% 

%start preamble
\documentclass[paper=a4,fontsize=11pt]{scrartcl}%kind of doc, font size, paper size
\usepackage[ngerman]{babel}%for special german letters etc			
%\usepackage{t1enc} obsolete, but some day we go back in time and could use this again
\usepackage[T1]{fontenc}%same as t1enc but better						
\usepackage[utf8]{inputenc}%utf-8 encoding, other systems could use others encoding
%\usepackage[latin9]{inputenc}			
\usepackage{amsmath}%get math done
\usepackage{amsthm}%get theorems and proofs done
\usepackage{graphicx}%get pictures & graphics done
\graphicspath{{pictures/}}%folder to stash all kind of pictures etc
\usepackage[pdftex,hidelinks]{hyperref}%for links to web
\usepackage{amssymb}%symbolics for math
\usepackage{amsfonts}%extra fonts
\usepackage []{natbib}%citation style
\usepackage{caption}%captions under everything
\usepackage{listings}
\usepackage[titletoc]{appendix}
\numberwithin{equation}{section} 
\usepackage[printonlyused,withpage]{acronym}%how to handle acronyms
\usepackage{float}%for garphics and how to let them floating around in the doc
\usepackage{cclicenses}%license!
\usepackage{xcolor}%nicer colors, here used for links
\usepackage{wrapfig}%making graphics floated by text and not done by minipage
\usepackage{dsfont}
\usepackage{stmaryrd}
\usepackage{geometry}
\usepackage{hyperref}
\usepackage{fancyhdr}
\usepackage{menukeys}

\pagestyle{fancy}
\lhead{Benjamin Tröster\\Netzwerke Übung (WiSe2018/19)}
\rhead{FB 4 -- Angewandte Informatik\\ HTW-Berlin}
\lfoot{Grundlagen Linux \& Shell}
\cfoot{}
\fancyfoot[R]{\thepage}
\renewcommand{\headrulewidth}{0.4pt}
\renewcommand{\footrulewidth}{0.4pt}

\lstdefinestyle{Bash}{
  language=bash,
  showstringspaces=false,
  basicstyle=\small\sffamily,
  numbers=left,
  numberstyle=\tiny,
  numbersep=5pt,
  frame=trlb,
  columns=fullflexible,
  backgroundcolor=\color{gray!20},
  linewidth=0.9\linewidth,
  %xleftmargin=0.5\linewidth
}


\newlength\labelwd
\settowidth\labelwd{\bfseries viii.)}
\usepackage{tasks}
\settasks{counter-format =tsk[a].), label-format=\bfseries, label-offset=3em, label-align=right, label-width
=\labelwd, before-skip =\smallskipamount, after-item-skip=0pt}
\usepackage[inline]{enumitem}
\setlist[enumerate]{% (
labelindent = 0pt, leftmargin=*, itemsep=12pt, label={\textbf{\arabic*.)}}}

\pdfpkresolution=2400%higher resolution

%settings colors for links
%\hypersetup{
 %   colorlinks,
  %  linkcolor={blue!50!black},
   % citecolor={blue},
    %urlcolor={blue!80!black}
%}

%\usepackage[pagetracker=true]{biblatex}

%%here begins the actual document%%
\newcommand{\horrule}[1]{\rule{\linewidth}{#1}} % Create horizontal rule command with 1 argument of height


\DeclareMathOperator{\id}{id}

\begin{document}
\center
\Large{Fakultative Hausaufgaben -- Shell Grundlagen}
\begin{center}\Large{\textbf{Shell}}\end{center}\vskip0.25in
%\setlist[enumerate, 1]{itemsep=\baselineskip}
\begin{enumerate}
	\item Mit welchem Kommando können Sie ...
   \begin{tasks}(1)
   \task~ Handbuchseiten (\glqq Man Pages\grqq) öffnen
   \task~ das aktuelle Verzeichnis in der Shell ausgeben?
   \task~ den Inhalt eines Verzeichnisses in der Shell ausgeben?
   \task~ eine leere Datei erzeugen?
   \task~ versuchen den Inhalt einer Datei zu bestimmen?
   \task~ den Inhalt verschiedener Dateien verknüpfen oder den Inhalt einer Datei ausgeben?
   \task~ Zeilen vom Ende einer Datei in der Shell ausgeben?
   \task~ Zeilen vom Anfang einer Datei in der Shell ausgeben?
   \task~ ein leeres Verzeichnis löschen?
   \task~ eine Zeichenkette in der Shell ausgeben?
   \task~ Das Password eines Benutzers ändern?
   \task~ das System neu starten?
   \task~ das System ausschalten
   \task~ einen neuen Benutzer erstellen?
   \task~ einen Benutzer löschen?
   \task~ einen Benutzer ändern?
   \task~ eine Liste der laufenden Prozesse in der Shell ausgeben?
   \task~ einen Prozess beenden?
   \task~ eine Gruppe von Prozessen beenden?
   \task~ eine Liste der existierenden Prozesse als Baumstruktur in der Shell ausgeben?
   \end{tasks}
      \end{enumerate}
   \begin{center}\Large{\textbf{CLI-Editoren}}\end{center}\vskip0.25in
   \begin{enumerate}
   		\item Jedes Linux/ Unix verfügt über den Editor \emph{vi}, bearbeiten Sie folgendes Tutorial:\\
  \url{https://www.tutorialspoint.com/unix/unix-vi-editor.htm}\\
  Alternativ können Sie auch mit dem \emph{vim} arbeiten, dieser ist eine Erweiterung des \emph{vi} und etwas einfacher zu bedienen.
  		\item Alternativ steht auch der Editor \emph{emacs} zur Verfügung. Entsprechend können Sie folgendes Tutorial durcharbeiten:\\
  		\url{https://www.gnu.org/software/emacs/tour/}
\end{enumerate}

\begin{center}\Large{\textbf{Umgebungsvariablen, Links \& Default-Shell}}\end{center}\vskip0.25in
%\setlist[enumerate, 1]{itemsep=\baselineskip}
\begin{enumerate}
\item Was sind Umgebungsvariablen, wozu werden diese gebraucht?
\item Ändern Sie die PS1-Variablen ihrer Shell derart um, dass Ihre Shell-Umgebung den Nutzernamen @ Hostnamen gefolgt von der Urzeit und in einer neuen Zeile den aktueller Pfad gefolgt von einem Leerzeichen und dem \$-Zeichen. Womit ihre Shell etwa wie folgt aussehen sollte:
\begin{figure}[H]
\includegraphics[scale=0.6]{ps1}
\end{figure}
\item Worin besteht der Unterschied zwischen Hard- und Soft-Links?
\item Als Standard-Shell ist auf den Laborrechnern als BASH voreingestellt, wie könnte man dies ändern? Auf dem Raspberry Pi ist momentan die Bourne-Again-Shell (bassh) eingestellt, als Alternativen stehen die Bourne-Shell (BASH) und die Z-Shell (ZSH) (mitsamt OhMyZSH) zur Verfügung. Es gibt noch viele weitere Kommandozeileninterpreter. Recherchieren Sie, was man unter dem Begriff Kommandozeileninterpreter versteht. \footnote{Suchen nach dem Unterschied zwischen interpretierten und kompilierten Sprachen!}.\\
Versuchen Sie ganz wesentliche Unterschiede zwischen Bourne-Shell und BASH/ZSH herauszufinden.
\item Auf den Laborrechnern befinden sich alle eben genannten Shells, Sie können ihre Default-Shell mithilfe des Befehls\\
    \begin{lstlisting}[style=Bash, language=Bash]
chsh [-s /path/to/shell] [s05XXXXX]
		\end{lstlisting}
		ändern. 
\end{enumerate}
\begin{center}\Large{\textbf{Advanced Tools -- grep, awk \& sed}}\end{center}\vskip0.25in
Jede Shell unter Unix hat einige sehr mächtige Tools (sed \& awk sind eigentlich richtige Programmiersprachen!). Diese Werkzeuge sind wahre \glqq allrounder\grqq\ in Sachen Textverarbeitung.
\begin{enumerate}
	\item global regular expression print (grep) ist eine Werkzeugsammlung, mit deren Hilfe Texte und Datein durchsucht werden kann. Durch den Einsatz von Regulären Ausdrücken (regualar expressions -- kurz regex) kann dies äußert effizient sein. Bearbeiten Sie folgendes Tutorial: \url{https://www.uccs.edu/~ahitchco/grep/} oder \url{https://www.thegeekstuff.com/2009/03/15-practical-unix-grep-command-examples/}
	\begin{tasks}(1)
		\task~
	\end{tasks}
	\item awk (Aho Weinberger Kernighan) ist eine schon etwas ältere Programmiersprache von Alfred Aho (Dragon Book -- Compilerbau), Peter J. Weinberger und Brian Kernighan (Unix, C).\\
	Bearbeiten Sie folgendes Tutorial: \url{https://www.tutorialspoint.com/awk/index.htm}
	\item sed (stream editor) ist eine Programmiersprache und Unix-Tool von Lee E. McMahon (comm, qsort, grep) zur effizienten Textbearbeitung.\\
	Bearbeiten Sie folgendes Tutorial: \url{https://www.tutorialspoint.com/sed/index.htm}
	\item Wenn Sie weiteren Übungsbedarf haben kann ich Folgende Seite empfehlen:\\
	\url{https://www.hackerrank.com/domains/shell/bash}\\
	Hier finden Sie eine schöne Sammlung exemplarischer Aufgaben, die mit den oben genannten Tools gelöst werden kann. Darüber hinaus bietet hackerrank eine Fülle an Übungsaufgaben in den Bereichen Algorithmen, Datenstrukturen, Programmiersprachen etc.
\end{enumerate} 
\end{document}