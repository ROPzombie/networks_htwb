%start preamble
\documentclass[paper=a4,fontsize=11pt]{scrartcl}%kind of doc, font size, paper size

\usepackage{fontspec}
\defaultfontfeatures{Ligatures=TeX}
%\setsansfont{Liberation Sans}
\usepackage{polyglossia}	
\setdefaultlanguage[spelling=new, babelshorthands=true]{german}

\usepackage{amsmath}%get math done
\usepackage{graphicx}%get pictures & graphics done
\graphicspath{{pictures/}}%folder to stash all kind of pictures etc
\usepackage{amssymb}%symbolics for math
\usepackage{amsfonts}%extra fonts
\usepackage{caption}%captions under everything
\usepackage{listings}
\usepackage[titletoc]{appendix}
\usepackage[printonlyused,withpage]{acronym}%how to handle acronyms
\usepackage{float}%for garphics and how to let them floating around in the doc
\usepackage{xcolor}%nicer colors, here used for links
\usepackage{wrapfig}%making graphics floated by text and not done by minipage
\usepackage{dsfont}
\usepackage{geometry}
\usepackage{hyperref}
\usepackage{fancyhdr}
\usepackage{multicol}
\usepackage{tasks}
\usepackage{csquotes}

%settings colors for links
\hypersetup{
    colorlinks,
    linkcolor={blue!50!black},
    citecolor={blue},
    urlcolor={blue!80!black}
}

\definecolor{pblue}{rgb}{0.13,0.13,1}
\definecolor{pgreen}{rgb}{0,0.5,0}
\definecolor{pred}{rgb}{0.9,0,0}
\definecolor{pgrey}{rgb}{0.46,0.45,0.48}

\pagestyle{fancy}
\lhead{Netzwerke Übung (SoSe 2020)}
\rhead{Angewandte Informatik\\ HTW-Berlin}
\lfoot{Übungsblatt 03 -- Wireshark}
\cfoot{}
\fancyfoot[R]{\thepage}
\renewcommand{\headrulewidth}{0.4pt}
\renewcommand{\footrulewidth}{0.4pt}

\lstdefinestyle{Bash}{
  language=bash,
  showstringspaces=false,
  basicstyle=\small\sffamily,
  numbers=left,
  numberstyle=\tiny,
  numbersep=5pt,
  frame=trlb,
  columns=fullflexible,
  backgroundcolor=\color{gray!20},
  %linewidth=0.9\linewidth,
  %xleftmargin=0.5\linewidth
}

%%here begins the actual document%%
\newcommand{\horrule}[1]{\rule{\linewidth}{#1}} % Create horizontal rule command with 1 argument of height

\DeclareMathOperator{\id}{id}

\begin{document}
\begin{center}
\Large{\textbf{Übungsblatt 04 -- Wireshark}}\\
\end{center}

\begin{center}\Large{\textbf{Aufgabe A -- TCP: 3-Way-Handshake}}\end{center}\vskip0.2in
Nachdem Sie sich bereits theoretisch mit dem 3-Way-Handshake auseinandergesetzt haben, sollen Sie nun schauen, ob der TCP-Handshake tatsächlich wie theoretisch beschrieben arbeitet.
\begin{enumerate}
	\item Überlegen Sie sich eine Anfragen an eine Website (dies sollte TCP nutzen, wie HTTP!), die Sie noch nicht von der VM aus getätigt haben. Da ansonsten bestimmte Inhalte bereits gecacht vorliegen könnten oder über andere Verfahren eine TCP-Handshake vereiteln könnten.
	\item Starten Sie Wireshark, richten Sie Interface und Protokoll-Type ein. Filtern Sie nur auf eine speziellen Request!
	\item Lösen Sie den Handshake durch aufrufen der Website (oder Ressource) aus, während Wireshark den Netzverkehr mitschneidet.
	\item Analysieren Sie den 3-Way-Handshake!
	\item Zum Vergleich: Analysieren Sie ihren Mitschnitt mit folgender Aufzeichnung: \url{https://wiki.wireshark.org/TCP_3_way_handshaking?action=AttachFile&do=view&target=3-way+handshake.pcap}
\end{enumerate}

\begin{center}\Large{\textbf{Aufgabe B -- ICMP}}\end{center}\vskip0.2in
Da die Befehle \emph{ping} und \emph{traceroute} \emph{ICMP} nutzen, sollen Sie mit Wireshark solche Request mitverfolgen.
\begin{enumerate}
	\item Setzen Sie alle notwendigen Parameter um Wireshark mitlaufen zu lassen, sodass Sie die ICMP-Nachrichten mitverfolgen können.
	\item Pingen Sie einen Rechner mit seinem Namen an (bspw.: \url{mi.fu-berlin.de}).
	\item Ping auf eine IP-Adresse (bspw.: 160.45.117.199).
	\item Ping auf die IP-Adresse Ihres Routers. \\\textbf{Hinweis:} Sie können diese durch \emph{ip r} oder \emph{route} in Erfahrung bringen. 
	\begin{lstlisting}[style=Bash, language=Bash]
ip r
default via XXX.XXX.XXX dev DEVICE proto dhcp src YOU.RIP.ADD metric VALUE
#or
route -n
Destination     	Gateway         	Genmask		Flags 	Metric 		Ref    	Use 	Iface
0.0.0.0          	XXX.XXX.XXX   	0.0.0.0         	UG    	VALUE    	0        	0 		DEVICE
	\end{lstlisting}
	\item Ping auf meine eigene IP-Adresse.
	\item Ping auf die Loopback-Adresse.
	\item Starten Sie eine Routenverfolgung via \emph{traceroute} auf eine beliebige Adresse. Verfolgen Sie dabei den Ausgabe auf der Konsole als auch in Wireshark (Filtern Sie in Wireshark entsprechend). Spiegeln sich die Einträge in Wireshark mit denen auf der Kommandozeile?
	\item Erläutern Sie die Ergebnisse Ihrer vorigen Aufgabe. Wie funktioniert \emph{traceroute} und wie hängt dies mit \emph{ICMP} zusammen?
\end{enumerate}

\begin{center}\Large{\textbf{Aufgabe C -- Routing \& Traceroute}}\end{center}\vskip0.25in
Nachdem Sie recherchiert haben, wie \emph{traceroute} arbeitet, welche Kritik an Traceroute geäußert wurde und wie diese mit dem Tool Paris-Traceroute abgestellt wurden, sollen beide Tools hier kurz erprobt werden.
\begin{enumerate}
	\item Überlegen Sie sich zunächst anhand Ihrer Recherche was \emph{traceroute} in etwa ausgeben müsste, wenn Sie im Labor eine Route von einem Rechner $A$ zu einem Rechner $B$ verfolgen würden. Wobei beide Rechner zu unterschiedlichen Netzwerken gehören (d.h. unterschiedlichen Tischreihen). 
	\item Nutzen Sie anschließend \emph{traceroute} um sich die Router zwischen zwei Laborrechnern anzeigen zu lassen. Stimmen Ihre theoretische Überlegungen mit denen von \emph{traceroute} überein? Falls nicht, sollten Sie analysieren woran dies liegen könnte.
	\item Vergleichen Sie die Ausgaben von \emph{traceroute} und \emph{paris-traceroute} für folgende IP-Adressen:
	\begin{enumerate}
		\item 41.231.21.44
		\item 91.198.174.192
		\item 37.220.21.130
		\item 80.239.142.229
	\end{enumerate}
	\textbf{Hinweis}: Für \emph{paris-traceroute} sollten Sie den \enquote{exhaustive algorithm} Nutzen (in machen Versionen als Parameter: \texttt{-na exhaustive})
	\item Analysieren Sie anschließend die Ausgabe beider Tools.
	\item Warum wurde Ihnen eine Liste von IP-Adressen genannt anstelle von Domainnamen? Nennen Sie mindestens zwei Gründe!
\end{enumerate}

\end{document}