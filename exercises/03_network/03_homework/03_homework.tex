%start preamble
\documentclass[paper=a4,fontsize=11pt]{scrartcl}%kind of doc, font size, paper size

\usepackage{fontspec}
\defaultfontfeatures{Ligatures=TeX}
%\setsansfont{Liberation Sans}
\usepackage{polyglossia}	
\setdefaultlanguage[spelling=new, babelshorthands=true]{german}

\usepackage{amsmath}%get math done
\usepackage{amsthm}%get theorems and proofs done
\usepackage{graphicx}%get pictures & graphics done
\graphicspath{{pictures/}}%folder to stash all kind of pictures etc
\usepackage{amssymb}%symbolics for math
\usepackage{amsfonts}%extra fonts
\usepackage []{natbib}%citation style
\usepackage{caption}%captions under everything
\usepackage{listings}
\usepackage[titletoc]{appendix}
\numberwithin{equation}{section} 
\usepackage[printonlyused,withpage]{acronym}%how to handle acronyms
\usepackage{float}%for garphics and how to let them floating around in the doc
\usepackage{wrapfig}%making graphics floated by text and not done by minipage
\usepackage{geometry}
\usepackage{hyperref}
\usepackage{fancyhdr}
\usepackage{menukeys}
\usepackage{xcolor}%nicer colors, here used for links
\usepackage{csquotes}
\usepackage{enumitem}

%settings colors for links
\hypersetup{
    colorlinks,
    linkcolor={blue!50!black},
    citecolor={blue},
    urlcolor={blue!80!black}
}

\definecolor{pblue}{rgb}{0.13,0.13,1}
\definecolor{pgreen}{rgb}{0,0.5,0}
\definecolor{pred}{rgb}{0.9,0,0}
\definecolor{pgrey}{rgb}{0.46,0.45,0.48}

\pagestyle{fancy}
\lhead{Netzwerke Übung (SoSe 2020)}
\rhead{FB 4 -- Angewandte Informatik\\ HTW-Berlin}
\lfoot{Übungsblatt 03 -- Static Networks}
\cfoot{}
\fancyfoot[R]{\thepage}
\renewcommand{\headrulewidth}{0.4pt}
\renewcommand{\footrulewidth}{0.4pt}

\lstdefinestyle{Bash}{
  language=bash,
  showstringspaces=false,
  basicstyle=\small\sffamily,
  numbers=left,
  numberstyle=\tiny,
  numbersep=5pt,
  frame=trlb,
  columns=fullflexible,
  backgroundcolor=\color{gray!20},
  linewidth=0.9\linewidth,
  %xleftmargin=0.5\linewidth
}

%%here begins the actual document%%
\newcommand{\horrule}[1]{\rule{\linewidth}{#1}} % Create horizontal rule command with 1 argument of height

\DeclareMathOperator{\id}{id}

\begin{document}
\begin{center}
\Large{\textbf{Übungsblatt 3 -- Static Networks}}
\end{center}

\begin{center}\Large{\textbf{Aufgabe A - Planung des physischen Netzes}}\end{center}

Sie planen ein kleine Netzwerk, bestehend aus drei Rechnern (also drei VMs). Hierfür sollen Sie zunächst die Infrastruktur planen.
\begin{enumerate}
	\item Machen Sie sich die Funktion der einzelnen Rechner- \& Netzwerkkomponenten klar. Wie sieht Ihre Infrastruktur aus? Gehen Sie hierbei auf den Rechner ein, der das Netzwerk virtualisiert, als auch Ihre tatsächliche physische Infrastruktur.
	\item \textbf{Recap:} Recherchieren Sie mithilfe \cite[S. 461ff]{Kurose2012}  was eine Netzwerktopologie ist.
	\item \textbf{Recap:} Lesen Sie Kapitel 4.3 in \cite[S. 320ff]{Kurose2012} zum Thema Routing und Router.
	\item Ihr Netzwerk soll aus drei Rechnern bestehen, zwei sind einfache Host, die nur am Netzwerk teilnehmen, ein Rechner soll als Router fungieren.\\
	Wählen Sie eine geeignete Netztopologie und skizzieren Sie diese mit geeigneten Symbolen.\\ 
	\textbf{Hinweis:} Unter \url{http://iacis.org/iis/2008/S2008_967.pdf} finden Sie auf S. 241 eine Möglichkeit, wie dies aussehen könnte.\\
	\item Planen Sie die Netzkonfiguration:
	\begin{enumerate}
		\item Rekapitulieren Sie kurz was eine IP-Adresse ist. Welche Aufgabe haben diese Adressen in einem Netzwerk? s. \cite[S. 331ff]{Kurose2012}
		\item Momentan werden vor allem \emph{IPv4} und \emph{IPv6} als Netzwerkschichtprotokolle genutzt. Recherchieren Sie einige wichtige Unterschiede zwischen \emph{IPv4} und \emph{IPv6}.
		\item Rekapitulieren Sie was eine Subnetzmaske ist und wofür diese gebraucht wird.
		\item Wie spielen IP-Adresse und Subnetzmaske zusammen?
		\item Bestimmte IP-Adressbereiche werden nicht ins Internet weitergeleitet, sie werden als private IP-Adressen bezeichnet. Diese Adressen gibt es sowohl unter \emph{IPv4} als auch unter \emph{IPv6}. Recherchieren Sie, welche IP-Adressbereiche nicht ins Internet geroutet werden.
		\item Wählen Sie beispielhaft eine Netzwerkadresse (IP-Addresse -- ip address) und Subnetzmaske (subnet mask) für einen möglichst kleinen IP-Adressbereich, der genau für zwei Rechner ausreicht. Wie sähe die Subnetzmaske für 7, 23, 42 oder 72 Rechner aus? 
		\item Ich habe ein Video für Sie bereitgestellt, dass zeigt, wie Sie ein virtuelles Netzwerk mit VirtualBox organisieren können \footnote{Mit Anmeldung in der HTW-Mediathek: \url{https://mediathek.htw-berlin.de/video/Virtualbox-Network-Preperations-amp-Cloning/276fab5dbd663d7589d12a30234da003}}. Schauen Sie sich das Video an und setzen Sie das Netzwerk um! Die Images finden Sie im Moodle oder via \url{https://cloud.htw-berlin.de/s/JNwcn7LL2qgt87n}. Ein FreeBSD können sie ebenfalls nutzen, wenn Sie möchten: \url{https://cloud.htw-berlin.de/s/yxn6ZQDczxrED4w}
		\item Die IP-Range für das Netzwerk ist $172.16.0.0/24$ und $172.16.1.0/24$. Wie viele Maschinen könnten Sie prinziepiell hier unterbringen? 
		\item Sie sollen jedoch Ihre Netzwerke minimal planen. Welche Netzadressen und Subnetzmasken müssen Sie in Ihre Skizze eintragen?
	\end{enumerate}
\end{enumerate}

\begin{center}\Large{\textbf{Aufgabe B -- Routing}}\end{center}

\begin{enumerate}
	\item Was sind die Aufgaben eines Routers. Wie erfolgt, im Groben, die Umsetzung des Routings?
	\item Machen Sie sich klar, wie Router und \emph{IP}-Protokoll zusammenhängen.
	\item In welcher Schicht des OSI-Modells würden Sie einen Router einordnen? (Begründung!)
	\item Wie haben sich bis jetzt Ihre VMs gefunden? Woher "'wussten" diese, welches Gerät gemeint war?
	\item Woher weiß ein \textit{Host} (Endknoten), wann er ein Paket direkt adressieren kann und wann er es an Router/Gateway weiter schicken muss?
	\item Woher weiß ein \textit{Router} (Zwischenknoten), wann er ein Paket weiter schicken soll und wann nicht?
\end{enumerate}

\begin{center}\Large{\textbf{Aufgabe C -- Tools}}\end{center}
Ziel der nächsten Übung ist es das Netzwerk nicht nur theoretisch, sondern auch praktisch umzusetzen. Daher sollen Sie die Nutzung einiger Tools in Erfahrung bringen.\\
Mithilfe der Werkzeugsammlungen \emph{iproute2} sowie \emph{net-tools} wird dies in der Regel unter Linux und Unix-Betriebssystemen bewerkstelligt.
\begin{enumerate}
	\item Im ersten Übungsblatt haben Sie bereits das Rechtemodell kennengelernt. Verschiedene Nutzer*innen haben verschiedene Rechte. Für die Konfiguration des Systems soll im allgemeinen nur bestimmte Nutzer*innen zuständig sein. Recherchieren Sie welche Rechte der \emph{root}-User hat und was das Kommando \emph{sudo} in diesem Zusammenhang leistet.
	\item In Betriebssystemen gibt es verschiedene Hintergrunddienste (Daemons), die die Verwaltung des Systems in Teilen organisieren. Da Debian das Betriebssystem auf der meisten VMs ist, kommt Systemd zum Einsatz \footnote{Wenn FreeBSD eingesetzt wird, haben Sie kein systemd, s. \url{https://www.freebsd.org/doc/handbook/configtuning-rcd.html}}. \footnote{Eigentlich war Systemd als Alternative des System-V Init-Daemons gedacht, hat aber über die Zeit immer mehr Funktionalitäten bekommen.}
	\begin{enumerate}
		\item \emph{systemd} verfügt über die Möglichkeit bestimmte Dienste zu starten, stopen, etc. Recherchieren Sie wie der entsprechende Befehl lautet. Die Man-Pages oder der Link der Fußnote sind gute Anlaufstellen!
		\footnote{\url{https://wiki.archlinux.org/index.php/Systemd\#Using_units}}\\
		Notieren Sie sich die Syntax Wort für Wort, sowie die Bedeutung jedes Wortes (Tokens). 
		\item Wichtige Dienste für die nächste Laborübungen sind der Networking-Service und DHCPclient. Notieren Sie sich:
		\begin{enumerate}
			\item Wie der Status eines Daemons abgefragt werden kann.
			\item Wie ein Daemon gestartet, gestoppt werden kann.
			\item Wie ein Daemon permanent eingeschaltet bzw. ausgeschaltet werden kann (d.h. auch nach einem Neustart automatisch gestartet werden kann.)
		\end{enumerate}
	\end{enumerate}
	\item Übliche Befehle zum Einrichten von Netzwerkadaptern sind \emph{ifconfig} (BSD \emph{net-tools}) oder auch \emph{ip} aus der Werkzeugsammlung \emph{iproute2}. Der Befehl \emph{ifconfig} gilt in manchen Linux-Distributionen als veraltet (In BSD, Solaris etc. ist dies nicht der Fall!). Recherchieren Sie kurz, worin sich beide Tools-Sammlungen unterscheiden und notieren Sie sich wesentliche Unterschiede.\\
	Digital Ocean hat ein schönes HowTo dazu: \url{goo.gl/w1MN5x}
 	\item Bringen Sie in Erfahrung, wie Sie die Konfiguration bereits existierende Netzwerkkonfigurationen mit den Tools \emph{ip} und \emph{ifconfig} in Erfahrung bringen.
	\item Recherchieren und notieren Sie sich, wie mithilfe des Befehls \emph{ip addr} Netzwerkadapter(n) eine (oder mehrere) IP-Adressen und Subnetzmasken zugewiesen wird.\\
	Wie wird dies mit \emph{ifconfig} gehandhabt. \footnote{\url{http://linux-ip.net/linux-ip/linux-ip.pdf} Appendix C: S 108}\\
	(Auch hier gilt: Notieren Sie sich das Kommando sowie dessen Bedeutung Wort/Schrittweise)!
	\item Recherchieren Sie, wie Sie die IP-Konfiguration in einer Datei festlegen und speichern können, sodass diese weiterhin nach einem Neustart gültig ist.  \footnote{\url{http://linux-ip.net/linux-ip/linux-ip.pdf} Kapitel 1 S 4.}\\
	\textbf{Achtung:} Bedenken Sie für welches Betriebssystem diese Konfiguration erfolgen soll!
	\begin{enumerate}
		\item In welcher Datei wird unter Debian/FreeBSD die Konfiguration abgelegt?
		\item Welcher User kann auf diese Datei zugreifen?
		\item Notieren Sie sich, wie eine Konfiguration beispielhaft aussieht und was die einzelnen Zeilen bedeuten!
	\end{enumerate}
	\item Recherchieren Sie wie der Status eines Netzwerkadapters mit den \emph{net-tools} und \emph{iproute2} abgefragt werden kann.\\
	Welche Stati kann ein Adapter besitzen und wie kann der Status geändert werden?
	\item Recherchieren Sie, welches Werkzeug aus \emph{iproute2} genutzt werden kann um Routen zu setzen.\\Notieren Sie sich die Syntax und was die Parameter bewerkstelligen.
	\item Analog dazu: Wie werden Routen mithilfe der \emph{net-tools} konfiguriert?
	\item Recherchieren Sie beispielhaft wie eine persistente Lösung aussähe. Kommentieren Sie 	Ihr Beispiel anschließend, sodass Sie wissen was die einzelnen Wörter/Token bedeuten.
	\item Gateways \& Router -- Gateways sind im allgemeinen nicht das Gleiche wie Router. Auch unter den Gateways gibt es Unterscheidungen.
	\begin{enumerate}
		\item Recherchieren Sie worin sich Router und Gateways unterscheiden.
		\item Beim aufsetzen des Netzwerkes kann unterschieden werden zwischen \emph{Gateways} und \emph{Default-Gateways}. Recherchieren Sie diese Unterscheidung.
	\item Lesen Sie Kapitel 4.1.1. \cite[S. 308ff]{Kurose2012} zum Thema Routing und Forwarding. Worin unterscheiden sich diese Mechanismen?
	\item Der Betriebssystemkern (Kernel) stellt die Infrastruktur für das Routing bereit. Die dafür genutzte Datenstruktur ist der Routing-Table. \footnote{\url{http://linux-ip.net/linux-ip/linux-ip.pdf} Kapitel 8, S 38ff}
	\item Rekapitulieren Sie den Unterschied zwischen Routing und Forwarding.
	\item Welchen Kernel-Parameter müssen Sie aktivieren (bzw. welche Datei im \path{/proc/sys} Verzeichnis müssen sie editieren), sodass das IP-Forwarding aktiviert wird? Welche Möglichkeiten zum Editieren dieser Datei haben Sie?
	\item In welcher Konfigurationsdatei müssen Sie einen Eintrag vornehmen, so das das Routing dauerhaft beim Systemstart aktiviert bleibt? Notieren Sie sich beispielhaft (auszugsweise) wie dies aussehen kann.
	\item \emph{ICMP} ist ein Diagnose-Protokoll, dass Sie bei der Wartung/Nutzung von Netzwerken unterstützt. Recherchieren Sie welchen Hinweis Ihnen dabei die Folgenden \emph{ICMP}-Messages geben. Wo wird jeweils der Fehler in der Konfiguration liegen?
	\begin{enumerate}
		\item Connect: network is unreachable
		\item Destination Host Unreachable
		\item Destination Network Unreachable
		\item keine Antwort auf ein Ping
	\end{enumerate}
	\item Zwei weitere bekannte Netzwerkanalyse-Tools sind \emph{netstat} (aus \emph{net-tools}) und \emph{ss} (aus der \emph{iproute2}-Werkzeugsammlung).
	\begin{enumerate}
		\item Recherchieren Sie die wesentliche Funktionen von \emph{netstat}, sowie \emph{ss} \footnote{\url{http://linux-ip.net/linux-ip/linux-ip.pdf} Kapitel 4, S 151ff; für \emph{ss} \url{https://www.linux.com/topic/networking/introduction-ss-command/}}.
		\item Notieren Sie sich anhand von Beispielen die Syntax der eben genannten Tools. 
	\end{enumerate}
	\item \emph{iptables} wird unter Linux allgemein als Firewall-Tool genutzt.\\
	In der kommenden Übung übernimmt \emph{iptables} eine etwas andere Aufgabe. Es sorgt zunächst dafür, dass unsere VMs via \emph{NAT}\footnote{Um genau zu sein: SNAT} Pakte in das Internet leiten können. Der Router arbeitet als \emph{NAT}-Gateway -- dieser übersetzt unsere privaten IP-Adressen auf Adressen auf die eigenen Adressen. 
	\begin{enumerate}
		\item Recherchieren Sie entweder unter \cite[S. 349f]{Kurose2012} oder mithilfe folgenden Links was \emph{NAT} ist und warum dies unter \emph{IPv4} genutzt wird.\\
		\url{https://en.wikipedia.org/wiki/Network_address_translation}
		\item Machen Sie sich im groben klar, wie \emph{NAT} umgesetzt wird.
		\item Recherchieren Sie kurz welche \emph{NAT}-Varianten es gibt. 
		\item \textbf{Fakultativ:} Mit sehr hoher Wahrscheinlichkeit nutzt auch Ihr Router/Modem \emph{NAT}, wie wird dies hier umgesetzt?
		\item Recherchieren Sie was unter einer Firewall im wesentlichen verstanden wird.
		\item Machen Sie sich klar, wie eine Firewall im Groben umgesetzt wird -- wie setzt eine Firewall seine Regeln um.
		\item \emph{iptables} kann genutzt werden, um die privaten Adressen auf öffentliche zu übersetzen. Lesen Sie folgenden Artikel:\\
		\url{https://access.redhat.com/documentation/en-US/Red_Hat_Enterprise_Linux/4/html/Security_Guide/s1-firewall-ipt-fwd.html}\\
		Versuchen Sie den Inhalt wirklich komplett zu verstehen. Notieren Sie sich alle notwendigen Schritte um das Masquerading via \emph{iptables} einzuschalten.
		\item \emph{iptables} unterstützt sowohl \emph{SNAT} (One-to-Many-NAT) als auch \emph{DNAT}. Recherchieren Sie kurz worin sich beide Arten unterscheiden.
	\end{enumerate}
	\end{enumerate}
\end{enumerate}

\begin{center}\Large{\textbf{Aufgabe C -- Ping}}\end{center}
Um festzustellen ob eine Verbindung funktionstüchtig ist, wird oftmals das Tool \emph{ping} genutzt. D.h. \emph{ping} analysiert ob Datenpakete überhaupt und wie viele Pakete von einem Host (bspw. Ihrem Rechner) zu einem Ziel (wie etwa der Webserver der HTW-Berlin) gelangen. Falls Sie ein wenig mehr zu Ping recherchieren wollen, kann ich Ihnen folgenden Artikel empfehlen: \url{https://openmaniak.com/ping.php}
\begin{enumerate}
	\item Recherchieren Sie die Syntax von \emph{ping}. Ein guter Anlaufpunkt wäre die Man-Page (\emph{man ping}) oder \url{https://linux.die.net/man/8/ping}.
\end{enumerate}
\bibliographystyle{plain}
\bibliography{sources}
\end{document}


