%%%%%%%%%%%%%%%%%%%%%%%%%%%%%%%%%%%%%%%%%%%%%%%%%%%%%%%%%%%%%%%%%%%%%%%%%%
%%LaTeX template for papers && theses									%%
%%Done by the incredible ||Z01db3rg||									%%
%%Under the do what ever you want license								%%
%%%%%%%%%%%%%%%%%%%%%%%%%%%%%%%%%%%%%%%%%%%%%%%%%%%%%%%%%%%%%%%%%%%%%%%%%% 

%start preamble
\documentclass[paper=a4,fontsize=11pt]{scrartcl}%kind of doc, font size, paper size
\usepackage[ngerman]{babel}%for special german letters etc			
%\usepackage{t1enc} obsolete, but some day we go back in time and could use this again
\usepackage[T1]{fontenc}%same as t1enc but better						
\usepackage[utf8]{inputenc}%utf-8 encoding, other systems could use others encoding
%\usepackage[latin9]{inputenc}			
\usepackage{amsmath}%get math done
\usepackage{amsthm}%get theorems and proofs done
\usepackage{graphicx}%get pictures & graphics done
\graphicspath{{pictures/}}%folder to stash all kind of pictures etc
\usepackage{amssymb}%symbolics for math
\usepackage{amsfonts}%extra fonts
\usepackage []{natbib}%citation style
\usepackage{caption}%captions under everything
\usepackage{listings}
\usepackage[titletoc]{appendix}
\numberwithin{equation}{section} 
\usepackage[printonlyused,withpage]{acronym}%how to handle acronyms
\usepackage{float}%for garphics and how to let them floating around in the doc
\usepackage{cclicenses}%license!
\usepackage{xcolor}%nicer colors, here used for links
\usepackage{wrapfig}%making graphics floated by text and not done by minipage
\usepackage{dsfont}
\usepackage{stmaryrd}
\usepackage{geometry}
\usepackage{hyperref}
\usepackage{fancyhdr}
\usepackage{menukeys}
\usepackage{multicol}

%settings colors for links
\hypersetup{
    colorlinks,
    linkcolor={blue!50!black},
    citecolor={blue},
    urlcolor={blue!80!black}
}

\definecolor{pblue}{rgb}{0.13,0.13,1}
\definecolor{pgreen}{rgb}{0,0.5,0}
\definecolor{pred}{rgb}{0.9,0,0}
\definecolor{pgrey}{rgb}{0.46,0.45,0.48}

%Header & Footers
\pagestyle{fancy}
\lhead{Netzwerke Übung (SoSe 2019)}
\rhead{FB 4 -- Angewandte Informatik\\ HTW-Berlin}
\lfoot{Übungsblatt 2 -- Netzwerkgrundlagen}
\cfoot{}
\fancyfoot[R]{\thepage}
\renewcommand{\headrulewidth}{0.4pt}
\renewcommand{\footrulewidth}{0.4pt}

\lstdefinestyle{Bash}{
  language=bash,
  showstringspaces=false,
  basicstyle=\small\sffamily,
  numbers=left,
  numberstyle=\tiny,
  numbersep=5pt,
  frame=trlb,
  columns=fullflexible,
  backgroundcolor=\color{gray!20},
  linewidth=0.9\linewidth,
  %xleftmargin=0.5\linewidth
}

\newlength\labelwd
\settowidth\labelwd{\bfseries viii.)}
\usepackage{tasks}
\settasks{counter-format =tsk[a].), label-format=\bfseries, label-offset=3em, label-align=right, label-width
=\labelwd, before-skip =\smallskipamount, after-item-skip=0pt}
\usepackage[inline]{enumitem}
\setlist[enumerate]{% (
labelindent = 0pt, leftmargin=*, itemsep=12pt, label={\textbf{\arabic*.)}}}

\pdfpkresolution=2400%higher resolution

%%here begins the actual document%%
\newcommand{\horrule}[1]{\rule{\linewidth}{#1}} % Create horizontal rule command with 1 argument of height

\DeclareMathOperator{\id}{id}

\begin{document}
\begin{center}
\Large{\textbf{Hausaufgaben Laborübung 2 -- Theorie, Netzwerkgrundlagen}}
\end{center}
\begin{center}\Large{\textbf{Zusammenfassung}}\end{center}
Ziel dieses Übungsblatts soll es sein Sie auf die kommende Laborübung gut vorzubereiten.\\
\textbf{Das Übungsblatt soll die Planung eines kleinen Netzwerkes theoretisch vorwegnehmen, sodass Sie Zeit haben sich einzuarbeiten, zu recherchieren und mögliche Probleme bzw. Fragen festzuhalten.}\\
Weiterhin dient das Hausaufgabenblatt der Aneignung zu Theorie und Grundlagen einiger Linux-Tools die in der kommenden Laborübung verwendet werden.
\paragraph{Inhalt:}
\begin{itemize}
	\item Grundlagen Laborhardware
	\item Netzwerktopologien
	\item Grundlegendes zu IP-Adressen
	\item Einige Kommandozeilenwerkzeuge für die Netzwerkadministration (\emph{ip}, \emph{ifconfig})
	\item Ping
\end{itemize}
Eine gute Anlaufstelle sind die Wikis der bekannteren Linux-Distributionen, wie das Debian- oder das Arch-Linux-Wiki.
\begin{itemize}
	\item \url{https://wiki.debian.org/}
	\item \url{https://www.debian.org/doc/manuals/debian-reference/index.en.html}
	\item \url{https://wiki.archlinux.org/}
\end{itemize}

\begin{center}\Large{\textbf{Aufgabe A - Planung des physischen Netzes}}\end{center}\vskip0.25in
Sie planen in Vierergruppen die Netzinfrastruktur für ein kleines LAN mit je vier Rasp\-berry Pis.
\begin{itemize}
	\item[1.)] Machen Sie sich die Funktion der einzelnen Rechner- \& Netzwerkkomponenten klar.
\begin{itemize}
    \item Raspberry Pi \& Peripherie (Monitor, Tastatur etc.)
    \item Netzwerkkabel -- Aufgabe im NW
    \item Switch -- Aufgabe im NW \& Einordnung ins OSI-Modell
    \item Ethernet-Port -- physikalisches Netzwerkinterface
\end{itemize}
	
	\item[2.)] Recherchieren Sie mithilfe der folgender Links was eine Netzwerktopologie ist.
	\begin{itemize}
		\item \url{https://www.elektronik-kompendium.de/sites/net/0503281.htm}
		\item \url{https://en.wikipedia.org/wiki/Network_topology}
		\item \url{https://www.lifewire.com/computer-network-topology-817884}
	\end{itemize}
	\item[3.)] Wählen Sie eine geeignete Netztopologie und skizzieren Sie diese mit geeigneten Symbolen.\\ \textbf{Hinweis:} Unter \url{http://iacis.org/iis/2008/S2008_967.pdf} finden Sie auf S. 241 eine Möglichkeit, wie dies aussehen könnte.\\
	Ordnen Sie die Geräte auf der Skizze so an, wie sie auch vor ihnen im Raum bzw. auf dem Tisch angeordnet sein sollten.
	\item[4.)] Planen Sie die Netzkonfiguration
\begin{tasks}[counter-format=(tsk[r])](1)	
	\task~ Recherchieren Sie kurz was eine IP-Adresse ist (zunächst genügt ein grobes Verständnis). Welche Aufgabe haben diese Adressen in einem Netzwerk?\\
	\textbf{Hinweis:} Ein guter Start wäre: \url{https://de.wikipedia.org/wiki/IP-Adresse}
	\task~ Momentan werden vor allem \emph{IPv4} und \emph{IPv6} als Netzwerkschichtprotokolle genutzt. Recherchieren Sie nur \textbf{kurz} einige wichtige Unterschiede zwischen \emph{IPv4} und \emph{IPv6}.
	\task~ Recherchieren Sie was eine Subnetzmaske ist und wofür diese gebraucht wird.
	\task~ Bestimmte IP-Adressbereiche werden nicht ins Internet weitergeleitet, sie werden als private IP-Adressen bezeichnet. Diese Adressen gibt es sowohl unter \emph{IPv4} als auch unter \emph{IPv6}. Recherchieren Sie, welche IP-Adressbereiche nicht ins Internet geroutet werden.
	\task~ Wählen Sie beispielhaft eine Netzwerkadresse (IP-Addresse -- ip address) und Subnetzmaske (subnet mask) für einen möglichst kleinen IP-Adressbereich, der genau für vier Raspberry Pis ausreicht. Als Hilfestellung können Sie folgenden Link nutzen:
	\begin{itemize}
		\item \url{https://www.calculator.net/ip-subnet-calculator.html}
		\item \url{https://www.tunnelsup.com/subnet-calculator/}
	\end{itemize}
\end{tasks} 
\end{itemize}

\begin{center}\Large{\textbf{Aufgabe B -- Tools}}\end{center}\vskip0.25in
Um den Übungsbetrieb etwas effizienter nutzen zu können, sollen Sie sich zunächst mit den Standardwerkzeugen der Netzwerkadministration vertraut machen. Mithilfe der Werkzeugsammlungen \emph{iproute2} sowie \emph{net-tools} wird dies in der Regel unter Linux und Unix-Betriebssystemen bewerkstelligt.
\begin{enumerate}
	\item Im letzten Übungsblatt haben Sie bereits das Rechtemodell kennengelernt. Verschiedene NutzerInnen haben verschiedene Rechte -- für die Konfiguration des Systems sollen im allgemeinen nur bestimmte Nutzer zuständig sein. Recherchieren Sie welche Rechte der \emph{root}-User hat und was das Kommando \emph{sudo} in diesem Zusammenhang leistet.
	\item In Betriebssystemen gibt es verschiedene Dienste/ Hintergrunddienste (Daemons genannt), die die Verwaltung des Systems in Teilen organisieren. Da Raspbian das Betriebssystem auf den Raspberry Pis ist, kommt Systemd zum Einsatz. \footnote{Eigentlich war Systemd als Alternative des System-V Init-Daemons gedacht, hat aber über die Zeit immer mehr Funktionalitäten bekommen.}
	\begin{tasks}
		\task~ Recherchieren Sie einige wichtige Dienste, die durch Systemd gesteuert werden.
		\task~ Systemd verfügt über die Möglichkeit bestimmte Dienste zu starten, stopen, etc. Recherchieren Sie wie der entsprechende Befehl lautet. Das Wiki bzw. die Man-Page ist eine gute Anlaufstelle!\\
		Notieren Sie sich die Syntax Wort für Wort, sowie die Bedeutung jedes Wortes (Tokens). 
		\task~ Wichtige Dienste für die nächste Laborübungen sind der Networking-Service und DHCP. Notieren Sie sich:
		\begin{itemize}
			\item[i] Wie der Status eines Daemons abgefragt werden kann.
			\item[ii] Wie ein Daemon gestartet, gestoppt werden kann.
			\item[iii] Wie ein Daemon permanent eingeschaltet bzw. ausgeschaltet werden kann (d.h. auch nach einem Neustart automatisch gestartet werden kann.)
		\end{itemize}
	\end{tasks}
	\item Übliche Befehle zum Einrichten von Netzwerkadaptern sind \emph{ifconfig} (BSD \emph{net-tools}) oder auch \emph{ip} aus der Werkzeugsammlung \emph{iproute2}. Der Befehl \emph{ifconfig} gilt in manchen Linux-Distributionen als veraltet (In BSD, Solaris etc. ist dies nicht der Fall!). Recherchieren Sie kurz, worin sich beide Tools-Sammlungen unterscheiden und notieren Sie sich wesentliche Unterschiede.\\
	Digital Ocean hat ein schönes HowTo dazu: \url{goo.gl/w1MN5x}
 %https://www.digitalocean.com/community/tutorials/how-to-use-iproute2-tools-to-manage-network-configuration-on-a-linux-vps
 % goo.gl/w1MN5x
 	\item Bringen Sie in Erfahrung, wie Sie die Konfiguration bereits existierende Netzwerkkonfigurationen mit den Tools \emph{ip} und \emph{ifconfig} in Erfahrung bringen.
	\item Recherchieren und notieren Sie sich, wie mithilfe des Befehls \emph{ip addr} Netzwerkadapter(n) eine (oder mehrere) IP-Adressen und Subnetzmasken zugewiesen wird. Wie wird dies mit \emph{ifconfig} gehandhabt. (Auch hier gilt: Notieren Sie sich das Kommando sowie dessen Bedeutung Wort/Schrittweise)!
	\item Recherchieren Sie, wie Sie die IP-Konfiguration in einer Datei festlegen und speichern können, sodass diese weiterhin nach einem Neustart gültig ist.\\
	\textbf{Achtung:} Bedenken Sie für welches Betriebssystem diese Konfiguration erfolgen soll!
	\begin{tasks}(1)
		\task~ In welcher Datei wird die Konfiguration abgelegt?
		\task~ Welcher User kann auf diese Datei zugreifen?
		\task~ Notieren Sie sich, wie eine Konfiguration beispielhaft aussieht und was die einzelnen Zeilen bedeuten!
	\end{tasks}
\end{enumerate}

\begin{center}\Large{\textbf{Aufgabe C -- Ping}}\end{center}\vskip0.25in
Um festzustellen ob eine Verbindung funktionstüchtig ist, wird oftmals das Tool \emph{ping} genutzt. D.h. \emph{ping} analysiert ob Datenpakete überhaupt und wie viele Pakete von einem Host (bspw. Ihrem Rechner) zu einem Ziel (wie etwa der Webserver der HTW-Berlin) gelangen. Falls Sie ein wenig mehr zu Ping recherchieren wollen, kann ich Ihnen folgenden Artikel empfehlen: \url{https://openmaniak.com/ping.php}
\begin{enumerate}
	\item Recherchieren Sie die Syntax von \emph{ping}. Ein guter Anlaufpunkt wäre die Man-Page (\emph{man ping}) oder \url{https://linux.die.net/man/8/ping}.
	\item \textbf{Optional:} Arbeiten Sie folgendes Tutorial durch: \url{https://www.thegeekstuff.com/2009/11/ping-tutorial-13-effective-ping-command-examples/}
\end{enumerate}
\end{document}
