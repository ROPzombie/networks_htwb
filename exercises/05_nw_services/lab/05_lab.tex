%start preamble
\documentclass[paper=a4,fontsize=11pt]{scrartcl}%kind of doc, font size, paper size

\usepackage{fontspec}
\defaultfontfeatures{Ligatures=TeX}
%\setsansfont{Liberation Sans}
\usepackage{polyglossia}	
\setdefaultlanguage[spelling=new, babelshorthands=true]{german}

\usepackage{amsmath}%get math done
\usepackage{amsthm}%get theorems and proofs done
\usepackage{graphicx}%get pictures & graphics done
\graphicspath{{pictures/}}%folder to stash all kind of pictures etc
\usepackage{amssymb}%symbolics for math
\usepackage{amsfonts}%extra fonts
\usepackage{caption}%captions under everything
\usepackage{listings}
\usepackage[titletoc]{appendix}
\numberwithin{equation}{section} 
\usepackage{float}%for garphics and how to let them floating around in the doc
\usepackage{wrapfig}%making graphics floated by text and not done by minipage
\usepackage{hyperref}
\usepackage{fancyhdr}
\usepackage{xcolor}%nicer colors, here used for links
\usepackage{csquotes}
\usepackage{enumitem}

\usepackage[backend=biber,style=alphabetic,
citestyle=alphabetic]{biblatex} %biblatex mit biber laden
\addbibresource{sources.bib}

%settings colors for links
\hypersetup{
    colorlinks,
    linkcolor={blue!50!black},
    citecolor={blue},
    urlcolor={blue!80!black}
}

\definecolor{pblue}{rgb}{0.13,0.13,1}
\definecolor{pgreen}{rgb}{0,0.5,0}
\definecolor{pred}{rgb}{0.9,0,0}
\definecolor{pgrey}{rgb}{0.46,0.45,0.48}

%Header & Footers
\pagestyle{fancy}
\lhead{Netzwerke -- Übung\\Sommersemester 2021}
\rhead{FB 4 -- Angewandte Informatik\\Hochschule für Technik und Wirtschft Berlin}
\lfoot{Übungsblatt 05 -- Netzwerkdienste}
\cfoot{}
\fancyfoot[R]{\thepage}
\renewcommand{\headrulewidth}{0.4pt}
\renewcommand{\footrulewidth}{0.4pt}

\lstdefinestyle{Bash}{
  language=bash,
  showstringspaces=false,
  basicstyle=\small\sffamily,
  numbers=left,
  numberstyle=\tiny,
  numbersep=5pt,
  frame=trlb,
  columns=fullflexible,
  backgroundcolor=\color{gray!20},
  %linewidth=0.9\linewidth,
  %xleftmargin=0.5\linewidth
}

%%here begins the actual document%%
\newcommand{\horrule}[1]{\rule{\linewidth}{#1}} % Create horizontal rule command with 1 argument of height

\DeclareMathOperator{\id}{id}

\begin{document}
\begin{center}
\Large{\textbf{Übungsblatt 5 -- Netzwerkdienste}}
\end{center}


\begin{center}\Large{\textbf{Aufgabe A -- DHCP}}\end{center}\vskip0.25in

\begin{enumerate}
	\item Auf dem \emph{freeBSD} mit grafischer Oberfläche läuft bereits DHCP für das Interface \emph{em0} im Bridge bzw. NAT-Mode. Dies nutzen wird für den ersten Teil.\\
	Überprüfen sie, ob der DHCP-Client auf diesem System ordnungsgemäß läuft.
	\item Lassen sie sich vom DHCP-Server für das \emph{freeBSD} eine neue IP-Adresse geben. Beobachten sie zeitgleich via Wireshark welche Nachrichten hierfür via BOOTP ausgetauscht werden.
	\begin{enumerate}[label=(\alph*)]
		\item Wie lautet die Sender-Adresse des DHCP-Client?
		\item Warum nutzt der DHCP-Cleint seine Sender-Adresse?
		\item An welche Ziel-IP-Adresse hat der DHCP-Client seine Nachrichten versandt?
		\item An welche Ziel-MAC-Adresse sendet der DHCP-Client seine Nachrichten?
		\item An welche Ziel-IP-Adresse hat der DHCP-Server seine Nachrichten versandt?
		\item An welche Ziel-MAC-Adresse sendet der DHCP-Server seine Nachrichten?
		\item Welche IP-Adresse wurde dem Client vom Server angeboten?
		\item Welche Lease-Time wurde durch den DHCP-Server angeboten?
		\item Welche IP-Adresse wählte und sendete der Client für die Antworten des DHCP-Servers?
		\item Welcher IP-Adresse bestätigte (acknowledges) der DHCP-Server dem DHCP-Client?
	\end{enumerate}
	\item Zeichen sie ein Sequenzdiagramm der IP-Adressvergabe die sie via Wireshark aufgezeichnet haben.
\end{enumerate}

\begin{center}\Large{\textbf{Aufgabe B -- DHCP II}}\end{center}\vskip0.25in
Im Folgenden soll ein weiteres Netzwerk unseren zwei bestehenden hinzugefügt werden. Das neueste Netzwerk soll jedoch dynamisch und automatisch IP-Adressen vergeben. Demnach haben wir zwei statische Netzwerke und ein durch DHCP organisiertes Netzwerk.
\begin{enumerate}
	\item Vorbereitend:
	\begin{enumerate}[label=(\alph*)]
		\item Legen sie ein weiteres \emph{Host-Only-Network} in virtualBox an. Dies soll ebenfalls in der IPv4-Range $172.16.X.Y$ liegen und maximal $/24$ sein.\\
		\textbf{Achtung:} Beim Erstellen nicht das Feld DHCP Server aktivieren, wir bauen einen eigenen DHCP-Server.
		\item Fügen sie dem \emph{freeBSD}-Router eine weitere Netzwerkschnittstelle für das eben angelegte Netzwerk hinzu.
		\item Klonen sie mindestens ein \emph{freeBSD} ohne grafische Oberfläche und geben dieser Maschine ebenfalls Zugang zum neu erstellten Netzwerk.
	\end{enumerate}
	\item Der DHCP-Server ist bereits vorinstalliert, d.h. wir können direkt mit der Konfiguration starten. Kopieren sie zunächst die die Datei \path{/usr/local/etc/dhcpd.conf.example} nach \emph{/usr/local/etc/dhcpd.conf.example}
	\item Passen diese Datei für ihre Bedürfnisse an, sodass dieser Rechner als DHCP-Server arbeitet.
	\item Passen sie den DHCP-Client entsprechend an, sodass die VM ohne grafische Oberfläche eine IP-Adresse beziehen kann.
	\item Starten sie den DHCP-Server als Daemon.
	\item Testen sie, ob der DHCP-Client eine IP-Adresse bekommt und Zugang ins Internet hat. Schneiden sie den Traffic in Wireshark mit, um dies zu überprüfen.
	\item Falls alles funktioniert hat, tragen sie den DHCP-Server persistent in die \emph{rc.conf} ein!
\end{enumerate}

\begin{center}\Large{\textbf{Aufgabe C -- Domain Name System (DNS) I}}\end{center}\vskip0.25in
\begin{enumerate}
	\item DNS-Requests:
\begin{enumerate}[label=(\alph*)]
		\item Fragen sie mit jedem der Kommando der Hausaufgaben jeweils einmal einen Hostnamen (bspw. \url{www.htw-berlin.de}), einen Domainnamen (htw-berlin.de) und eine IP-Adresse (bspw. 141.45.5.100) ab.
		\item Schauen sie sich die Ausgabe von \emph{dig} bei der Abfrage der IP-Adresse genauer an -- dort werden sie in der \enquote{Question Section} sehen, dass nach dem A-Resource-Record mit dem Namen 141.45.5.100 gefragt wurde. Wenn Sie den Namen zu dieser IP-Adresse suchen -- welchen Resource-Record müssen sie anstelle des A-Records erfragen? 
		\item In welcher Form müssen sie dann die IP-Adresse angeben? (Test mit \texttt{dig -t <record-type> <richtiges-format-ip-adresse>}).
		\item Denken sie sich einen Domainnamen aus, den es wahrscheinlich geben könnte, welcher aber in den letzten Stunden nicht aufgelöst worden ist. Erfragen sie diesen Namen zweimal kurz hintereinander via \emph{dig} und vergleichen sie die beiden Ausgaben. Worin unterscheiden sich beide Einträge? Falls eine größere zeitliche Differenz vorhanden ist, worin liegt die Ursache?
		\item Erfragen sie mit \emph{dig} und \emph{nslookup} den zuständigen Mail-Server für die Domain \url{htw-berlin.de}.
		\item Erzwingen sie mit \emph{dig} und \emph{nslookup} eine Namensauflösung ohne den Standard-DNS-Server des Betriebssystems, sondern mit einem öffentlichen Nameserver (bspw.: 9.9.9.9) erfolgt. Testen sie dies am Besten zuerst mit \emph{dig}, da dieses Werkzeug immer den genutzten Namensserver angibt.
	\end{enumerate}
	\item DNS-Resolver: Das Listing zeigt die \enquote{resolv.conf} eines Servers. 
	\begin{lstlisting}[style=Bash, language=Bash]
nameserver 141.45.3.100
search f4.htw-berlin.de
\end{lstlisting} \label{dns}
Was bedeuten die Einträge mit den Schlüsselwörtern: \enquote{nameserver} und \enquote{search}?
\end{enumerate}

\begin{center}\Large{\textbf{Aufgabe D -- Domain Name System (DNS) II}}\end{center}\vskip0.25in
Im Folgenden soll ein DNS-Server für eine eigene Domain aufgesetzt werden.
\begin{enumerate}
	\item Sichern sie zunächst ihre VM als Snapshot an, falls etwas schiefgehen sollte!
	\item Im Moodle-Kurs sind einige Beispieldateien hinterlegt, wie eine Zone konfiguriert werden kann, hinterlegt. 
	\item Mit dem Kommando \texttt{rndc-confgen -a} können sie einen Schlüssel für die Administration der \emph{rndc (Remote Name Daemon Control)} Werkzeuge hinterlegen.
	\item Da dieser Schlüssel für den Nutzer \emph{bind} und nicht \emph{root} laufen soll, muss noch die Zugehörigkeit im Dateisystem angepasst werden:
	\begin{lstlisting}[style=Bash, language=Bash]
sudo chown root:bind /usr/local/etc/namedb/rndc.key
sudo chmod 640 /usr/local/etc/namedb/rndc.key
\end{lstlisting}
	\item Der Bind-Server ist bereits vorinstalliert. Mit der Zeile \texttt{named\_enable="YES"} in der \path{rc.conf} kann dieser aktiviert werden. Falls der Dienst direkt gestartet werden soll kann dies mit \texttt{service named onestart} geschehen.
	\item Mithilfe von \emph{tail} können sie schauen, ob \emph{rndc} korrekt läuft:
			\begin{lstlisting}[style=Bash, language=Bash]
tail /var/log/messages
Oct 22 15:11:11 ns1 named[1161]: ------------------------------
Oct 22 15:11:11 ns1 named[1161]: BIND 9 is maintained by Internet Systems Consortium,
Oct 22 15:11:11 ns1 named[1161]: Inc. (ISC), a non-profit 501(c)(3) public-benefit
Oct 22 15:11:11 ns1 named[1161]: corporation.  Support and training for BIND 9 are
Oct 22 15:11:11 ns1 named[1161]: available at https://www.isc.org/support
Oct 22 15:11:11 ns1 named[1161]: ------------------------------
Oct 22 15:11:11 ns1 named[1161]: command channel listening on 127.0.0.1#953
Oct 22 15:11:11 ns1 named[1161]: command channel listening on ::1#953
Oct 22 15:11:11 ns1 named[1161]: all zones loaded
Oct 22 15:11:11 ns1 named[1161]: running
		\end{lstlisting}
		\item Zentrale Stell für die Konfiguration ist die Datei \path{/usr/local/etc/namedb/named.conf}. Diese Datei ist bereits vorhanden. Der Bind-Server läuft zunächst nur lokal, daher muss der Eintrag \texttt{listen-on} angepasst werden, sodass auch ihre IP-Adressen hinterlegt sind.
		\item Der oben erzeugte Schlüssel \path{rndc.key} muss noch in \emph{named.conf} eingetragen werden, da wir sonst den Binder-Server nicht administrieren können. Mit Folgender Modifikation kann der Schlüssel eingetragen werden:
		\begin{lstlisting}[style=Bash, language=Bash]
include "/usr/local/etc/namedb/rndc.key";
 
controls {
        inet 127.0.0.1 allow { localhost; } keys { "rndc-key"; };
};
\end{lstlisting}
Falls der DNS-Server von außen administriert werden soll, können an dieser Stelle auch andere IPs hinterlegt werden.			
	\item Der DNS-Server ist nun vorbereitet! Jetzt müssen die Zonen angelegt werden. Im Moodle-Kurs habe ich eine kleinere Beispielkonfiguration hinterlegt.\\
	Bevor sie die Einträge ihrer \texttt{named.conf} anpassen, machen sie sich die Bedeutung Folgender Einträge klar:
	\begin{enumerate}
		\item ~\\
\begin{lstlisting}[style=Bash, language=Bash]
options { 
	directory "/usr/local/etc/named/working"; 
	forwarders { 62.104.191.241; 62.104.196.134; };
	listen-on port 53 { 127.0.0.1; 172.16.0.1; };
	allow-query { 127.0/16;  };
	cleaning-interval 120;
	notify no;
};
\end{lstlisting}
		\item ~\\
\begin{lstlisting}[style=Bash, language=Bash]
zone "localhost" in {
	type master;
	file "localhost.zone";
};
\end{lstlisting}
		\item ~\\
\begin{lstlisting}[style=Bash, language=Bash]
zone "0.0.127.in-addr.arpa" in {
	type master;
	file "127.0.0.zone";
};
\end{lstlisting}
	\item ~\\
\begin{lstlisting}[style=Bash, language=Bash]
zone "crypto.all." in {
	type master;
	file "crypto.zone";
};
\end{lstlisting}
		\item ~\\
\begin{lstlisting}[style=Bash, language=Bash]
zone "0.16.172.in-addr.arpa" in {
	type master;
	file "172.16.0.zone";
};
	\end{lstlisting}
	\item ~\\
\begin{lstlisting}[style=Bash, language=Bash]
zone "." in {
	type hint;	
	file "root.hint";
};
\end{lstlisting}
	\end{enumerate}
	\item Eine \emph{zone}-Datei ist wie folgt aufgebaut:
\begin{lstlisting}[style=Bash, language=Bash]
$TTL 2D
@		IN SOA	@   root (
				42		; serial (d. adams)
				1D		; refresh
				2H		; retry
				1W		; expiry
				2D )		; minimum

		IN NS		@
		IN A		127.0.0.1
\end{lstlisting}

\begin{lstlisting}[style=Bash, language=Bash]
$TTL 2D
crypto.all.	IN SOA		mceliece   root.localhost. (
				2001091300	; serial
				1D		; refresh
				2H		; retry
				1W		; expiry
				2D )		; minimum

		IN NS		diffie
		IN MX		10 hellman

diffie		IN A		172.16.0.1
hellman	    IN A		172.16.0.24
peikerts		IN A		172.16.0.23
bernstein		IN A		172.16.0.25

www		IN CNAME	mceliece
ftp		IN CNAME	www
\end{lstlisting}

\begin{lstlisting}[style=Bash, language=Bash]
$TTL 2D
0.16.172.in-addr.arpa.  IN SOA  mceliece.crypto.all.  root.localhost. (
				2001091300	; serial
				1D		; refresh
				2H		; retry
				1W		; expiry
				2D )		; minimum

		IN NS		mceliece.cyrpto.all.

11		IN PTR		diffie.crypto.all.
24		IN PTR		hellman.crypto.all.
23		IN PTR		peikerts.crypto.all.
25		IN PTR  	bernstein.crypto.all.
\end{lstlisting}
Wie sind diese Dateien zu interpretieren?
	\item Anhand des obigen Beispiels können sie sich eine eigene Zone ausdenken, oder mein Beispiel anpassen. Das heißt die Zonen des Beispiel für die \path{named.conf} müssen angepasst oder ähnlich erstellt werden.
	\item Tragen sie die nun erstellten Zonen in die \path{named.conf} Datei ein!
	\item Legen sie die Dateien entsprechend ihrer Konfiguration im richtigen Ordner ab (bspw. unter \path{/usr/local/etc/named/working}).
	\item Überprüfen sie den Status des DNS-Servers mit dem Kommando \texttt{rndc status}. Gibt es Fehlermeldungen?
	\item Mit dem Kommando \texttt{rndc reload} können sie den Server die neuen Konfigurationen geben, sodass dieser den Server neu lädt.
		\item Tragen sie ihren DNS-Server in der \emph{/etc/resolv.conf} als weiteren Name-Server ein.
		\item Ein Rechner eines anderen Netzes sollte nun VMs ihrer Zone via namen auflösen können. D.h. statt:
		\begin{lstlisting}[style=Bash, language=Bash]
ping -c 1 172.16.0.25 
		\end{lstlisting}
		könnten sie nun:
		\begin{lstlisting}[style=Bash, language=Bash]
ping -c 1 bernstein.cyrpto.all
		\end{lstlisting}
		erreichen.
\end{enumerate}


\end{document}