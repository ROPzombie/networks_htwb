%start preamble
\documentclass[paper=a4,fontsize=11pt]{scrartcl}%kind of doc, font size, paper size

\usepackage{fontspec}
\defaultfontfeatures{Ligatures=TeX}
%\setsansfont{Liberation Sans}
\usepackage{polyglossia}	
\setdefaultlanguage[spelling=new, babelshorthands=true]{german}
\usepackage{csquotes}
		
\usepackage{amsmath}%get math done
\usepackage{amsthm}%get theorems and proofs done
\usepackage{graphicx}%get pictures & graphics done
\graphicspath{{pictures/}}%folder to stash all kind of pictures etc
\usepackage{hyperref}%for links to web
\usepackage{amssymb}%symbolics for math
\usepackage{amsfonts}%extra fonts
\usepackage []{natbib}%citation style
\usepackage{caption}%captions under everything
\usepackage{listings}
\usepackage[titletoc]{appendix}
\numberwithin{equation}{section} 
\usepackage[printonlyused,withpage]{acronym}%how to handle acronyms
\usepackage{float}%for garphics and how to let them floating around in the doc
\usepackage{cclicenses}%license!
\usepackage{xcolor}%nicer colors, here used for links
\usepackage{wrapfig}%making graphics floated by text and not done by minipage
\usepackage{dsfont}
\usepackage{stmaryrd}
\usepackage{geometry}
\usepackage{fancyhdr}
\usepackage{menukeys}
\usepackage{enumitem}



%settings colors for links
\hypersetup{
    colorlinks,
    linkcolor={blue!50!black},
    citecolor={blue},
    urlcolor={blue!80!black}
}

\definecolor{pblue}{rgb}{0.13,0.13,1}
\definecolor{pgreen}{rgb}{0,0.5,0}
\definecolor{pred}{rgb}{0.9,0,0}
\definecolor{pgrey}{rgb}{0.46,0.45,0.48}

\pagestyle{fancy}
\lhead{Netzwerke -- Übung\\Wintersemester 2020/21}
\rhead{FB 4 -- Angewandte Informatik\\ HTW-Berlin}
\lfoot{Übungsblatt 1 -- Grundlagen Linux \& Shell}
\cfoot{}
\fancyfoot[R]{\thepage}
\renewcommand{\headrulewidth}{0.4pt}
\renewcommand{\footrulewidth}{0.4pt}

\lstdefinestyle{Bash}{
  language=bash,
  showstringspaces=false,
  basicstyle=\small\sffamily,
  numbers=left,
  numberstyle=\tiny,
  numbersep=5pt,
  frame=trlb,
  columns=fullflexible,
  backgroundcolor=\color{gray!20},
  linewidth=0.9\linewidth,
  %xleftmargin=0.5\linewidth
  upquote=true,
  columns=fullflexible,
  literate={*}{{\char42}}1
         {-}{{\char45}}1
}


%%here begins the actual document%%
\newcommand{\horrule}[1]{\rule{\linewidth}{#1}} % Create horizontal rule command with 1 argument of height


\DeclareMathOperator{\id}{id}

\title{	
\normalfont \normalsize 
\textsc{Übungsblatt 00 -- Setup}
}
\begin{document}
\begin{center}\Large{\textbf{Übungsblatt 0 -- Setup}}\end{center}

Da dieses Semester etwas anders verlaufen wird, starten wir den Übungsbetrieb mit dem Aufsetzen einer virtuellen Maschine. 
\begin{center}\Large{\textbf{Aufgabe A -- VirtualBox}}\end{center}\vskip0.25in
Ziel dieser Aufgabe ist es, dass Sie ein einsatzfähiges Setup haben.
\begin{enumerate}
	\item Installieren Sie VirtualBox auf Ihren Rechner. Im Moodlekurs finden Sie für die gängigen Betriebssystem Links.\\
	Falls Ihre Hardware altersschwach oder keine Virtualisierung unterstützt, schreiben sie mir im Moodle-Forum.
	\item Im Kurs liegt eine PDF-Datei, die den Import einer VM unter virtualBox erklärt (\emph{Tutorial VirtualBox}). Lesen sie das Dokument aufmerksam.
	\item Laden Sie das bereitgestellte Image zum Importieren der VM herunter. Importieren sie das Image, wie im PDF-Tutorial angegeben. Falls sie Platz auf einer \emph{SSD} haben, sollte die VM hier für die Nutzung abgelegt werden.\\
	(Link zum Image liegt im Moodle-Kurs)
	\item Wenn alles geklappt hat, starten sie die VM, loggen Sie als Nutzer \emph{student} mit dem Passwort \emph{student} ein.
	\item Schmökern sie ein bisschen herum, das Betriebssystem ist recht Benutzerfreundlich. Mit dem Befehlt \emph{startx} können sie die grafische Oberfläche starten.
	\item Da sich der Großteil des Semesters auf der Konsole (Shell oder Terminal) abspielt, sollten sie an dessen Umgang gewöhnt werden. Beginnen Sie mit dem Übungsblatt \emph{Aufgabenblatt Basics -- Shell}.
	\item Wenn sie bereits Fit sind und bessere Lösungen als virtualBox bevorzugen:
	\begin{itemize}
		\item Sie können sich ein \emph{Docker}-Image bauen, welches die gleichen Funktionalitäten bietet.
		\item \emph{bhyve} oder \emph{libvirt} sind sehr leistungsfähige Alternativen.
		\item Jedoch müssen Sie viele Tools selber installieren.
	\end{itemize}
\end{enumerate}

\end{document}